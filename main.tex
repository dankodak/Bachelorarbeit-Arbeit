%%% File encoding: UTF-8
%%% äöüÄÖÜß  <-- keine deutschen Umlaute hier? UTF-faehigen Editor verwenden!

%%% Magic Comments zum Setzen der korrekten Parameter in kompatiblen IDEs
% !TeX encoding = utf8
% !TeX program = pdflatex 
% !TeX spellcheck = de_DE
% !BIB program = biber

\documentclass[bachelor,german]{hgbthesis}
% Zulässige Optionen in [..]: 
%   Typ der Arbeit: diploma, master (default), bachelor, internship 
%   Hauptsprache: german (default), english
%%%----------------------------------------------------------

\RequirePackage[utf8]{inputenc}		% bei der Verw. von lualatex oder xelatex entfernen!

\graphicspath{{images/}}    % Verzeichnis mit Bildern und Grafiken
\logofile{logo}				% Logo-Datei = images/logo.pdf (\logofile{}, wenn kein Logo gewünscht)
\bibliography{references}  	% Biblatex-Literaturdatei (references.bib)
\usepackage{acronym}
\usepackage{amsthm}
%%%----------------------------------------------------------
% Angaben für die Titelei (Titelseite, Erklärung etc.)
%%%----------------------------------------------------------

%%% Einträge für ALLE Arbeiten: -----------------------------
\title{Elliptische Randwertprobleme mit gewichteter Kernkollokation}
\author{Daniel Koch}
\programname{Mathematik}
\placeofstudy{Stuttgart}
\dateofsubmission{2017}{02}{28}	% {YYYY}{MM}{DD}

%%% Zusätzlich für eine Bachelorarbeit: ---------------------
\thesisnumber{XXXXXXXXXX-A}   % Stud-ID, z.B. 1310238045-A  
% (A = 1. Bachelorarbeit)
\semester{Sommersemester 2018} 
\coursetitle{Einführung in die Tiefere Problematik 1} 
\advisor{Prof. Dr. Bernard Haasdonk}

%%% Restriktive Lizenformel anstatt CC (nur für Typ master) -
%\strictlicense

\theoremstyle{plain}
\newtheorem{theorem}{Satz}[chapter]
\theoremstyle{definition}
\newtheorem{definition}[theorem]{Definition}
\theoremstyle{definition}
\newtheorem{example}[theorem]{Beispiel}
\theoremstyle{remark}
\newtheorem{remark}[theorem]{Bemerkung}

%%%----------------------------------------------------------
\begin{document}
%%%----------------------------------------------------------

%%%----------------------------------------------------------
\frontmatter                    % Titelei (röm. Seitenzahlen)
%%%----------------------------------------------------------

\maketitle
\tableofcontents

% !TEX root = ../maturaarbeit.tex
\chapter{Vorwort}\label{chap:vorwort}
\begin{enumerate}
\item Weshalb haben Sie (als Einzelperson, ggf. als Gruppe) dieses Thema gewählt?
\item Verdankungen: Personen, Institutionen usw., welche Sie unterstützt haben
\end{enumerate}
Also zum Beispiel: Ich habe diese Maturaarbeit begonnen, weil ich \LaTeX\ lernen wollte.
\newpage % Optional. Ggf. weglassen
\include{front/kurzfassung}		
\include{front/abstract}			

%%%----------------------------------------------------------
\mainmatter          % Hauptteil (ab hier arab. Seitenzahlen)
%%%----------------------------------------------------------

\chapter{Einleitung}
\label{cha:Einleitung}


Unser Ziel ist es Lösungen von \acp{PDE} zu approximieren. Diese sind allgemein gegeben durch:
\begin{align*}
L u(x) &= f(x), x \in \Omega \\
B u(x) &= g(x), x \in \partial \Omega
\end{align*}
, wobei $\Omega \subset \mathbb{R}^n$, $L$ ein linearer, beschränkter Differentialoperator und $B$ ein linearer, beschränkter Auswertungsoperator ist.\\
Für den größten Teil dieser Arbeit werden wir folgende \ac{PDE} im $\mathbb{R}^2$ betrachten:
\begin{align*}
\Delta u(x) &= f(x), x \in \Omega \\
u(x) &= 0 , x \in \partial \Omega
\end{align*}

Es genügt die Nullrandbedingung zu betrachten, da jede \ac{PDE} auf eine mit Nullrandbedingung umgeformt werden kann.\\
HIER KOMMT DIE BEGRÜNDUNG!!

Wir wollen zur Approximation der \ac{PDE} einen interpolierenden Ansatz wählen. Dazu müssen wir die Interpolation zunächst verallgemeinern.

\begin{definition}
Sei $\Omega \subset \mathbb{R}^n$ eine nicht leere Menge, $\mathcal{H}$ ein Hilbertraum mit Funktionen $f:\Omega \rightarrow \mathbb{R}$, $u \in \mathcal{H}$  und $\Lambda_N := \{\lambda_1, \dots, \lambda_N\} \subset \mathcal{H}'$ eine Menge von linearen, stetigen und linear unabhängigen Funktionalen. Dann ist eine Funktion $s_u \in \mathcal{H}$ der gesuchte Interpolant von $u$, wenn gilt, dass
\begin{align*}
\lambda_i(u) = \lambda_i(s_u) , 1\le i \le N
\end{align*}
\end{definition}

\begin{example}
\begin{itemize}
\item
Sei $\Omega \subset \mathbb{R}^d$ ,$X_N := \{x_1, \dots, x_N\} \subset \Omega$ eine Menge von Punkten und $\mathcal{H}$ ein Hilbertraum mit Funktionen , in dem die Punktauswertungfunktionale $\delta_{x_i}(f) = f(x_i), 1\le i \le N$  stetig sind. Dann bekommen wir die Standardinterpolation mit $\Lambda_N := \{\delta_{x_1}, \dots,\delta_{x_N}\} \subset \mathcal{H}'$.
\begin{align*}
s(x_i) = \delta_{x_i}(s) = \delta_{x_i}(s_u) = s_u(x_i), 1\le i \le N
\end{align*}
\item
Mit $\lambda_i := \delta_{x_i} \circ D^a$ für einen Multiindex $a \in \mathbb{N}_0^d$ erhält man noch zusätzliche Informationen über die Ableitung der Funktion.
\item
Sei eine \ac{PDE} gegeben:
\begin{align*}
L u(x) &= f(x), x \in \Omega \\
B u(x) &= g(x), x \in \partial \Omega
\end{align*}
Sei $X_N \subset \Omega$ eine Menge an Kollokationspunkten. Dann möchten wir, dass $s_u$ die \ac{PDE} in den Punkten $X_N$ erfüllt, also:
\begin{align*}
L s_u(x_i) &= L u(x_i) = f(x_i), x_i \in \Omega \\
B s_u(x_i) &= B u(x_i) = g(x_i), x_i \in \partial \Omega
\end{align*}
\end{itemize}
\end{example}

Wir müssen einen geeigneten Ansatz wählen um das Interpolationsproblem zu lösen, also einen $N$-dimensionalen Unterraum $V_N := \text{span}\{\nu_1, \dots, \nu_N\} \subset \mathcal{H}$ und fordern, dass $s_u \in V_N$, also 
\begin{align*}
s_u(x) := \sum_{j=1}^N \alpha_j \nu_j(x), x \in \Omega, \alpha \in \mathbb{R}^N
\end{align*}
Also lassen sich die Interpolationsbedingungen schreiben als:
\begin{align*}
\lambda_i(u) = \lambda_i(s_u) = \sum_{j=1}^N \alpha_j \lambda_i(\nu_j)
\end{align*}
Diese lassen sich auch als lineares Gleichungssystem $A_\Lambda \alpha = b$ schreiben  mit $(A_\Lambda)_{i,j} := \lambda_i(\nu_j), b_i := \lambda_i(u)$.

\section{Standardkollokation}
Wir suchen jetzt nach geeigneten Ansatzfunktionen und einem Hilbertraum, in dem die Auswertungs- und Differentialfunktionale stetig sind. Dies führt uns zur Definition von Kern Funktionen mit denen wir einen Hilbertraum konstruieren werden, der uns das Geforderte liefern wird.

\begin{definition}
\label{Kern}
Sei $\Omega$ eine nicht leere Menge. Ein reeller Kern auf $\Omega$ ist eine symmetrische Funktion $K: \Omega \times \Omega \rightarrow \mathbb{R}$.\\
Für alle $N \in \mathbb{N}$ und für eine Menge $X_N = \{x_i\}_{i=1}^N$ ist die Kern Matrix (oder Gram'sche Matrix) $A:= A_{K,X_N} \in \mathbb{R}^{N \times N}$  definiert als $A:=[K(x_i, x_j)]_{i,j=1}^N$.\\
Ein Kern $K$ heißt \ac{PD} auf $\Omega$, wenn für alle $N \in \mathbb{N}$ und alle Mengen $X_N$ mit paarweise verschiedenen Elementen $x_{i=1}^N$ gilt, dass die Kern Matrix positiv definit ist. Der Kern $K$ heißt \ac{SPD}, falls die Kern Matrix strikt positiv definit ist.
\end{definition}

\begin{example} Sei $\Omega \subset \mathbb{R}^n$. Dann sind folgende Funktionen Kerne auf $\Omega$:\\
\begin{itemize}
\item $K(x,y) := \exp(-\gamma \|x-y\|),\gamma > 0$
\item $K(x,y) := (x,y)$
\end{itemize}
\end{example}

\begin{remark}
$\Omega$ kann eine beliebige Menge sein, es kann also auch ein Kern auf Strings oder Bildern definiert werden. Dies führt noch zu vielfältigeren Anwendungen.
\end{remark}

Wir kommen mit der Definition von Kernen direkt zu den gesuchten Hilberträumen.

\begin{definition}[Reproduzierender Kern Hilbertraum]
Sei $\Omega$ eine nicht leere Menge und $\mathcal{H}$ ein Hilbertraum mit Funktionen $f:\Omega \rightarrow \mathbb{R}$ und Skalarprodut $(\cdot, \cdot)_\mathcal{H}$. Dann nennt man $\mathcal{H}$ \ac{RKHR} auf $\Omega$, wenn eine Funktion $K:\Omega \times \Omega \rightarrow \mathbb{R}$ existiert, sodass
\begin{enumerate}
\item $K(\cdot, x) \in \mathcal{H}$ für alle $x \in \Omega$
\item $(f, K(\cdot,x))_\mathcal{H} = f(x)$ für alle $ x \in \Omega$, $f \in \mathcal{H}$
\end{enumerate}
\end{definition}

\begin{remark}
Die Funktion $K$ in einem \ac{RKHR} ist tatsächlich ein Kern nach Definition \ref{Kern}, welcher sogar positiv definit ist.
\end{remark}

Bei Interpolationsproblemen kommen wir jedoch aus der anderen Richung und haben zunächst einen Kern $K$ gegeben und wollen damit eine Funktion approximieren. Also stellt sich die Frage ob zu jedem Kern $K$ ein \ac{RKHR} existiert. Diese wird durch folgenden Satz beantwortet:

\begin{theorem}[Moore, Aronszajn]
Sei $\Omega$ eine nicht leere Menge und $K:\Omega \times \Omega \rightarrow \mathbb{R}$ ein positiv definiter Kern. Dann existiert genau ein \ac{RKHR} $\mathcal{H}_K (\Omega)$ mit reproduzierendem Kern $K$.
\end{theorem}
\begin{proof}
SIEHE SKRIPT!
\end{proof}

Mit diesem Wissen können wir uns erste Eigenschaften von \ac{RKHR} anschauen:

\begin{theorem}
\label{stetig}
Sei $\Omega$ eine nicht leere Menge und $\mathcal{H}$ ein Hilbertraum mit Funktionen $f: \Omega \rightarrow \mathbb{R}$. Dann gilt:
\begin{enumerate}
\item \label{stetig1} $\mathcal{H}$ ist genau dann ein \ac{RKHR}, wenn die Auswertungsfunktionale stetig sind.
\item \label{stetig2} Wenn $\mathcal{H}$ ein \ac{RKHR} mit Kern $K$ ist, dann ist $K(\cdot,x)$ der Riesz-Repräsentant des Funktionals $\delta_x \in \mathcal{H}'$.
\end{enumerate}
\end{theorem}

\begin{proof}
\begin{enumerate}
\item Für alle $f \in \mathcal{H}$ und alle $x \in \Omega$ gilt:
\begin{align*}
|f(x)| &= |(f, K(\cdot,x))_\mathcal{H}| \le \|f\|_\mathcal{H}\|K(\cdot,x)\|_\mathcal{H}\\
&= \|f\|_\mathcal{H} \sqrt{(K(\cdot,x),K(\cdot,x))_\mathcal{H}} = \|f\|_\mathcal{H} \sqrt{K(x,x)}
\end{align*}
, wobei für die erste und die letzte Gleichung die Reproduzierbarkeit des Kerns benutzt wurde.

Sei $\mathcal{H}$ ein \ac{RKHR}. Dann gilt mit dem eben gezeigten:
\begin{align*}
|\delta_x(f)| &= |f(x)| \le \|f\|_\mathcal{H} \sqrt{K(x,x)}\\
\Leftrightarrow \frac{|\delta_x(f)|}{\|f\|_\mathcal{H}} &\le \sqrt{K(x,x)}
\end{align*}
Also ist $\delta_x$ beschränkt und damit stetig.

Für die andere Richtung nehmen wir an, dass $\delta_x  \in \mathcal{H}'$ für alle $x \in \Omega$. Also existiert ein Riesz-Repräsentant $\nu_{\delta_x} \in \mathcal{H}$. Definieren wir $K(\cdot,x):= \nu_{\delta_x}$, dann ist $K$ ein Kern. Es ist klar, dass $K(\cdot,x) \in \mathcal{H}$ und nach der Definition des Riesz-Repräsentanten gilt:
\begin{align*}
(f, K(\cdot,x))_\mathcal{H} = (f, \nu_{\delta_x})_\mathcal{H} = \delta_x(f) = f(x)
\end{align*}
\item Die Behauptung folgt sofort aus der Reproduzierbarkeit von $K$, da $(f, K(\cdot,x))_\mathcal{H}= f(x)$ für alle $x \in \Omega$ und alle $f \in \mathcal{H}$ gilt.
\end{enumerate}
\end{proof}



Wir haben also gesehen, dass in einem \ac{RKHR} $\mathcal{H}_K$ die Auswertungsfunktionale stetig sind. Da wir uns mit Differentialgleichungen beschäftigen, wollen wir auch Ableitungen auswerten. Dafür benötigen wir, dass diese ebenfalls in $\mathcal{H}_K$ liegen.

\begin{theorem}
Sei $k \in \mathbb{N}$. Angenommen $\Omega \subset \mathbb{R}^n$ ist offen, K ist \ac{SPD} auf $\Omega$ und $K \in C^{2k}(\Omega \times \Omega)$. Dann gilt für alle Multiindizes $a \in \mathbb{N}_0^d$ mit $|a| \le k$ und alle $x \in \Omega$, dass $D_2^a K(\cdot , x) \in \mathcal{H}_K(\Omega)$.

Außerdem gilt für alle $f \in \mathcal{H}_K(\Omega)$:
\begin{align*}
D^a f(x) = \left(f,D_2^a K(\cdot,x)\right)_{\mathcal{H}_K(\Omega)}
\end{align*}
und damit dass $\lambda := \delta_x \circ D^a$ stetig ist.
\end{theorem}

\begin{proof}
BEWEIS IST LANG

Der Beweis der Stetigkeit von $\lambda := \delta_x \circ D^a$ verläuft komplett analog zum Beweis von \ref{stetig}.\ref{stetig1}.
\end{proof}

In Satz \ref{stetig} haben wir gesehen, wie der Riesz-Repräsentant des Auswertungsfunktionals aussieht. Dies wollen wir jetzt auf alle Funktionale verallgemeinern.

\begin{theorem}
\label{Riesz}
Sei $K$ ein \ac{SPD} Kern auf $\Omega \neq \emptyset$. Sei $\lambda \in \mathcal{H}_K (\Omega)'$. Dann ist $\lambda^y K(\cdot,y) \in \mathcal{H}_k(\Omega)$ und es gilt $\lambda(f) = \left(f,\lambda^y K(\cdot,y)\right)_{\mathcal{H}_K(\Omega)}$ für alle $f \in \mathcal{H}_K(\Omega)$, also ist $\lambda^y K(\cdot,y)$ der Riesz-Repräsentant von $\lambda$.
\end{theorem}

\begin{proof}
Da $\lambda \in \mathcal{H}_K(\Omega)$ existiert ein Riesz-Repräsentant $\nu_\lambda \in \mathcal{H}_K(\Omega)$ mit $\lambda (f) = \left(f, \nu_\lambda\right)_{\mathcal{H}_K(\Omega)}$. Außerdem ist $f_x(y) := K(x,y)$ für alle $x \in \Omega$ eine Funktion in $\mathcal{H}_K (\Omega)$. Dann bekommen wir:
\begin{align*}
\lambda^y K(x,y) = \lambda(f_x) = \left(f_x, \nu_\lambda\right)_{\mathcal{H}_K (\Omega)} = \left(K(\cdot,x), \nu_\lambda\right)_{\mathcal{H}_K (\Omega)} = \nu_\lambda(x)
\end{align*}
Damit gilt $\nu_\lambda(\cdot) = \lambda^y K(\cdot,y)$ und auch $\lambda^y K(\cdot,y) \in \mathcal{H}_K (\Omega)$.
\end{proof}

Jetzt fehlt nur noch die lineare Unabhängigkeit aller verwendeten Funktionale. Zunächst die der Auswertungsfunktionale:
\begin{theorem}
Sei $\Omega$ eine nicht leere Menge und $\mathcal{H}$ ein \ac{RKHR} mit Kern $K$. Dann sind $\{\delta_x,x\in \Omega\}$ genau dann linear unabhängig, wenn $K$ \ac{SPD} ist.
\end{theorem}

\begin{proof}
Seien $\lambda_1, \dots, \lambda_n \in \mathcal{H}'$ und $\nu_{\lambda_1},\dots, \nu_{\lambda_n} \in \mathcal{H}$ die dazugehörigen Riesz Repräsentanten. Diese sind linear abhängig, wenn ein $\alpha \in \mathbb{R}^n$ existiert mit $\lambda := \sum_{i=1}^n \alpha_i \lambda_i = 0$, also dass $\lambda(f) = 0$ für alle $f \in \mathcal{H}$. Das gilt genau dann, wenn die Riesz Repräsentanten linear abhängig sind, da
\begin{align*}
0 = \lambda(f) = \sum_{i=1}^n \alpha_i \lambda_i(f) = \sum_{i=1}^n \alpha_i \left( \nu_{\lambda_i},f\right)_\mathcal{H} = \left( \sum_{i=1}^n \alpha_i \nu_{\lambda_i}, f \right)_\mathcal{H}
\end{align*}
Also gilt nach \ref{stetig}.\ref{stetig2}, dass $\{\delta_x,x\in \Omega\}$ genau dann linear unabhängig sind, wenn $\{K(\cdot,x) , x \in \Omega\}$ linear unabhängig sind.

Um die strikte positive Definitheit nachzuweisen, betrachten wir die Matrix $A=[K(x_i, x_j)]_{i,j=1}^N$ für paarweise unterschiedliche Punkte $x_i, 1 \le i \le N$. Sei also $\beta \in \mathbb{R}^n, \beta \neq 0$. Dann gilt:
\begin{align*}
\beta^T A \beta &= \sum_{i,j=1}^n \beta_i \beta_j K(x_i, x_j)\\
&= \sum_{i,j=1}^n \beta_i  \beta_j \left(K(\cdot, x_i),K(\cdot,x_j)\right)_\mathcal{H}\\
&= \left( \sum_{i=1}^n \beta_i K(\cdot,x_i),\sum_{j=1}^n \beta_j K(\cdot, x_j) \right)_\mathcal{H}\\
&= \left\| \sum_{i=1}^n \beta_i K(\cdot, x_i) \right\|_\mathcal{H}^2 > 0
\end{align*}
Für die letzte strikte Ungleichung benötigen wir die lineare Unabhängigkeit. Also gilt, dass K \ac{SPD} ist, wenn $\{\delta_x,x\in \Omega\}$ linear unabhängig sind.
\end{proof}

Und jetzt die der Auswertungen der Ableitungen:

\begin{theorem}
\label{linUn}
Sei $K$ ein translationsinvarianter Kern auf $\mathbb{R}^d$, also $K(x,y) = \Phi (x-y)$ für alle $x,y \in \mathbb{R}^d$. Sei $k \in \mathbb{N}$ und angenommen, dass $\Phi \in L_1(\mathbb{R}^d) \cap C^{2k}(\mathbb{R}^d)$. Sei $a_1, \dots, a_N \in \mathbb{N}_0^d$ mit $|a_i| \le k$ und sei $X_N \subset \mathbb{R}^d$. Angenommen, dass $a_i \neq a_j$, wenn $x_i = x_j$, dann sind die Funktionale $\Lambda_N := \{\lambda_1, \dots, \lambda_N\}, \lambda_i := \delta_{x_i} \circ D^{a_i}$ linear unabhängig in $\mathcal{H}_K(\mathbb{R}^d)$.
\end{theorem}

\begin{proof}
BUCH ODER SKRIPT
\end{proof}


Damit haben wir alle nötigen Werkzeuge um die Interpolation durchzuführen. Wir haben Ansatzfunktionen $K$, den dazugehörigen Hilbertraum $\mathcal{H}_K(\Omega)$, die Stetigkeit und lineare Unabhängigkeit aller benötigten Operatoren. Jetzt müssen wir nur noch einen geeigneten Ansatz wählen.
\subsection{Symmetrische Kollokation}
Sei wieder $\Omega \subset \mathbb{R}^n$ offen und beschränkt, $L,B$ lineare Differentialoperatoren, $K$ ein positiv definiter Kern und folgendes Problem gegeben:
\begin{align*}
L u(x) &= f(x), x \in \Omega \\
B u(x) &= g(x), x \in \partial \Omega
\end{align*}
Für ein $N \in \mathbb{N}$ betrachten wir die Menge $X_N \subset \Omega$, die wir in $N_{in}$ Punkte im Inneren und $N_{bd}$ Punkte auf dem Rand aufteilen. Also haben wir die beiden Mengen
\begin{align*}
X_{in} &= X_N \cap \Omega\\
X_{bd} &= X_N \cap \partial \Omega
\end{align*}
Wir definieren die Menge $\Lambda_N = \{\lambda_1, \dots, \lambda_N\}$ an linearen Funktionalen mit
\begin{align*}
\lambda_i =
\begin{cases}
\delta_{x_i} \circ L & x_i \in \Omega\\
\delta_{x_i} \circ B & x_i \in \partial \Omega
\end{cases}
\end{align*}
Wir wissen aus Satz \ref{stetig}, dass in $\mathcal{H}_K(\Omega)$ alle $\lambda_i$ stetig und aus Satz \ref{linUn}, dass sie linear unabhängig sind. Als Ansatzfunktionen, also den Unterraum $V_N \subset \mathcal{H}_K(\Omega)$, wählen wir die Riesz Repräsentanten der $\lambda_i$:
\begin{align*}
V_N &= \text{span} \{\lambda_1^y K(x,y), \dots , \lambda_N^y K(x,y)\}\\
&= \text{span} \{(\delta_{x_1} \circ L)^y K(x,y), \dots, (\delta_{x_{N_in}} \circ L)^y K(x,y), (\delta_{x_{N_{in} + 1}} \circ B)^y K(x,y), \dots, (\delta_{x_{N}} \circ B)^y K(x,y)\}\\
&=: \text{span} \{\nu_1, \dots, \nu_N\}
\end{align*}
, wobei der hochgesetzte Index y bedeutet, dass der Operator auf das zweite Argument angewandt wird.

Damit bekommen wir folgenden Interpolanten:
\begin{align*}
s_u(x) &= \sum_{j=1}^N \alpha_j \lambda_j^y K(x,y)\\
&= \sum_{j=1}^{N_{in}} \alpha_j (\delta_{x_j} \circ L)^y K(x,y) + \sum_{j=N_{in}}^{N} \alpha_j (\delta_{x_j} \circ L)^y K(x,y)
\end{align*}
Die $\alpha_j$ erhält man als Lösung des \ac{LGS} $A \alpha = b$ mit $A_{i,j} := (\nu_i,\nu_j)_{\mathcal{H}_K}$, da
\begin{align*}
\left<\lambda_i, s_u\right> = \left< \lambda_i, \sum_{j=1}^N \alpha_j \nu_j \right> = \sum_{j=1}^N \alpha_j \left< \lambda_i,\nu_j\right> \overset{\ref{Riesz}}{=} \sum_{j=1}^N \alpha_j \left(\nu_j, \nu_i\right)
\end{align*},
 also
\begin{align*}
\begin{pmatrix}
A_{L,L} & A_{L,B} \\ 
A_{L,B}^T & A_{B,B}
\end{pmatrix} 
\alpha =
\begin{pmatrix}
b_L \\ 
b_B
\end{pmatrix} 
\end{align*}
mit
\begin{align*}
(A_{L,L})_{i,j} &= (\delta_{x_i} \circ L)^x(\delta_{x_j} \circ L)^y K(x,y),x_i, x_j \in X_{in}\\
(A_{L,B})_{i,j} &= (\delta_{x_i} \circ L)^x(\delta_{x_j} \circ B)^y K(x,y),x_i \in X_{in}, x_j \in X_{bd} \\
(A_{B,B})_{i,j} &= (\delta_{x_i} \circ B)^x(\delta_{x_j} \circ B)^y K(x,y), x_i, x_j \in X_{bd}
\end{align*}
und
\begin{align*}
(b_L)_i &= f(x_i), x_i \in X_{in}\\
(b_B)_i &= g(x_i), x_i \in X_{bd}
\end{align*}
Das \ac{LGS} ist lösbar, da A offensichtlich symmetrisch und positiv definit ist, da:
\begin{align*}
\alpha^T A \alpha = \sum_{i,j = 1}^N \alpha_i \alpha_j (\nu_i, \nu_j)_{\mathcal{H}_K} = \left(\sum_{i=1}^N \alpha_i \nu_i, \sum_{j=1}^N \alpha_j \nu_j \right)_{\mathcal{H}_K} = \left\| \sum_{i=1}^N \alpha_i \nu_i \right\|_{\mathcal{H}_K}^2 > 0
\end{align*}
Für die letzte Abschätzung benutzen wir die lineare Unabhängigkeit der Funktionale aus Satz \ref{linUn}.
\subsection{Nicht-Symmetrische Kollokation}
\include{chapters/abschlussarbeit}
\include{chapters/latex}
\include{chapters/abbildungen}
\include{chapters/mathematik}
\include{chapters/literatur}
\include{chapters/drucken}
\chapter{Zusammenfassung und Ausblick}
\label{cha:schluss}

In diesem Kapitel fassen wir die wichtigsten Erkenntnisse dieser Arbeit zusammen und zeigen interessante weiterführende Fragen auf.

Wir haben in Kapitel \ref{cha:Grundlagen} und \ref{cha:Standardkollokation} die bekannte Theorie der Kernkollokation vorgestellt, welche wir in Kapitel \ref{cha:Gewichtet} um Gewichtsfunktionen erweitert haben. Diese könnte man noch weiter untersuchen, auch im Hinblick darauf, Gewichtsfunktionen für beliebige Gebiete mit möglichst wenig Singularitäten in ihren Ableitungen zu finden. Als Ansatz dafür bietet sich die in Kapitel \ref{sec:andereGewicht} experimentell vorgestellte Methode an, in der wir eine Gewichtsfunktion nur auf einer Umgebung gegeben hatten.

Daraufhin haben wir in Kapitel \ref{cha:Implementierung} eine Implementierung der theoretisch hergeleiteten Verfahren vorgestellt, welche wir in Kapitel \ref{cha:NumerischeTests} ausgewertet haben. Dort haben wir sehr verschiedene Ergebnisse für unterschiedliche \acp{PDE} erhalten. Insbesondere haben die gewichteten Verfahren teilweise sehr schlechte Ergebnisse geliefert. Diese Beobachtung haben wir auf die Singularitäten der partiellen Ableitungen der Gewichtsfunktionen zurückgeführt.

Wir haben außerdem verschiedene Möglichkeiten der Kollokationspunktwahl untersucht. Dabei haben wir festgestellt, dass eine Greedy-Punktwahl bezüglich der Anzahl der Kollokationspunkte die geschickteste der vorgestellten Wahlmöglichkeiten ist. Hier wäre es noch interessant einen anderen Fehlerschätzer zur Punktwahl zu benutzen, beispielsweise mit den $C^1, C^2, \dots$ Normen. Außerdem wäre eine Greedy-Punktwahl wünschenswert, bei der die Punkte auf dem Rand nicht festgesetzt sind, sondern auch erst nach und nach gesetzt werden. Dafür könnte man auf dem Rand einen zweiten Fehlerschätzer einführen und diesen mithilfe einer Konstante gegen den Fehler im Inneren abwägen.

Wir haben die Kernkollokation zum Großteil nur in zwei Dimensionen getestet, haben aber gesehen, dass einer Umsetzung in höheren Dimensionen nichts im Wege steht. Diesen Teil haben wir allerdings nur kurz angeschnitten und man könnte dort noch weitere Experimente durchführen, um auf eine größere Allgemeingültigkeit zu schließen.

%%%----------------------------------------------------------
\appendix                                            % Anhang 
%%%----------------------------------------------------------

\chapter{Technische Informationen}
\label{app:TechnischeInfos}
\begin{acronym}[PDE]
\acro{PDE}{partielle Differentialgleichung}
\acrodefplural{PDE}[PDEs]{partielle Differentialgleichungen}
\end{acronym}

\begin{acronym}[PD]
\acro{PD}{positiv definit}
\end{acronym}

\begin{acronym}[SPD]
\acro{SPD}{strikt positiv definit}
\end{acronym}

\begin{acronym}[RKHR]
\acro{RKHR}{reproduzierender Kern Hilbert Raum}
\end{acronym}

\begin{acronym}[LGS]
\acro{LGS}{lineares Gleichungssystem}
\end{acronym}	% Technische Ergänzungen
\include{back/anhang_b}	% Inhalt der CD-ROM/DVD
\include{back/anhang_c}	% Chronologische Liste der Änderungen
\chapter*{Abkürzungsverzeichnis}
\label{app:abkuerzungen}
\addcontentsline{toc}{chapter}{Abkürzungsverzeichnis}

\begin{acronym}[PDE]
\acro{PDE}{partielle Differentialgleichung}
\acrodefplural{PDE}[PDEs]{partielle Differentialgleichungen}
\end{acronym}

\begin{acronym}[PD]
\acro{PD}{positiv definit}
\end{acronym}

\begin{acronym}[SPD]
\acro{SPD}{strikt positiv definit}
\end{acronym}

\begin{acronym}[RKHR]
\acro{RKHR}{reproduzierender Kern Hilbert Raum}
\end{acronym}

\begin{acronym}[LGS]
\acro{LGS}{lineares Gleichungssystem}
\end{acronym}

\begin{acronym}[oBdA]
\acro{oBdA}{ohne Beschränkung der Allgemeinheit}
\end{acronym}	% Quelltext dieses Dokuments

%%%----------------------------------------------------------
\MakeBibliography                        % Quellenverzeichnis
%%%----------------------------------------------------------

%%% Messbox zur Druckkontrolle ------------------------------
\include{back/messbox}

%%%----------------------------------------------------------
\end{document}
%%%----------------------------------------------------------