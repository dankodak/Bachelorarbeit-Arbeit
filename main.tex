%%% File encoding: UTF-8
%%% äöüÄÖÜß  <-- keine deutschen Umlaute hier? UTF-faehigen Editor verwenden!

%%% Magic Comments zum Setzen der korrekten Parameter in kompatiblen IDEs
% !TeX encoding = utf8
% !TeX program = pdflatex 
% !TeX spellcheck = de_DE
% !BIB program = biber

\documentclass[bachelor,german]{hgbthesis}
% Zulässige Optionen in [..]: 
%   Typ der Arbeit: diploma, master (default), bachelor, internship 
%   Hauptsprache: german (default), english
%%%----------------------------------------------------------

\RequirePackage[utf8]{inputenc}		% bei der Verw. von lualatex oder xelatex entfernen!

\graphicspath{{images/}}    % Verzeichnis mit Bildern und Grafiken
\logofile{logo}				% Logo-Datei = images/logo.pdf (\logofile{}, wenn kein Logo gewünscht)
\bibliography{references}  	% Biblatex-Literaturdatei (references.bib)
\usepackage{acronym}
\usepackage{amsthm}
\usepackage{pgfplots}
  \pgfplotsset{compat=newest}
  %% the following commands are needed for some matlab2tikz features
  \usetikzlibrary{plotmarks}
  \usetikzlibrary{arrows.meta}
  \usepgfplotslibrary{patchplots}
  \usepackage{grffile}
%%%----------------------------------------------------------
% Angaben für die Titelei (Titelseite, Erklärung etc.)
%%%----------------------------------------------------------

%%% Einträge für ALLE Arbeiten: -----------------------------
\title{Elliptische Randwertprobleme mit gewichteter Kernkollokation}
\author{Daniel Koch}
\programname{Mathematik}
\placeofstudy{Stuttgart}
\dateofsubmission{2017}{02}{28}	% {YYYY}{MM}{DD}

%%% Zusätzlich für eine Bachelorarbeit: ---------------------
\thesisnumber{XXXXXXXXXX-A}   % Stud-ID, z.B. 1310238045-A  
% (A = 1. Bachelorarbeit)
\semester{Sommersemester 2018} 
\coursetitle{Einführung in die Tiefere Problematik 1} 
\advisor{Prof. Dr. Bernard Haasdonk}

%%% Restriktive Lizenformel anstatt CC (nur für Typ master) -
%\strictlicense

\theoremstyle{plain}
\newtheorem{theorem}{Satz}[chapter]
\theoremstyle{definition}
\newtheorem{definition}[theorem]{Definition}
\theoremstyle{definition}
\newtheorem{example}[theorem]{Beispiel}
\theoremstyle{remark}
\newtheorem{remark}[theorem]{Bemerkung}

%%%----------------------------------------------------------
\begin{document}
%%%----------------------------------------------------------

%%%----------------------------------------------------------
\frontmatter                    % Titelei (röm. Seitenzahlen)
%%%----------------------------------------------------------

\maketitle
\tableofcontents

%% !TEX root = ../maturaarbeit.tex
\chapter{Vorwort}\label{chap:vorwort}
\begin{enumerate}
\item Weshalb haben Sie (als Einzelperson, ggf. als Gruppe) dieses Thema gewählt?
\item Verdankungen: Personen, Institutionen usw., welche Sie unterstützt haben
\end{enumerate}
Also zum Beispiel: Ich habe diese Maturaarbeit begonnen, weil ich \LaTeX\ lernen wollte.
\newpage % Optional. Ggf. weglassen
%\include{front/kurzfassung}		
\include{front/abstract}			

%%%----------------------------------------------------------
\mainmatter          % Hauptteil (ab hier arab. Seitenzahlen)
%%%----------------------------------------------------------

\chapter{Einleitung}
\label{cha:Einleitung}


Unser Ziel ist es Lösungen von \acp{PDE} zu approximieren. Diese sind allgemein gegeben durch:
\begin{align*}
L u(x) &= f(x), x \in \Omega \\
B u(x) &= g(x), x \in \partial \Omega
\end{align*}
, wobei $\Omega \subset \mathbb{R}^n$, $L$ ein linearer, beschränkter Differentialoperator und $B$ ein linearer, beschränkter Auswertungsoperator ist.\\
Für den größten Teil dieser Arbeit werden wir folgende \ac{PDE} im $\mathbb{R}^2$ betrachten:
\begin{align*}
\Delta u(x) &= f(x), x \in \Omega \\
u(x) &= 0 , x \in \partial \Omega
\end{align*}

Es genügt die Nullrandbedingung zu betrachten, da jede \ac{PDE} auf eine mit Nullrandbedingung umgeformt werden kann.\\
HIER KOMMT DIE BEGRÜNDUNG!!

Wir wollen zur Approximation der \ac{PDE} einen interpolierenden Ansatz wählen. Dazu müssen wir die Interpolation zunächst verallgemeinern.

\begin{definition}
Sei $\Omega \subset \mathbb{R}^n$ eine nicht leere Menge, $\mathcal{H}$ ein Hilbertraum mit Funktionen $f:\Omega \rightarrow \mathbb{R}$, $u \in \mathcal{H}$  und $\Lambda_N := \{\lambda_1, \dots, \lambda_N\} \subset \mathcal{H}'$ eine Menge von linearen, stetigen und linear unabhängigen Funktionalen. Dann ist eine Funktion $s_u \in \mathcal{H}$ der gesuchte Interpolant von $u$, wenn gilt, dass
\begin{align*}
\lambda_i(u) = \lambda_i(s_u) , 1\le i \le N
\end{align*}
\end{definition}

\begin{example}
\begin{itemize}
\item
Sei $\Omega \subset \mathbb{R}^d$ ,$X_N := \{x_1, \dots, x_N\} \subset \Omega$ eine Menge von Punkten und $\mathcal{H}$ ein Hilbertraum mit Funktionen , in dem die Punktauswertungfunktionale $\delta_{x_i}(f) = f(x_i), 1\le i \le N$  stetig sind. Dann bekommen wir die Standardinterpolation mit $\Lambda_N := \{\delta_{x_1}, \dots,\delta_{x_N}\} \subset \mathcal{H}'$.
\begin{align*}
s(x_i) = \delta_{x_i}(s) = \delta_{x_i}(s_u) = s_u(x_i), 1\le i \le N
\end{align*}
\item
Mit $\lambda_i := \delta_{x_i} \circ D^a$ für einen Multiindex $a \in \mathbb{N}_0^d$ erhält man noch zusätzliche Informationen über die Ableitung der Funktion.
\item
Sei eine \ac{PDE} gegeben:
\begin{align*}
L u(x) &= f(x), x \in \Omega \\
B u(x) &= g(x), x \in \partial \Omega
\end{align*}
Sei $X_N \subset \Omega$ eine Menge an Kollokationspunkten. Dann möchten wir, dass $s_u$ die \ac{PDE} in den Punkten $X_N$ erfüllt, also:
\begin{align*}
L s_u(x_i) &= L u(x_i) = f(x_i), x_i \in \Omega \\
B s_u(x_i) &= B u(x_i) = g(x_i), x_i \in \partial \Omega
\end{align*}
\end{itemize}
\end{example}

Wir müssen einen geeigneten Ansatz wählen um das Interpolationsproblem zu lösen, also einen $N$-dimensionalen Unterraum $V_N := \text{span}\{\nu_1, \dots, \nu_N\} \subset \mathcal{H}$ und fordern, dass $s_u \in V_N$, also 
\begin{align*}
s_u(x) := \sum_{j=1}^N \alpha_j \nu_j(x), x \in \Omega, \alpha \in \mathbb{R}^N
\end{align*}
Also lassen sich die Interpolationsbedingungen schreiben als:
\begin{align*}
\lambda_i(u) = \lambda_i(s_u) = \sum_{j=1}^N \alpha_j \lambda_i(\nu_j)
\end{align*}
Diese lassen sich auch als lineares Gleichungssystem $A_\Lambda \alpha = b$ schreiben  mit $(A_\Lambda)_{i,j} := \lambda_i(\nu_j), b_i := \lambda_i(u)$.

\section{Standardkollokation}
Wir suchen jetzt nach geeigneten Ansatzfunktionen und einem Hilbertraum, in dem die Auswertungs- und Differentialfunktionale stetig sind. Dies führt uns zur Definition von Kern Funktionen mit denen wir einen Hilbertraum konstruieren werden, der uns das Geforderte liefern wird.

\begin{definition}
\label{Kern}
Sei $\Omega$ eine nicht leere Menge. Ein reeller Kern auf $\Omega$ ist eine symmetrische Funktion $K: \Omega \times \Omega \rightarrow \mathbb{R}$.\\
Für alle $N \in \mathbb{N}$ und für eine Menge $X_N = \{x_i\}_{i=1}^N$ ist die Kern Matrix (oder Gram'sche Matrix) $A:= A_{K,X_N} \in \mathbb{R}^{N \times N}$  definiert als $A:=[K(x_i, x_j)]_{i,j=1}^N$.\\
Ein Kern $K$ heißt \ac{PD} auf $\Omega$, wenn für alle $N \in \mathbb{N}$ und alle Mengen $X_N$ mit paarweise verschiedenen Elementen $x_{i=1}^N$ gilt, dass die Kern Matrix positiv definit ist. Der Kern $K$ heißt \ac{SPD}, falls die Kern Matrix strikt positiv definit ist.
\end{definition}

\begin{example} Sei $\Omega \subset \mathbb{R}^n$. Dann sind folgende Funktionen Kerne auf $\Omega$:\\
\begin{itemize}
\item $K(x,y) := \exp(-\gamma \|x-y\|),\gamma > 0$
\item $K(x,y) := (x,y)$
\end{itemize}
\end{example}

\begin{remark}
$\Omega$ kann eine beliebige Menge sein, es kann also auch ein Kern auf Strings oder Bildern definiert werden. Dies führt noch zu vielfältigeren Anwendungen.
\end{remark}

Wir kommen mit der Definition von Kernen direkt zu den gesuchten Hilberträumen.

\begin{definition}[Reproduzierender Kern Hilbertraum]
Sei $\Omega$ eine nicht leere Menge und $\mathcal{H}$ ein Hilbertraum mit Funktionen $f:\Omega \rightarrow \mathbb{R}$ und Skalarprodut $(\cdot, \cdot)_\mathcal{H}$. Dann nennt man $\mathcal{H}$ \ac{RKHR} auf $\Omega$, wenn eine Funktion $K:\Omega \times \Omega \rightarrow \mathbb{R}$ existiert, sodass
\begin{enumerate}
\item $K(\cdot, x) \in \mathcal{H}$ für alle $x \in \Omega$
\item $(f, K(\cdot,x))_\mathcal{H} = f(x)$ für alle $ x \in \Omega$, $f \in \mathcal{H}$
\end{enumerate}
\end{definition}

\begin{remark}
Die Funktion $K$ in einem \ac{RKHR} ist tatsächlich ein Kern nach Definition \ref{Kern}, welcher sogar positiv definit ist.
\end{remark}

Bei Interpolationsproblemen kommen wir jedoch aus der anderen Richung und haben zunächst einen Kern $K$ gegeben und wollen damit eine Funktion approximieren. Also stellt sich die Frage ob zu jedem Kern $K$ ein \ac{RKHR} existiert. Diese wird durch folgenden Satz beantwortet:

\begin{theorem}[Moore, Aronszajn]
Sei $\Omega$ eine nicht leere Menge und $K:\Omega \times \Omega \rightarrow \mathbb{R}$ ein positiv definiter Kern. Dann existiert genau ein \ac{RKHR} $\mathcal{H}_K (\Omega)$ mit reproduzierendem Kern $K$.
\end{theorem}
\begin{proof}
SIEHE SKRIPT!
\end{proof}

Mit diesem Wissen können wir uns erste Eigenschaften von \ac{RKHR} anschauen:

\begin{theorem}
\label{stetig}
Sei $\Omega$ eine nicht leere Menge und $\mathcal{H}$ ein Hilbertraum mit Funktionen $f: \Omega \rightarrow \mathbb{R}$. Dann gilt:
\begin{enumerate}
\item \label{stetig1} $\mathcal{H}$ ist genau dann ein \ac{RKHR}, wenn die Auswertungsfunktionale stetig sind.
\item \label{stetig2} Wenn $\mathcal{H}$ ein \ac{RKHR} mit Kern $K$ ist, dann ist $K(\cdot,x)$ der Riesz-Repräsentant des Funktionals $\delta_x \in \mathcal{H}'$.
\end{enumerate}
\end{theorem}

\begin{proof}
\begin{enumerate}
\item Für alle $f \in \mathcal{H}$ und alle $x \in \Omega$ gilt:
\begin{align*}
|f(x)| &= |(f, K(\cdot,x))_\mathcal{H}| \le \|f\|_\mathcal{H}\|K(\cdot,x)\|_\mathcal{H}\\
&= \|f\|_\mathcal{H} \sqrt{(K(\cdot,x),K(\cdot,x))_\mathcal{H}} = \|f\|_\mathcal{H} \sqrt{K(x,x)}
\end{align*}
, wobei für die erste und die letzte Gleichung die Reproduzierbarkeit des Kerns benutzt wurde.

Sei $\mathcal{H}$ ein \ac{RKHR}. Dann gilt mit dem eben gezeigten:
\begin{align*}
|\delta_x(f)| &= |f(x)| \le \|f\|_\mathcal{H} \sqrt{K(x,x)}\\
\Leftrightarrow \frac{|\delta_x(f)|}{\|f\|_\mathcal{H}} &\le \sqrt{K(x,x)}
\end{align*}
Also ist $\delta_x$ beschränkt und damit stetig.

Für die andere Richtung nehmen wir an, dass $\delta_x  \in \mathcal{H}'$ für alle $x \in \Omega$. Also existiert ein Riesz-Repräsentant $\nu_{\delta_x} \in \mathcal{H}$. Definieren wir $K(\cdot,x):= \nu_{\delta_x}$, dann ist $K$ ein Kern. Es ist klar, dass $K(\cdot,x) \in \mathcal{H}$ und nach der Definition des Riesz-Repräsentanten gilt:
\begin{align*}
(f, K(\cdot,x))_\mathcal{H} = (f, \nu_{\delta_x})_\mathcal{H} = \delta_x(f) = f(x)
\end{align*}
\item Die Behauptung folgt sofort aus der Reproduzierbarkeit von $K$, da $(f, K(\cdot,x))_\mathcal{H}= f(x)$ für alle $x \in \Omega$ und alle $f \in \mathcal{H}$ gilt.
\end{enumerate}
\end{proof}



Wir haben also gesehen, dass in einem \ac{RKHR} $\mathcal{H}_K$ die Auswertungsfunktionale stetig sind. Da wir uns mit Differentialgleichungen beschäftigen, wollen wir auch Ableitungen auswerten. Dafür benötigen wir, dass diese ebenfalls in $\mathcal{H}_K$ liegen.

\begin{theorem}
Sei $k \in \mathbb{N}$. Angenommen $\Omega \subset \mathbb{R}^n$ ist offen, K ist \ac{SPD} auf $\Omega$ und $K \in C^{2k}(\Omega \times \Omega)$. Dann gilt für alle Multiindizes $a \in \mathbb{N}_0^d$ mit $|a| \le k$ und alle $x \in \Omega$, dass $D_2^a K(\cdot , x) \in \mathcal{H}_K(\Omega)$.

Außerdem gilt für alle $f \in \mathcal{H}_K(\Omega)$:
\begin{align*}
D^a f(x) = \left(f,D_2^a K(\cdot,x)\right)_{\mathcal{H}_K(\Omega)}
\end{align*}
und damit dass $\lambda := \delta_x \circ D^a$ stetig ist.
\end{theorem}

\begin{proof}
BEWEIS IST LANG

Der Beweis der Stetigkeit von $\lambda := \delta_x \circ D^a$ verläuft komplett analog zum Beweis von \ref{stetig}.\ref{stetig1}.
\end{proof}

In Satz \ref{stetig} haben wir gesehen, wie der Riesz-Repräsentant des Auswertungsfunktionals aussieht. Dies wollen wir jetzt auf alle Funktionale verallgemeinern.

\begin{theorem}
\label{Riesz}
Sei $K$ ein \ac{SPD} Kern auf $\Omega \neq \emptyset$. Sei $\lambda \in \mathcal{H}_K (\Omega)'$. Dann ist $\lambda^y K(\cdot,y) \in \mathcal{H}_k(\Omega)$ und es gilt $\lambda(f) = \left(f,\lambda^y K(\cdot,y)\right)_{\mathcal{H}_K(\Omega)}$ für alle $f \in \mathcal{H}_K(\Omega)$, also ist $\lambda^y K(\cdot,y)$ der Riesz-Repräsentant von $\lambda$.
\end{theorem}

\begin{proof}
Da $\lambda \in \mathcal{H}_K(\Omega)$ existiert ein Riesz-Repräsentant $\nu_\lambda \in \mathcal{H}_K(\Omega)$ mit $\lambda (f) = \left(f, \nu_\lambda\right)_{\mathcal{H}_K(\Omega)}$. Außerdem ist $f_x(y) := K(x,y)$ für alle $x \in \Omega$ eine Funktion in $\mathcal{H}_K (\Omega)$. Dann bekommen wir:
\begin{align*}
\lambda^y K(x,y) = \lambda(f_x) = \left(f_x, \nu_\lambda\right)_{\mathcal{H}_K (\Omega)} = \left(K(\cdot,x), \nu_\lambda\right)_{\mathcal{H}_K (\Omega)} = \nu_\lambda(x)
\end{align*}
Damit gilt $\nu_\lambda(\cdot) = \lambda^y K(\cdot,y)$ und auch $\lambda^y K(\cdot,y) \in \mathcal{H}_K (\Omega)$.
\end{proof}

Jetzt fehlt nur noch die lineare Unabhängigkeit aller verwendeten Funktionale. Zunächst die der Auswertungsfunktionale:
\begin{theorem}
Sei $\Omega$ eine nicht leere Menge und $\mathcal{H}$ ein \ac{RKHR} mit Kern $K$. Dann sind $\{\delta_x,x\in \Omega\}$ genau dann linear unabhängig, wenn $K$ \ac{SPD} ist.
\end{theorem}

\begin{proof}
Seien $\lambda_1, \dots, \lambda_n \in \mathcal{H}'$ und $\nu_{\lambda_1},\dots, \nu_{\lambda_n} \in \mathcal{H}$ die dazugehörigen Riesz Repräsentanten. Diese sind linear abhängig, wenn ein $\alpha \in \mathbb{R}^n$ existiert mit $\lambda := \sum_{i=1}^n \alpha_i \lambda_i = 0$, also dass $\lambda(f) = 0$ für alle $f \in \mathcal{H}$. Das gilt genau dann, wenn die Riesz Repräsentanten linear abhängig sind, da
\begin{align*}
0 = \lambda(f) = \sum_{i=1}^n \alpha_i \lambda_i(f) = \sum_{i=1}^n \alpha_i \left( \nu_{\lambda_i},f\right)_\mathcal{H} = \left( \sum_{i=1}^n \alpha_i \nu_{\lambda_i}, f \right)_\mathcal{H}
\end{align*}
Also gilt nach \ref{stetig}.\ref{stetig2}, dass $\{\delta_x,x\in \Omega\}$ genau dann linear unabhängig sind, wenn $\{K(\cdot,x) , x \in \Omega\}$ linear unabhängig sind.

Um die strikte positive Definitheit nachzuweisen, betrachten wir die Matrix $A=[K(x_i, x_j)]_{i,j=1}^N$ für paarweise unterschiedliche Punkte $x_i, 1 \le i \le N$. Sei also $\beta \in \mathbb{R}^n, \beta \neq 0$. Dann gilt:
\begin{align*}
\beta^T A \beta &= \sum_{i,j=1}^n \beta_i \beta_j K(x_i, x_j)\\
&= \sum_{i,j=1}^n \beta_i  \beta_j \left(K(\cdot, x_i),K(\cdot,x_j)\right)_\mathcal{H}\\
&= \left( \sum_{i=1}^n \beta_i K(\cdot,x_i),\sum_{j=1}^n \beta_j K(\cdot, x_j) \right)_\mathcal{H}\\
&= \left\| \sum_{i=1}^n \beta_i K(\cdot, x_i) \right\|_\mathcal{H}^2 > 0
\end{align*}
Für die letzte strikte Ungleichung benötigen wir die lineare Unabhängigkeit. Also gilt, dass K \ac{SPD} ist, wenn $\{\delta_x,x\in \Omega\}$ linear unabhängig sind.
\end{proof}

Und jetzt die der Auswertungen der Ableitungen:

\begin{theorem}
\label{linUn}
Sei $K$ ein translationsinvarianter Kern auf $\mathbb{R}^d$, also $K(x,y) = \Phi (x-y)$ für alle $x,y \in \mathbb{R}^d$. Sei $k \in \mathbb{N}$ und angenommen, dass $\Phi \in L_1(\mathbb{R}^d) \cap C^{2k}(\mathbb{R}^d)$. Sei $a_1, \dots, a_N \in \mathbb{N}_0^d$ mit $|a_i| \le k$ und sei $X_N \subset \mathbb{R}^d$. Angenommen, dass $a_i \neq a_j$, wenn $x_i = x_j$, dann sind die Funktionale $\Lambda_N := \{\lambda_1, \dots, \lambda_N\}, \lambda_i := \delta_{x_i} \circ D^{a_i}$ linear unabhängig in $\mathcal{H}_K(\mathbb{R}^d)$.
\end{theorem}

\begin{proof}
BUCH ODER SKRIPT
\end{proof}


Damit haben wir alle nötigen Werkzeuge um die Interpolation durchzuführen. Wir haben Ansatzfunktionen $K$, den dazugehörigen Hilbertraum $\mathcal{H}_K(\Omega)$, die Stetigkeit und lineare Unabhängigkeit aller benötigten Operatoren. Jetzt müssen wir nur noch einen geeigneten Ansatz wählen.
\subsection{Symmetrische Kollokation}
Sei wieder $\Omega \subset \mathbb{R}^n$ offen und beschränkt, $L,B$ lineare Differentialoperatoren, $K$ ein positiv definiter Kern und folgendes Problem gegeben:
\begin{align*}
L u(x) &= f(x), x \in \Omega \\
B u(x) &= g(x), x \in \partial \Omega
\end{align*}
Für ein $N \in \mathbb{N}$ betrachten wir die Menge $X_N \subset \Omega$, die wir in $N_{in}$ Punkte im Inneren und $N_{bd}$ Punkte auf dem Rand aufteilen. Also haben wir die beiden Mengen
\begin{align*}
X_{in} &= X_N \cap \Omega\\
X_{bd} &= X_N \cap \partial \Omega
\end{align*}
Wir definieren die Menge $\Lambda_N = \{\lambda_1, \dots, \lambda_N\}$ an linearen Funktionalen mit
\begin{align*}
\lambda_i =
\begin{cases}
\delta_{x_i} \circ L & x_i \in \Omega\\
\delta_{x_i} \circ B & x_i \in \partial \Omega
\end{cases}
\end{align*}
Wir wissen aus Satz \ref{stetig}, dass in $\mathcal{H}_K(\Omega)$ alle $\lambda_i$ stetig und aus Satz \ref{linUn}, dass sie linear unabhängig sind. Als Ansatzfunktionen, also den Unterraum $V_N \subset \mathcal{H}_K(\Omega)$, wählen wir die Riesz Repräsentanten der $\lambda_i$:
\begin{align*}
V_N &= \text{span} \{\lambda_1^y K(x,y), \dots , \lambda_N^y K(x,y)\}\\
&= \text{span} \{(\delta_{x_1} \circ L)^y K(x,y), \dots, (\delta_{x_{N_in}} \circ L)^y K(x,y), (\delta_{x_{N_{in} + 1}} \circ B)^y K(x,y), \dots, (\delta_{x_{N}} \circ B)^y K(x,y)\}\\
&=: \text{span} \{\nu_1, \dots, \nu_N\}
\end{align*}
, wobei der hochgesetzte Index y bedeutet, dass der Operator auf das zweite Argument angewandt wird.

Damit bekommen wir folgenden Interpolanten:
\begin{align*}
s_u(x) &= \sum_{j=1}^N \alpha_j \lambda_j^y K(x,y)\\
&= \sum_{j=1}^{N_{in}} \alpha_j (\delta_{x_j} \circ L)^y K(x,y) + \sum_{j=N_{in}}^{N} \alpha_j (\delta_{x_j} \circ L)^y K(x,y)
\end{align*}
Die $\alpha_j$ erhält man als Lösung des \ac{LGS} $A \alpha = b$ mit $A_{i,j} := (\nu_i,\nu_j)_{\mathcal{H}_K}$, da
\begin{align*}
\left<\lambda_i, s_u\right> = \left< \lambda_i, \sum_{j=1}^N \alpha_j \nu_j \right> = \sum_{j=1}^N \alpha_j \left< \lambda_i,\nu_j\right> \overset{\ref{Riesz}}{=} \sum_{j=1}^N \alpha_j \left(\nu_j, \nu_i\right)
\end{align*},
 also
\begin{align*}
\begin{pmatrix}
A_{L,L} & A_{L,B} \\ 
A_{L,B}^T & A_{B,B}
\end{pmatrix} 
\alpha =
\begin{pmatrix}
b_L \\ 
b_B
\end{pmatrix} 
\end{align*}
mit
\begin{align*}
(A_{L,L})_{i,j} &= (\delta_{x_i} \circ L)^x(\delta_{x_j} \circ L)^y K(x,y),x_i, x_j \in X_{in}\\
(A_{L,B})_{i,j} &= (\delta_{x_i} \circ L)^x(\delta_{x_j} \circ B)^y K(x,y),x_i \in X_{in}, x_j \in X_{bd} \\
(A_{B,B})_{i,j} &= (\delta_{x_i} \circ B)^x(\delta_{x_j} \circ B)^y K(x,y), x_i, x_j \in X_{bd}
\end{align*}
und
\begin{align*}
(b_L)_i &= f(x_i), x_i \in X_{in}\\
(b_B)_i &= g(x_i), x_i \in X_{bd}
\end{align*}
Das \ac{LGS} ist lösbar, da A offensichtlich symmetrisch und positiv definit ist, da:
\begin{align*}
\alpha^T A \alpha = \sum_{i,j = 1}^N \alpha_i \alpha_j (\nu_i, \nu_j)_{\mathcal{H}_K} = \left(\sum_{i=1}^N \alpha_i \nu_i, \sum_{j=1}^N \alpha_j \nu_j \right)_{\mathcal{H}_K} = \left\| \sum_{i=1}^N \alpha_i \nu_i \right\|_{\mathcal{H}_K}^2 > 0
\end{align*}
Für die letzte Abschätzung benutzen wir die lineare Unabhängigkeit der Funktionale aus Satz \ref{linUn}.
\subsection{Nicht-Symmetrische Kollokation}
\chapter{Kerne und reproduzierende Kern Hilberträume}
\label{cha:Grundlagen}

Die Kernkollokation ist ein Verfahren, welches auf der Idee der Interpolation beruht. Diese ist zunächst aber nur für die Punktauswertung bekannt. Das genügt uns hier nicht mehr, da wir auch Ableitungen betrachten müssen und benötigen daher eine verallgemeinerte Form.

\begin{definition}
Sei $\Omega \subset \mathbb{R}^n$ eine nicht leere Menge, $\mathcal{H}$ ein Hilbertraum bestehend aus Funktionen $f:\Omega \rightarrow \mathbb{R}$, $\mathcal{H}'$ der dazugehörige Dualraum, $u \in \mathcal{H}$  und $\Lambda_N := \{\lambda_1, \dots, \lambda_N\} \subset \mathcal{H}'$ eine Menge von linearen, stetigen und linear unabhängigen Funktionalen. Dann ist eine Funktion $s_u \in \mathcal{H}$ ein verallgemeinerter Interpolant von $u$, wenn gilt, dass
\begin{align*}
\langle \lambda_i,u \rangle = \langle \lambda_i,s_u \rangle , 1\le i \le N,
\end{align*}
wobei wir $\langle \lambda_i, u \rangle := \lambda_i(u)$ für die Anwendung des Funktionals schreiben.
\end{definition}

\begin{example}
$\mbox{}$
\begin{itemize}
\item
Sei $\Omega \subset \mathbb{R}^d$, $X_N := \{x_1, \dots, x_N\} \subset \Omega$ eine Menge von Punkten und $\mathcal{H}$ ein Hilbertraum bestehend aus Funktionen $f:\Omega \rightarrow \mathbb{R}$, in dem die Punktauswertungfunktionale $\delta_{x_i}(f) := f(x_i), 1\le i \le N$  stetig sind. Dann bekommen wir die Standardinterpolation mit $\Lambda_N := \{\delta_{x_1}, \dots,\delta_{x_N}\} \subset \mathcal{H}'$:
\begin{align*}
s(x_i) = \langle \delta_{x_i},s \rangle = \langle \delta_{x_i},s_u \rangle = s_u(x_i), 1\le i \le N
\end{align*}
\item
Für einen Multiindex $a \in \mathbb{N}_0^d$ sei $D^a$ ein linearer partieller Differentialoperator. Dann erhält man mit $\lambda_i := \delta_{x_i} \circ D^a$ noch zusätzliche Informationen über die Ableitung der Funktion.
\item
Sei eine \ac{PDE} mit Lösung $u \in C^k(\Omega) \cap C^0(\widebar \Omega)$ gegeben:
\begin{align*}
L u(x) &= f(x), x \in \Omega \\
B u(x) &= g(x), x \in \partial \Omega,
\end{align*}
wobei $L$ ein linearer Differentialoperator und $B$ ein linearer Randwertoperator ist.
Sei $X_N \subset \Omega$ eine Menge an Kollokationspunkten. Dann möchten wir, dass $s_u$ die \ac{PDE} in den Punkten $X_N$ erfüllt, also:
\begin{align} \label{eq:PDE}
\begin{split}
L s_u(x_i) &= L u(x_i) = f(x_i), x_i \in \Omega\\
B s_u(x_i) &= B u(x_i) = g(x_i), x_i \in \partial \Omega
\end{split}
\end{align}
\end{itemize}
\end{example}

Wir müssen einen geeigneten diskreten Ansatz wählen um das Interpolationsproblem numerisch zu lösen, also einen $N$-dimensionalen Unterraum $V_N := \text{span}\{\nu_1, \dots, \nu_N\} \subset \mathcal{H}$ und fordern, dass $s_u \in V_N$, also 
\begin{align*}
s_u(x) := \sum_{j=1}^N \alpha_j \nu_j(x), x \in \Omega, \alpha \in \mathbb{R}^N.
\end{align*}
Also lassen sich die Interpolationsbedingungen schreiben als:
\begin{align*}
\langle \lambda_i,u \rangle = \langle \lambda_i,s_u \rangle = \sum_{j=1}^N \alpha_j \langle \lambda_i,\nu_j \rangle
\end{align*}
Diese lassen sich auch als lineares Gleichungssystem $A_\Lambda \alpha = b$ schreiben  mit $(A_\Lambda)_{i,j} := \langle \lambda_i,\nu_j \rangle, b_i := \langle \lambda_i,u \rangle$.

Wir suchen jetzt nach geeigneten Ansatzfunktionen und einem Hilbertraum, in dem die Auswertungs- und Differentialfunktionale stetig sind. Dies führt uns zur Definition von Kern Funktionen, mit denen wir einen Hilbertraum konstruieren werden, der uns das Geforderte liefern wird.

\begin{definition}
\label{Kern}
Sei $\Omega$ eine nicht leere Menge. Ein reeller Kern auf $\Omega$ ist eine symmetrische Funktion $K: \Omega \times \Omega \rightarrow \mathbb{R}$.

Für alle $N \in \mathbb{N}$ und für eine Menge $X_N = \{x_i\}_{i=1}^N \subset \Omega$ ist die Kernmatrix (oder Gram'sche Matrix) $A:= A_{K,X_N} \in \mathbb{R}^{N \times N}$  definiert als $A:=[K(x_i, x_j)]_{i,j=1}^N$.

Ein Kern $K$ heißt \ac{PD} auf $\Omega$, wenn für alle $N \in \mathbb{N}$ und alle Mengen $X_N$ mit paarweise verschiedenen Elementen $\left\{x_i\right\}_{i=1}^N \subset \Omega$ gilt, dass die Kernmatrix positiv semidefinit ist. Der Kern $K$ heißt \ac{SPD}, falls die Kernmatrix für alle solche $N$ und $X_N$ positiv definit ist.
\end{definition}

\begin{theorem}
\label{thm:Kombi}
Sei $\Omega$ eine nicht leere Menge, $K_1, K_2:\Omega \rightarrow \mathbb{R}$ zwei \ac{PD} Kerne auf $\Omega$ und $a \geq 0$. Dann sind folgende Funktionen wieder \ac{PD} Kerne auf $\Omega$:
\begin{enumerate}
\item
$K(x,y) := K_1(x,y) + K_2(x,y)$
\item
$K(x,y) := aK_1 (x,y)$
\item
$K(x,y) := K_1(x,y)K_2(x,y)$
\end{enumerate}
\end{theorem}
\begin{proof}
Die Symmetrie ist in allen Fällen offensichtlich. Wir betrachten daher nur die positive Definitheit.

Sei $X_N \subset \Omega$ eine Menge mit paarweise verschiedenen Punkten $\left\{x_i\right\}_{i=1}^N$.
\begin{enumerate}
\item
Für die Kernmatrix von $K$ gilt:
\begin{align*}
A_K &= 
\begin{pmatrix}
K(x_1, x_1) & \cdots & K(x_1, x_N) \\ 
\vdots & \ddots & \vdots \\ 
K(x_N, x_1) & \cdots & K(x_N, x_N)
\end{pmatrix} \\
&=
\begin{pmatrix}
K_1(x_1, x_1) & \cdots & K_1(x_1, x_N) \\ 
\vdots & \ddots & \vdots \\ 
K_1(x_N, x_1) & \cdots & K_1(x_N, x_N)
\end{pmatrix} 
+
\begin{pmatrix}
K_2(x_1, x_1) & \cdots & K_2(x_1, x_N) \\ 
\vdots & \ddots & \vdots \\ 
K_2(x_N, x_1) & \cdots & K_2(x_N, x_N)
\end{pmatrix} \\
&= A_{K_1} + A_{K_2}
\end{align*}
Wir erhalten also für ein beliebiges $\alpha \neq 0$
\begin{align*}
\alpha^T A_K \alpha &= \alpha^T \left( A_{K_1} + A_{K_2} \right) \alpha \\
&=\underbrace{\alpha^T A_{K_1} \alpha}_{\geq 0} + \underbrace{\alpha^T A_{K_2} \alpha}_{\geq 0} \geq 0
\end{align*}
\item
Für ein beliebiges $\alpha \neq 0$ gilt
\begin{align*}
\alpha^T A_K \alpha = \alpha^T a A_{K_1} \alpha = a \alpha^T A_{K_1} \alpha \geq 0
\end{align*}
\item
Wir betrachten wieder die Kernmatrix.
\begin{align*}
K &= 
\begin{pmatrix}
K_1(x_1,x_1)K_2(x_1, x_1) & \cdots & K_1(x_1,x_N)K_2(x_1, x_N) \\ 
\vdots & \ddots & \vdots \\ 
K_1(x_N,x_1)K_2(x_N, x_1) & \cdots & K_1(x_N,x_N)K_2(x_N, x_N)
\end{pmatrix} \\
&= 
\begin{pmatrix}
K_1(x_1,x_1) & \cdots & K_1(x_1,x_N) \\ 
\vdots & \ddots & \vdots \\ 
K_1(x_N,x_1) & \cdots & K_1(x_N,x_N)
\end{pmatrix}
\circ
\begin{pmatrix}
K_2(x_1, x_1) & \cdots & K_2(x_1, x_N) \\ 
\vdots & \ddots & \vdots \\ 
K_2(x_N, x_1) & \cdots & K_2(x_N, x_N)
\end{pmatrix},
\end{align*}
wobei $\circ$ das Hadamard-Produkt der beiden Matrizen bezeichnet. 
Die beiden letzten Matrizen sind positiv semidefinit und damit nach dem Satz von Schur \cite{.30.07.2018} auch das Produkt der beiden.
\end{enumerate}
\end{proof}

\begin{example}
\label{ex:Kern}
Sei $\Omega \subset \mathbb{R}^n$ und $\gamma \in \mathbb{R}^+$. Dann sind folgende Funktionen \ac{PD} Kerne auf $\Omega$:
\begin{enumerate}
\item Skalarprodukt: $K(x,y) := \gamma^{-1} (x,y)$
\item Gauß Kern: $K(x,y) := \exp\left(-\gamma \|x-y\|^2\right)$ ist sogar \ac{SPD}
\end{enumerate}
\end{example}

\begin{proof}
$\mbox{}$
\begin{enumerate}
\item
Die Symmetrie folgt aus der Symmetrie des Skalarprodukts. Die Kernmatrix entspricht der Gram Matrix des Skalarprodukts. Diese ist aufgrund der positiven Definitheit des Skalarprodukt positiv definit. 
\item
Einen Beweis dafür findet man in \textcite[Theorem 6.10]{Wendland.2005}.
\end{enumerate}
\end{proof}
Wir kommen mit der Definition von Kernen direkt zu den gesuchten Hilberträumen. Diese sind zunächst ohne Bezug zu Kernen definiert, wir werden aber feststellen, dass sie eng miteinander verknüpft sind.

\begin{definition}[Reproduzierender Kern Hilbertraum]
Sei $\Omega$ eine nicht leere Menge und $\mathcal{H}$ ein Hilbertraum mit Funktionen $f:\Omega \rightarrow \mathbb{R}$ und Skalarprodukt $(\cdot, \cdot)_\mathcal{H}$. Dann nennt man $\mathcal{H}$ einen \glslink{RKHR}{Reproduzierenden Kern Hilbert Raum} auf $\Omega$, wenn eine Funktion $K:\Omega \times \Omega \rightarrow \mathbb{R}$ existiert, sodass
\begin{enumerate}
\item $K(\cdot, x) \in \mathcal{H}$ für alle $x \in \Omega$
\item $(f, K(\cdot,x))_\mathcal{H} = f(x)$ für alle $ x \in \Omega$, $f \in \mathcal{H}$ (Reproduzierbarkeit)
\end{enumerate}
Man nennt $K$ den reproduzierenden Kern von $\mathcal{H}$.
\end{definition}

Dass $K$ tatsächlich ein Kern nach Definition \ref{Kern} ist, zeigt folgender Satz.
\begin{theorem}
\label{thm:EindeutigkeitKern}
Sei $\mathcal{H}$ ein \gls{RKHR} mit reproduzierendem Kern $K$. Dann ist $K$ ein Kern, eindeutig und positiv definit.
\end{theorem}

\begin{proof}
Wir folgen dem Beweis in \textcite[Theorem 3.6]{Santin.2017}.

Wir zeigen zunächst, dass $K$ tatsächlich ein Kern ist.
\begin{align*}
K(x,y) &= \left( K(\cdot,y), K(x, \cdot)\right)_\mathcal{H} &&\text{(Reproduzierbarkeit)}\\
&= \left(K(x,\cdot), K(\cdot,y)\right)_\mathcal{H}\\
&= K(y,x) &&\text{(Reproduzierbarkeit)}
\end{align*}

Sei $X_N \subset \Omega$ eine Menge von paarweise verschiedenen Punkten und $\alpha \in \mathbb{R}^N, \alpha \neq 0$. Dann gilt:
\begin{align*}
\alpha^T A \alpha &= \sum_{i,j=1}^N \alpha_i \alpha_j K(x_i, x_j)\\
&= \sum_{i,j=1}^N \alpha_i \alpha_j \left(K(\cdot, x_i), K(\cdot,x_j)\right)_\mathcal{H}\\
&= \left(\sum_{i=1}^N \alpha_i K(\cdot,x_i), \sum_{j=1}^N \alpha_j K(\cdot, x_j)\right)_\mathcal{H}\\
&= \left\| \sum_{i=1}^N \alpha_i K(\cdot,x_i)\right\|_\mathcal{H}^2 \geq 0
\end{align*}
$K$ ist somit \ac{PD}.

Seien jetzt $K_1, K_2$ zwei Kerne auf $\mathcal{H}$. Dann gilt für alle $x,y \in \Omega$:
\begin{align*}
K_1(x,y) &= (K_1(\cdot,y), K_2(x, \cdot))_\mathcal{H} &&\text{(Reproduzierbarkeit von }K_1\text{)}\\
&= K_2(x,y) &&\text{(Reproduzierbarkeit von }K_2\text{)}
\end{align*}
Also ist $K$ eindeutig.
\end{proof}

Bei Interpolationsproblemen kommen wir jedoch aus der anderen Richtung und haben Ansatzfunktionen, also einen Kern $K$, gegeben und wollen damit eine Funktion approximieren. Also stellt sich die Frage ob zu jedem Kern $K$ ein \ac{RKHR} existiert. Diese wird durch folgenden Satz beantwortet:

\begin{theorem}
Sei $\Omega$ eine nicht leere Menge und $K:\Omega \times \Omega \rightarrow \mathbb{R}$ ein positiv definiter Kern. Dann existiert genau ein \ac{RKHR} $\mathcal{H}_K (\Omega)$ mit reproduzierendem Kern $K$.
\end{theorem}
\begin{proof}
Einen Beweis findet man in \textcite[Kap. 10.2]{Wendland.2005}. 

Man betrachtet dort zunächst den $\text{span} \{K(\cdot,y), y \in \Omega \}$ und stellt fest, dass dieser mit einem geeigneten Innenprodukt ein Prähilbertraum ist. Der Abschluss dessen ist der gesuchte \ac{RKHR}.
\end{proof}

Wir wollen an einen Satz aus der Funktionalanalysis erinnern, den wir oft brauchen werden.

\begin{theorem}[Fréchet-Riesz]
Sei $\mathcal{H}$ ein Hilbertraum und $\lambda \in \mathcal{H}'$ ein beschränktes lineares Funktional. Dann existiert ein eindeutig bestimmtes Element $\nu_\lambda \in \mathcal{H}$, so dass für alle $x \in \mathcal{H}$ gilt:
\begin{align*}
\langle \lambda, x \rangle = \left( x, \nu_\lambda \right)
\end{align*}
Wir nennen $\nu_\lambda$ den Riesz-Repräsentanten von $\lambda$.
\end{theorem}

Zur Wohldefiniertheit unserer Interpolation benötigen wir die Stetigkeit aller benutzten Funktionale. Zunächst betrachten wir die Punktauswertungsfunktionale.

\begin{theorem}
\label{stetig}
Sei $\Omega$ eine nicht leere Menge und $\mathcal{H}$ ein Hilbertraum mit Funktionen $f: \Omega \rightarrow \mathbb{R}$. Dann gilt:
\begin{enumerate}
\item \label{stetig1} $\mathcal{H}$ ist genau dann ein \ac{RKHR}, wenn die Auswertungsfunktionale stetig sind.
\item \label{stetig2} Wenn $\mathcal{H}$ ein \ac{RKHR} mit reproduzierendem Kern $K$ ist, dann ist $K(\cdot,x)$ der Riesz-Repräsentant des Funktionals $\delta_x \in \mathcal{H}'$.
\end{enumerate}
\end{theorem}

\begin{proof}
Wir folgen dem Beweis in \textcite[Proposition 3.8]{Santin.2017}.
\begin{enumerate}
\item 
Sei $\mathcal{H}$ ein \ac{RKHR}. Für alle $f \in \mathcal{H}$ und alle $x \in \Omega$ gilt:
\begin{align*}
|\langle \delta_x,f \rangle | &= |f(x)|\\ 
&= |(f, K(\cdot,x))_\mathcal{H}| &&\text{(Reproduzierbarkeit)}\\
&\le \|f\|_\mathcal{H}\|K(\cdot,x)\|_\mathcal{H} &&\text{(Cauchy Schwarz)}\\
&= \|f\|_\mathcal{H} \sqrt{(K(\cdot,x),K(\cdot,x))_\mathcal{H}}\\
&= \|f\|_\mathcal{H} \sqrt{K(x,x)} &&\text{(Reproduzierbarkeit)}\\
\Leftrightarrow \frac{|\langle \delta_x,f \rangle|}{\|f\|_\mathcal{H}} &\le \sqrt{K(x,x)}
\end{align*}

Also ist $\delta_x$ beschränkt und damit stetig.

Für die andere Richtung nehmen wir an, dass $\delta_x  \in \mathcal{H}'$ für alle $x \in \Omega$. Also existiert ein Riesz-Repräsentant $\nu_{\delta_x} \in \mathcal{H}$. Definieren wir $K(\cdot,x):= \nu_{\delta_x}$, dann ist $K$ ein reproduzierender Kern. Es ist klar, dass $K(\cdot,x) \in \mathcal{H}$ und nach der Definition des Riesz-Repräsentanten gilt:
\begin{align*}
(f, K(\cdot,x))_\mathcal{H} = (f, \nu_{\delta_x})_\mathcal{H} = \langle \delta_x,f \rangle = f(x)
\end{align*}
\item Die Behauptung folgt sofort aus der Reproduzierbarkeit von $K$, da $(f, K(\cdot,x))_\mathcal{H}= f(x)$ für alle $x \in \Omega$ und alle $f \in \mathcal{H}$ gilt.
\end{enumerate}
\end{proof}

Wir haben also gesehen, dass in einem \ac{RKHR} $\mathcal{H}_K$ die Auswertungsfunktionale stetig sind. Da wir uns mit Differentialgleichungen beschäftigen, wollen wir auch Ableitungen auswerten. Dafür benötigen wir, dass diese ebenfalls in $\mathcal{H}_K$ liegen.

\begin{theorem}
Sei $k \in \mathbb{N}$. Angenommen $\Omega \subset \mathbb{R}^n$ ist offen, K ist \ac{SPD} auf $\Omega$ und $K \in C^{2k}(\Omega \times \Omega)$. Dann gilt für alle Multiindizes $a \in \mathbb{N}_0^d$ mit $|a| \le k$ und alle $x \in \Omega$, dass $D_2^a K(\cdot , x) \in \mathcal{H}_K(\Omega)$, wobei der tiefgestellte Index bedeutet, dass der Operator auf das zweite Argument angewandt wird.

Außerdem gilt für alle $f \in \mathcal{H}_K(\Omega)$:
\begin{align*}
D^a f(x) = \left(f,D_2^a K(\cdot,x)\right)_{\mathcal{H}_K(\Omega)}
\end{align*}
und damit dass $\lambda := \delta_x \circ D^a$ stetig ist.
\end{theorem}

\begin{proof}
Einen Beweis des ersten Teils findet man im Vorlesungsskript von \textcite[Proposition 7.13]{Santin.2017} und einen Beweis des zweiten Teils in \textcite[Proposition 3.14]{Santin.2017}. Der Beweis der Stetigkeit von $\lambda := \delta_x \circ D^a$ verläuft komplett analog zum Beweis von \ref{stetig}.\ref{stetig1}.
\end{proof}

In Satz \ref{stetig} haben wir gesehen, wie der Riesz-Repräsentant des Auswertungsfunktionals aussieht. Dies wollen wir jetzt auf alle Funktionale verallgemeinern.

\begin{theorem}
\label{Riesz}
Sei $K$ ein \ac{PD} Kern auf $\Omega \neq \emptyset$. Sei $\lambda \in \mathcal{H}_K (\Omega)'$. Dann ist $\lambda^y K(\cdot,y) \in \mathcal{H}_k(\Omega)$ und es gilt $\langle \lambda,f \rangle = \left(f,\lambda^y K(\cdot,y)\right)_{\mathcal{H}_K(\Omega)}$ für alle $f \in \mathcal{H}_K(\Omega)$, wobei der hochgestellte Index bedeutet, dass das Funktional auf die zweite Komponente angewandt wird. Es ist also $\lambda^y K(\cdot,y)$ der Riesz-Repräsentant von $\lambda$.
\end{theorem}

\begin{proof}
Wir folgen dem Beweis in \textcite[Proposition 7.8]{Santin.2017}.

Da $\lambda \in \mathcal{H}_K(\Omega)$, existiert ein Riesz-Repräsentant $\nu_\lambda \in \mathcal{H}_K(\Omega)$ mit $\langle \lambda ,f \rangle = \left(f, \nu_\lambda\right)_{\mathcal{H}_K(\Omega)}$. Außerdem ist $f_x(y) := K(x,y)$ für alle $x \in \Omega$ eine Funktion in $\mathcal{H}_K (\Omega)$. Damit bekommen wir:
\begin{align*}
\langle \lambda, K(x,\cdot) \rangle &= \langle \lambda,f_x \rangle\\ 
&= \left(f_x, \nu_\lambda\right)_{\mathcal{H}_K (\Omega)} &&\text{(Riesz-Repräsentant)}\\ 
&= \left(K(\cdot,x), \nu_\lambda\right)_{\mathcal{H}_K (\Omega)}\\ 
&= \nu_\lambda(x) &&\text{(Reproduzierbarkeit)}
\end{align*}
Damit gilt $\nu_\lambda(\cdot) = \lambda^y K(\cdot,y)$ und damit ist $\lambda^yK(\cdot,y)$ der Riesz-Repräsentant von $\lambda$. Da $\nu_\lambda \in \mathcal{H}_K (\Omega)$ gilt auch $\lambda^y K(\cdot,y) \in \mathcal{H}_K (\Omega)$.
\end{proof}

Jetzt fehlt nur noch die lineare Unabhängigkeit aller verwendeten Funktionale. Zunächst betrachten wir die der Auswertungsfunktionale:
\begin{theorem}
Sei $\Omega$ eine nicht leere Menge und $\mathcal{H}$ ein \ac{RKHR} mit Kern $K$. Dann sind $\{\delta_x,x\in \Omega\}$ genau dann linear unabhängig, wenn $K$ \ac{SPD} ist.
\end{theorem}

\begin{proof}
Wir folgen dem Beweis in \textcite[Proposition 3.8]{Santin.2017}.

Seien $\lambda_1, \dots, \lambda_n \in \mathcal{H}'$ und $\nu_{\lambda_1},\dots, \nu_{\lambda_n} \in \mathcal{H}$ die dazugehörigen Riesz Repräsentanten. Diese sind linear abhängig, wenn ein $\alpha \in \mathbb{R}^n$ existiert mit $\lambda := \sum_{i=1}^n \alpha_i \lambda_i = 0$, also dass $\langle \lambda,f \rangle = 0$ für alle $f \in \mathcal{H}$. Das gilt genau dann, wenn die Riesz Repräsentanten linear abhängig sind, da
\begin{align*}
0 = \langle \lambda,f \rangle = \sum_{i=1}^n \alpha_i \langle \lambda_i,f \rangle = \sum_{i=1}^n \alpha_i \left( \nu_{\lambda_i},f\right)_\mathcal{H} = \left( \sum_{i=1}^n \alpha_i \nu_{\lambda_i}, f \right)_\mathcal{H}
\end{align*}

Also gilt nach \ref{stetig}.\ref{stetig2}, dass $\{\delta_x,x\in \Omega\}$ genau dann linear unabhängig sind, wenn $\{K(\cdot,x) , x \in \Omega\}$ linear unabhängig sind.

Um die strikte positive Definitheit nachzuweisen, betrachten wir die Matrix $A=[K(x_i, x_j)]_{i,j=1}^N$ für paarweise verschiedene Punkte $x_i, 1 \le i \le N$. Sei also $\beta \in \mathbb{R}^n, \beta \neq 0$. Dann gilt:
\begin{align*}
\beta^T A \beta &= \sum_{i,j=1}^n \beta_i \beta_j K(x_i, x_j)\\
&= \sum_{i,j=1}^n \beta_i  \beta_j \left(K(\cdot, x_i),K(\cdot,x_j)\right)_\mathcal{H}\\
&= \left( \sum_{i=1}^n \beta_i K(\cdot,x_i),\sum_{j=1}^n \beta_j K(\cdot, x_j) \right)_\mathcal{H}\\
&= \left\| \sum_{i=1}^n \beta_i K(\cdot, x_i) \right\|_\mathcal{H}^2 > 0
\end{align*}
Für die letzte strikte Ungleichung benötigen wir die lineare Unabhängigkeit. Also gilt, dass K \ac{SPD} ist, wenn $\{\delta_x,x\in \Omega\}$ linear unabhängig sind.

Die andere Richtung folgt genauso aus der letzten Ungleichung.
\end{proof}

Jetzt betrachten wir die lineare Unabhängigkeit der Auswertungen der Ableitungen:

\begin{theorem}
\label{linUn}
Sei $K$ ein translationsinvarianter Kern auf $\mathbb{R}^d$, also $K(x,y) = \Phi (x-y)$ für alle $x,y \in \mathbb{R}^d$. Sei $k \in \mathbb{N}$ und angenommen, dass $\Phi \in L_1(\mathbb{R}^d) \cap C^{2k}(\mathbb{R}^d)$. Sei $a_1, \dots, a_N \in \mathbb{N}_0^d$ mit $|a_i| \le k$ und sei $X_N \subset \mathbb{R}^d$. Angenommen, dass $a_i \neq a_j$, wenn $x_i = x_j$, dann sind die Funktionale $\Lambda_N := \{\lambda_1, \dots, \lambda_N\}, \lambda_i := \delta_{x_i} \circ D^{a_i}$ linear unabhängig in $\mathcal{H}_K(\mathbb{R}^d)'$.
\end{theorem}

\begin{proof}
Den Beweis findet man in \textcite[Theorem 16.4]{Wendland.2005}.
\end{proof}


Damit haben wir alle nötigen Werkzeuge, um die Interpolation durchzuführen. Wir haben einen Kern $K$, den dazugehörigen Hilbertraum $\mathcal{H}_K(\Omega)$ und die Stetigkeit und lineare Unabhängigkeit aller benötigten Operatoren. Jetzt müssen wir nur noch einen geeigneten Ansatz wählen.
\chapter{Standardkollokation}
\label{cha:Standardkollokation}

\section{Symmetrische Kollokation}
\label{sec:SymKol}
Sei wieder $\Omega \subset \mathbb{R}^d$ offen und beschränkt, $L$ und $B$ lineare Differentialoperatoren der Ordnung $k$ und $l$, $K \in C^{2\max\{l,k\}}(\Omega \times \Omega)$  ein \gls{SPD} Kern und eine \gls{PDE} wie in \eqref{eq:PDE} gegeben. Für ein $N \in \mathbb{N}$ betrachten wir die Menge $X_N = \left\{ x_i \right\}_{i=1}^N \subset \widebar{\Omega}$, die wir in $N_{in} > 0$ Punkte im Inneren und $N_{bd} > 0$ Punkte auf dem Rand aufteilen. Also haben wir die beiden Mengen
\begin{align*}
X_{in} &= X_N \cap \Omega\\
X_{bd} &= X_N \cap \partial \Omega
\end{align*}
Wir definieren die Menge $\Lambda_N = \{\lambda_1, \dots, \lambda_N\}$ an linearen Funktionalen mit
\begin{align*}
\lambda_i =
\begin{cases}
\delta_{x_i} \circ L & x_i \in X_{in}\\
\delta_{x_i} \circ B & x_i \in X_{bd}
\end{cases}
\end{align*}
Wir wissen aus Satz \ref{stetig}, dass in $\mathcal{H}_K(\Omega)$ alle $\lambda_i$ stetig und aus Satz \ref{linUn}, dass sie linear unabhängig sind. Als Ansatzfunktionen, also den Unterraum $V_N \subset \mathcal{H}_K(\Omega)$, wählen wir die Riesz Repräsentanten der $\lambda_i$:
\begin{align*}
V_N &= \text{span} \{\lambda_1^y K(x,y), \dots , \lambda_N^y K(x,y)\}\\
&= \text{span} \{\{(\delta_{x_1} \circ L)^y K(x,y), \dots, (\delta_{x_{N_{in}}} \circ L)^y K(x,y)\}\\
&\cup \{ (\delta_{x_{N_{in} + 1}} \circ B)^y K(x,y), \dots, (\delta_{x_{N}} \circ B)^y K(x,y)\}\}\\
&=: \text{span} \{\nu_1, \dots, \nu_N\}
\end{align*}

Damit bekommen wir folgenden Interpolanten:
\begin{align*}
s_u(x) &= \sum_{j=1}^N \alpha_j \lambda_j^y K(x,y)\\
&= \sum_{j=1}^{N_{in}} \alpha_j (\delta_{x_j} \circ L)^y K(x,y) + \sum_{j=N_{in}+1}^{N} \alpha_j (\delta_{x_j} \circ B)^y K(x,y)
\end{align*}
Die $\alpha_j$ erhält man als Lösung des \ac{LGS} $A \alpha = b$ mit $A_{i,j} := (\nu_i,\nu_j)_{\mathcal{H}_K}$, da
\begin{align*}
\langle \lambda_i, s_u\rangle = \langle \lambda_i, \sum_{j=1}^N \alpha_j \nu_j \rangle = \sum_{j=1}^N \alpha_j \langle \lambda_i,\nu_j\rangle \overset{\ref{Riesz}}{=} \sum_{j=1}^N \alpha_j \left(\nu_j, \nu_i\right),
\end{align*}
also
\begin{align*}
\begin{pmatrix}
A_{L,L} & A_{L,B} \\ 
A_{L,B}^T & A_{B,B}
\end{pmatrix} 
\alpha =
\begin{pmatrix}
b_L \\ 
b_B
\end{pmatrix} 
\end{align*}
mit
\begin{align*}
(A_{L,L})_{i,j} &= (\delta_{x_i} \circ L)^x(\delta_{x_j} \circ L)^y K(x,y),x_i, x_j \in X_{in}\\
(A_{L,B})_{i,j} &= (\delta_{x_i} \circ L)^x(\delta_{x_j} \circ B)^y K(x,y),x_i \in X_{in}, x_j \in X_{bd} \\
(A_{B,B})_{i,j} &= (\delta_{x_i} \circ B)^x(\delta_{x_j} \circ B)^y K(x,y), x_i, x_j \in X_{bd}
\end{align*}
und
\begin{align*}
(b_L)_i &= f(x_i), x_i \in X_{in}\\
(b_B)_i &= g(x_i), x_i \in X_{bd}
\end{align*}
Das \ac{LGS} ist lösbar, da A offensichtlich symmetrisch und positiv definit ist, da:
\begin{align*}
\alpha^T A \alpha = \sum_{i,j = 1}^N \alpha_i \alpha_j (\nu_i, \nu_j)_{\mathcal{H}_K} = \left(\sum_{i=1}^N \alpha_i \nu_i, \sum_{j=1}^N \alpha_j \nu_j \right)_{\mathcal{H}_K} = \left\| \sum_{i=1}^N \alpha_i \nu_i \right\|_{\mathcal{H}_K}^2 > 0
\end{align*}
Für die letzte Abschätzung benutzen wir die lineare Unabhängigkeit der Funktionale aus Satz \ref{linUn}.
\section{Nicht-Symmetrische Kollokation}
\label{sec:NSymKol}
Sei die gleiche Problemstellung wie im vorherigen Kapitel gegeben. Wir wählen jedoch einen anderen Unterraum $V_N$ für die Ansatzfunktionen:
\begin{align*}
V_N := \text{span} \{K(x,x_1), \dots, K(x,x_N) \}
\end{align*}
Damit bekommen wir folgenden Interpolanten:
\begin{align*}
s_u(x) = \sum_{j=1}^N \alpha_j K(x,x_j)
\end{align*}
Die $\alpha_i$ erhält man wieder als Lösung eines \ac{LGS} $A \alpha = b$ mit
\begin{align*}
A := \begin{pmatrix}
A_L \\ 
A_B
\end{pmatrix} 
\end{align*}
mit
\begin{align*}
(A_L)_{i,j} &= (\delta_{x_i} \circ L)^x K(x,x_j), x_i \in X_{in}, x_j \in X_N\\
(A_B)_{i,j} &= (\delta_{x_i} \circ B)^x K(x,x_j), x_i \in X_{bd}, x_j \in X_N
\end{align*}
und
$b$ wie im vorherigen Abschnitt.

Bei diesem Ansatz kann jedoch nicht garantiert werden, dass die Matrix $A$ invertierbar ist und damit, dass das \ac{LGS} auch lösbar ist. Allerdings ist die Konstruktion eines Gegenbeispiels aufwendig und es ist in der Praxis unwahrscheinlich, dass man auf einen solchen Fall trifft. Man findet ein Beispiel in \textcite{Hon.2001}.

\chapter{Gewichtete Kollokation}
\label{cha:Gewichtet}
\section{Motivation für die gewichtete Kollokation}
Die Standardkollokation hat, egal ob symmetrisch oder nicht-symmetrisch, das Problem, dass wir Punkte im Inneren und auf dem Rand unseres Definitionsbereichs benötigen. Dies macht zum einen die Implementierung etwas komplexer, da man dabei beide Mengen beachten muss, zum anderen werden die Werte auf dem Rand nicht zwingend genau angenommen. In Abbildung \ref{fig:rand} ist die approximierte Lösung einer \ac{PDE} mit Nullrandwerten über den Rand geplottet. Man erkennt deutlich, wo die Stützstellen der Ansatzfunktionen liegen und auch die Schwankungen zwischen den Stützstellen.
\begin{figure}[h]
\centering
\resizebox {\columnwidth} {!} {
% This file was created by matlab2tikz.
%
%The latest updates can be retrieved from
%  http://www.mathworks.com/matlabcentral/fileexchange/22022-matlab2tikz-matlab2tikz
%where you can also make suggestions and rate matlab2tikz.
%
\definecolor{mycolor1}{rgb}{0.00000,0.44700,0.74100}%
%
\begin{tikzpicture}

\begin{axis}[%
width=7.072in,
height=5.348in,
at={(1.186in,0.722in)},
scale only axis,
xmin=0,
xmax=6.28,
xlabel style={font=\color{white!15!black}},
xlabel={$\text{0 }\leq\text{ x }\leq\text{ 2}\pi$},
ymin=-0.0004,
ymax=0.0004,
axis background/.style={fill=white},
title style={font=\bfseries},
%title={Interpolant auf dem Rand},
legend style={legend cell align=left, align=left, draw=white!15!black}
]
\addplot [color=mycolor1]
  table[row sep=crcr]{%
0	-0.000393708935007453\\
0.00628947478196155	-0.000394012065953575\\
0.0125789495639231	-0.00039349839425995\\
0.0188684243458846	-0.000392166712117614\\
0.0251578991278462	-0.000390017979952972\\
0.0314473739098077	-0.000387055137252901\\
0.0377368486917693	-0.000383283342671348\\
0.0440263234737308	-0.000378709810320288\\
0.0503157982556924	-0.000373343711544294\\
0.0566052730376539	-0.000367196509614587\\
0.0628947478196155	-0.00036028142858413\\
0.069184222601577	-0.000352613940776791\\
0.0754736973835386	-0.000344211162882857\\
0.0817631721655001	-0.000335092398017878\\
0.0880526469474617	-0.000325278619129676\\
0.0943421217294232	-0.000314792534481967\\
0.100631596511385	-0.000303658871416701\\
0.106921071293346	-0.000291903648758307\\
0.113210546075308	-0.000279554711596575\\
0.119500020857269	-0.000266641320195049\\
0.125789495639231	-0.000253194153629011\\
0.132078970421193	-0.000239245124248555\\
0.138368445203154	-0.000224827683268813\\
0.144657919985116	-0.00020997631145292\\
0.150947394767077	-0.000194726271729451\\
0.157236869549039	-0.000179114154889248\\
0.163526344331	-0.000163177239301149\\
0.169815819112962	-0.000146953450894216\\
0.176105293894923	-0.000130481563246576\\
0.182394768676885	-0.00011380073192413\\
0.188684243458846	-9.69505636021495e-05\\
0.194973718240808	-7.99710833234712e-05\\
0.20126319302277	-6.29022579232696e-05\\
0.207552667804731	-4.57844107586425e-05\\
0.213842142586693	-2.86577451333869e-05\\
0.220131617368654	-1.15622060548048e-05\\
0.226421092150616	5.46227602171712e-06\\
0.232710566932577	2.23766037379391e-05\\
0.239000041714539	3.91414723708294e-05\\
0.2452895164965	5.57185849174857e-05\\
0.251578991278462	7.20702264516149e-05\\
0.257868466060423	8.81596679391805e-05\\
0.264157940842385	0.00010395065692137\\
0.270447415624347	0.000119408257887699\\
0.276736890406308	0.000134498444822384\\
0.28302636518827	0.000149188454088289\\
0.289315839970231	0.000163446577062132\\
0.295605314752193	0.000177242742211092\\
0.301894789534154	0.000190547987585887\\
0.308184264316116	0.000203334984689718\\
0.314473739098077	0.000215577936614864\\
0.320763213880039	0.000227252574404702\\
0.327052688662	0.000238336404436268\\
0.333342163443962	0.000248808530159295\\
0.339631638225924	0.000258650041359942\\
0.345921113007885	0.000267843421170255\\
0.352210587789847	0.000276373484666692\\
0.358500062571808	0.000284226509393193\\
0.36478953735377	0.000291390984784812\\
0.371079012135731	0.000297857099212706\\
0.377368486917693	0.000303616954624886\\
0.383657961699654	0.000308664813928772\\
0.389947436481616	0.000312996562570333\\
0.396236911263578	0.000316610166919418\\
0.402526386045539	0.000319505583320279\\
0.408815860827501	0.000321684521622956\\
0.415105335609462	0.000323150648910087\\
0.421394810391424	0.000323909414873924\\
0.427684285173385	0.000323968160955701\\
0.433973759955347	0.000323335927532753\\
0.440263234737308	0.000322023490298307\\
0.44655270951927	0.000320043323881691\\
0.452842184301231	0.000317409565468552\\
0.459131659083193	0.000314137963869143\\
0.465421133865155	0.000310245530272368\\
0.471710608647116	0.000305751047562808\\
0.478000083429078	0.000300674422760494\\
0.484289558211039	0.000295036999887088\\
0.490579032993001	0.000288861327135237\\
0.496868507774962	0.000282171018625377\\
0.503157982556924	0.000274990867183078\\
0.509447457338885	0.000267346385953715\\
0.515736932120847	0.000259264532360248\\
0.522026406902809	0.000250772467552451\\
0.52831588168477	0.000241898360400228\\
0.534605356466732	0.000232670863624662\\
0.540894831248693	0.000223119306610897\\
0.547184306030655	0.000213273564440897\\
0.553473780812616	0.00020316343579907\\
0.559763255594578	0.000192819294170476\\
0.566052730376539	0.000182271698577097\\
0.572342205158501	0.000171551266248571\\
0.578631679940462	0.000160688236064743\\
0.584921154722424	0.000149713174323551\\
0.591210629504386	0.000138656268973136\\
0.597500104286347	0.000127547315059928\\
0.603789579068309	0.000116416063974611\\
0.61007905385027	0.0001052915067703\\
0.616368528632232	9.42021906666923e-05\\
0.622658003414193	8.31760225992184e-05\\
0.628947478196155	7.2240403824253e-05\\
0.635236952978116	6.14216151006985e-05\\
0.641526427760078	5.07455260958523e-05\\
0.647815902542039	4.0236995118903e-05\\
0.654105377324001	2.99196362902876e-05\\
0.660394852105963	1.98164016183e-05\\
0.666684326887924	9.94890069705434e-06\\
0.672973801669886	3.38244717568159e-07\\
0.679263276451847	-8.99644146556966e-06\\
0.685552751233809	-1.80367132998072e-05\\
0.69184222601577	-2.67653012997471e-05\\
0.698131700797732	-3.51666349160951e-05\\
0.704421175579693	-4.32259403169155e-05\\
0.710710650361655	-5.09299716213718e-05\\
0.717000125143616	-5.82669163122773e-05\\
0.723289599925578	-6.52262424409855e-05\\
0.72957907470754	-7.1798396675149e-05\\
0.735868549489501	-7.79757319833152e-05\\
0.742158024271463	-8.37517181935254e-05\\
0.748447499053424	-8.9121163910022e-05\\
0.754736973835386	-9.40803074627183e-05\\
0.761026448617347	-9.86263585218694e-05\\
0.767315923399309	-0.000102758283901494\\
0.77360539818127	-0.000106476229120744\\
0.779894872963232	-0.000109781634819228\\
0.786184347745193	-0.000112676974822534\\
0.792473822527155	-0.000115165879833512\\
0.798763297309117	-0.000117253712232923\\
0.805052772091078	-0.000118946365546435\\
0.81134224687304	-0.000120251144835493\\
0.817631721655001	-0.000121176337415818\\
0.823921196436963	-0.000121731209219433\\
0.830210671218924	-0.000121925884741358\\
0.836500146000886	-0.000121771696285577\\
0.842789620782847	-0.000121280401799595\\
0.849079095564809	-0.000120464988867752\\
0.855368570346771	-0.000119338987133233\\
0.861658045128732	-0.000117916471936041\\
0.867947519910694	-0.000116212358989287\\
0.874236994692655	-0.000114242084237048\\
0.880526469474617	-0.000112021360109793\\
0.886815944256578	-0.00010956665937556\\
0.89310541903854	-0.000106894618511433\\
0.899394893820501	-0.000104022379673552\\
0.905684368602463	-0.000100967004982522\\
0.911973843384424	-9.77460804278962e-05\\
0.918263318166386	-9.43772865866777e-05\\
0.924552792948348	-9.08780348254368e-05\\
0.930842267730309	-8.72660457389429e-05\\
0.937131742512271	-8.35590799397323e-05\\
0.943421217294232	-7.97745888121426e-05\\
0.949710692076194	-7.59296672185883e-05\\
0.956000166858155	-7.20416828698944e-05\\
0.962289641640117	-6.81272285874002e-05\\
0.968579116422078	-6.4202984503936e-05\\
0.97486859120404	-6.02850086579565e-05\\
0.981158065986001	-5.63889771001413e-05\\
0.987447540767963	-5.25300965819042e-05\\
0.993737015549925	-4.87230900034774e-05\\
1.00002649033189	-4.49821927759331e-05\\
1.00631596511385	-4.13208254030906e-05\\
1.01260543989581	-3.77520154870581e-05\\
1.01889491467777	-3.42881030519493e-05\\
1.02518438945973	-3.09405186271761e-05\\
1.03147386424169	-2.77202889265027e-05\\
1.03776333902366	-2.46375311689917e-05\\
1.04405281380562	-2.17015513044316e-05\\
1.05034228858758	-1.89209495147225e-05\\
1.05663176336954	-1.63036238518544e-05\\
1.0629212381515	-1.38563773361966e-05\\
1.06921071293346	-1.15856710181106e-05\\
1.07550018771542	-9.49654713622294e-06\\
1.08178966249739	-7.59366957936436e-06\\
1.08807913727935	-5.88080729357898e-06\\
1.09436861206131	-4.36064874520525e-06\\
1.10065808684327	-3.03528941003606e-06\\
1.10694756162523	-1.90603896044195e-06\\
1.11323703640719	-9.73159330897033e-07\\
1.11952651118916	-2.36239429796115e-07\\
1.12581598597112	3.05681169265881e-07\\
1.13210546075308	6.54828909318894e-07\\
1.13839493553504	8.13659426057711e-07\\
1.144684410317	7.85727024776861e-07\\
1.15097388509896	5.75211743125692e-07\\
1.15726335988092	1.86799297807738e-07\\
1.16355283466289	-3.74093360733241e-07\\
1.16984230944485	-1.10112887341529e-06\\
1.17613178422681	-1.98798079509288e-06\\
1.18242125900877	-3.02712942357175e-06\\
1.18871073379073	-4.21097865910269e-06\\
1.19500020857269	-5.53120844415389e-06\\
1.20128968335466	-6.97901487001218e-06\\
1.20757915813662	-8.54538302519359e-06\\
1.21386863291858	-1.0220734111499e-05\\
1.22015810770054	-1.19952492241282e-05\\
1.2264475824825	-1.38586947286967e-05\\
1.23273705726446	-1.58005641424097e-05\\
1.23902653204642	-1.78102236532141e-05\\
1.24531600682839	-1.98766610992607e-05\\
1.25160548161035	-2.19890935113654e-05\\
1.25789495639231	-2.41362176893745e-05\\
1.26418443117427	-2.63067922787741e-05\\
1.27047390595623	-2.84896887023933e-05\\
1.27676338073819	-3.06735928461421e-05\\
1.28305285552016	-3.28471214743331e-05\\
1.28934233030212	-3.49995425494853e-05\\
1.29563180508408	-3.71196765627246e-05\\
1.30192127986604	-3.9196565921884e-05\\
1.308210754648	-4.12196459365077e-05\\
1.31450022942996	-4.31786575063597e-05\\
1.32078970421193	-4.50631123385392e-05\\
1.32707917899389	-4.68633334094193e-05\\
1.33336865377585	-4.85696873511188e-05\\
1.33965812855781	-5.01730864925776e-05\\
1.34594760333977	-5.16646323376335e-05\\
1.35223707812173	-5.30358556716237e-05\\
1.35852655290369	-5.4278858442558e-05\\
1.36481602768566	-5.53860518266447e-05\\
1.37110550246762	-5.63505564059597e-05\\
1.37739497724958	-5.71654818486422e-05\\
1.38368445203154	-5.78250801481772e-05\\
1.3899739268135	-5.832387614646e-05\\
1.39626340159546	-5.86567075515632e-05\\
1.40255287637743	-5.88193215662614e-05\\
1.40884235115939	-5.88078546570614e-05\\
1.41513182594135	-5.86191526963376e-05\\
1.42142130072331	-5.82506690989248e-05\\
1.42771077550527	-5.77002501813695e-05\\
1.43400025028723	-5.69664771319367e-05\\
1.44028972506919	-5.60486150789075e-05\\
1.44657919985116	-5.49463838979136e-05\\
1.45286867463312	-5.36601728526875e-05\\
1.45915814941508	-5.21911788382567e-05\\
1.46544762419704	-5.0540802476462e-05\\
1.471737098979	-4.87112884002272e-05\\
1.47802657376096	-4.67054960608948e-05\\
1.48431604854293	-4.45268378825858e-05\\
1.49060552332489	-4.21791810367722e-05\\
1.49689499810685	-3.96669129258953e-05\\
1.50318447288881	-3.69951631000731e-05\\
1.50947394767077	-3.41695922543295e-05\\
1.51576342245273	-3.1196053896565e-05\\
1.52205289723469	-2.80812673736364e-05\\
1.52834237201666	-2.48320975515526e-05\\
1.53463184679862	-2.1456100512296e-05\\
1.54092132158058	-1.7961130652111e-05\\
1.54721079636254	-1.43554425449111e-05\\
1.5535002711445	-1.06476618384477e-05\\
1.55978974592646	-6.84698898112401e-06\\
1.56607922070843	-2.96249345410615e-06\\
1.57236869549039	9.95991285890341e-07\\
1.57865817027235	5.01876274938695e-06\\
1.58494764505431	9.09559457795694e-06\\
1.59123711983627	1.32160348584875e-05\\
1.59752659461823	1.7369762645103e-05\\
1.60381606940019	2.15460568142589e-05\\
1.61010554418216	2.57339488598518e-05\\
1.61639501896412	2.99228195217438e-05\\
1.62268449374608	3.41015947924461e-05\\
1.62897396852804	3.82593025278766e-05\\
1.63526344331	4.23851088271476e-05\\
1.64155291809196	4.64679251308553e-05\\
1.64784239287393	5.04971467307769e-05\\
1.65413186765589	5.44618342246395e-05\\
1.66042134243785	5.83514265599661e-05\\
1.66671081721981	6.21555482211988e-05\\
1.67300029200177	6.58638928143773e-05\\
1.67928976678373	6.94662703608628e-05\\
1.68557924156569	7.29531675460748e-05\\
1.69186871634766	7.63148054829799e-05\\
1.69815819112962	7.95421656221151e-05\\
1.70444766591158	8.26263349154033e-05\\
1.71073714069354	8.55587531987112e-05\\
1.7170266154755	8.8331235019723e-05\\
1.72331609025746	9.09360496734735e-05\\
1.72960556503943	9.33659466682002e-05\\
1.73589503982139	9.56139992922544e-05\\
1.74218451460335	9.76737574092112e-05\\
1.74847398938531	9.95393202174455e-05\\
1.75476346416727	0.000101205234386725\\
1.76105293894923	0.000102666785096517\\
1.76734241373119	0.000103919439425226\\
1.77363188851316	0.000104959326563403\\
1.77992136329512	0.000105783252365654\\
1.78621083807708	0.000106388437416172\\
1.79250031285904	0.000106772633444052\\
1.798789787641	0.000106934534414904\\
1.80507926242296	0.000106873092590831\\
1.81136873720493	0.000106587718619267\\
1.81765821198689	0.000106078823591815\\
1.82394768676885	0.000105347007774981\\
1.83023716155081	0.000104393733636243\\
1.83652663633277	0.000103221005701926\\
1.84281611111473	0.00010183125777985\\
1.8491055858967	0.000100227567600086\\
1.85539506067866	9.84136495389976e-05\\
1.86168453546062	9.63936763582751e-05\\
1.86797401024258	9.41724283620715e-05\\
1.87426348502454	9.17551187740173e-05\\
1.8805529598065	8.914764111978e-05\\
1.88684243458846	8.63561690493952e-05\\
1.89313190937043	8.33875637908932e-05\\
1.89942138415239	8.02488684712444e-05\\
1.90571085893435	7.69479629525449e-05\\
1.91200033371631	7.34927962184884e-05\\
1.91828980849827	6.98919393471442e-05\\
1.92457928328023	6.61540398141369e-05\\
1.9308687580622	6.22884836047888e-05\\
1.93715823284416	5.83046821702737e-05\\
1.94344770762612	5.42125089850742e-05\\
1.94973718240808	5.00218338856939e-05\\
1.95602665719004	4.57432215625886e-05\\
1.962316131972	4.13869420299307e-05\\
1.96860560675396	3.69638692063745e-05\\
1.97489508153593	3.24847387673799e-05\\
1.98118455631789	2.79603882518131e-05\\
1.98747403109985	2.34020953939762e-05\\
1.99376350588181	1.8820705008693e-05\\
2.00005298066377	1.42274911922868e-05\\
2.00634245544573	9.63345883064903e-06\\
2.0126319302277	5.04965646541677e-06\\
2.01892140500966	4.87110810354352e-07\\
2.02521087979162	-4.04336606152356e-06\\
2.03150035457358	-8.53098026709631e-06\\
2.03778982935554	-1.29652471514419e-05\\
2.0440793041375	-1.73356565937866e-05\\
2.05036877891946	-2.16321386687923e-05\\
2.05665825370143	-2.58447325904854e-05\\
2.06294772848339	-2.99639614240732e-05\\
2.06923720326535	-3.39804137183819e-05\\
2.07552667804731	-3.78851327695884e-05\\
2.08181615282927	-4.16695438616443e-05\\
2.08810562761123	-4.53252832812723e-05\\
2.0943951023932	-4.88446457893588e-05\\
2.10068457717516	-5.22201444255188e-05\\
2.10697405195712	-5.54448270122521e-05\\
2.11326352673908	-5.85123889322858e-05\\
2.11955300152104	-6.14164455328137e-05\\
2.125842476303	-6.41516307950951e-05\\
2.13213195108496	-6.67131753289141e-05\\
2.13842142586693	-6.90963388478849e-05\\
2.14471090064889	-7.12971777829807e-05\\
2.15100037543085	-7.3312327003805e-05\\
2.15728985021281	-7.51389343349729e-05\\
2.16357932499477	-7.67745696066413e-05\\
2.16986879977673	-7.82177376095206e-05\\
2.1761582745587	-7.94670668256003e-05\\
2.18244774934066	-8.05218551249709e-05\\
2.18873722412262	-8.13820515759289e-05\\
2.19502669890458	-8.20482055132743e-05\\
2.20131617368654	-8.25213937787339e-05\\
2.2076056484685	-8.28030970296822e-05\\
2.21389512325046	-8.28954180178698e-05\\
2.22018459803243	-8.28008414828219e-05\\
2.22647407281439	-8.25228198664263e-05\\
2.23276354759635	-8.20647765067406e-05\\
2.23905302237831	-8.14308950793929e-05\\
2.24534249716027	-8.06256612122525e-05\\
2.25163197194223	-7.96542408352252e-05\\
2.2579214467242	-7.85219526733272e-05\\
2.26421092150616	-7.72346938902047e-05\\
2.27050039628812	-7.57988491386641e-05\\
2.27678987107008	-7.42208612791728e-05\\
2.28307934585204	-7.2507780714659e-05\\
2.289368820634	-7.06668179191183e-05\\
2.29565829541597	-6.87055726302788e-05\\
2.30194777019793	-6.66319101583213e-05\\
2.30823724497989	-6.44537431071512e-05\\
2.31452671976185	-6.21793733444065e-05\\
2.32081619454381	-5.98172446188983e-05\\
2.32710566932577	-5.73759316466749e-05\\
2.33339514410773	-5.48641619388945e-05\\
2.3396846188897	-5.22906884725671e-05\\
2.34597409367166	-4.96643442602362e-05\\
2.35226356845362	-4.69939732283819e-05\\
2.35855304323558	-4.42884957010392e-05\\
2.36484251801754	-4.15566901210696e-05\\
2.3711319927995	-3.8807382225059e-05\\
2.37742146758147	-3.60492194886319e-05\\
2.38371094236343	-3.32908311975189e-05\\
2.39000041714539	-3.05405628751032e-05\\
2.39628989192735	-2.78067018371075e-05\\
2.40257936670931	-2.50973898801021e-05\\
2.40886884149127	-2.24203540710732e-05\\
2.41515831627323	-1.97832341655158e-05\\
2.4214477910552	-1.71933352248743e-05\\
2.42773726583716	-1.46577258419711e-05\\
2.43402674061912	-1.21830562420655e-05\\
2.44031621540108	-9.77573290583678e-06\\
2.44660569018304	-7.4416930146981e-06\\
2.452895164965	-5.18685919814743e-06\\
2.45918463974697	-3.01620093523525e-06\\
2.46547411452893	-9.34855052037165e-07\\
2.47176358931089	1.05287836049683e-06\\
2.47805306409285	2.94285928248428e-06\\
2.48434253887481	4.73142426926643e-06\\
2.49063201365677	6.41553197056055e-06\\
2.49692148843873	7.99232657300308e-06\\
2.5032109632207	9.45954525377601e-06\\
2.50950043800266	1.08154126792215e-05\\
2.51578991278462	1.20583208627068e-05\\
2.52207938756658	1.31874694488943e-05\\
2.52836886234854	1.42023600346874e-05\\
2.5346583371305	1.51028434629552e-05\\
2.54094781191247	1.58894945343491e-05\\
2.54723728669443	1.65628007380292e-05\\
2.55352676147639	1.71242463693488e-05\\
2.55981623625835	1.75754212250467e-05\\
2.56610571104031	1.79182425199542e-05\\
2.57239518582227	1.81552895810455e-05\\
2.57868466060423	1.82891562872101e-05\\
2.5849741353862	1.83231677510776e-05\\
2.59126361016816	1.82608418981545e-05\\
2.59755308495012	1.81056784640532e-05\\
2.60384255973208	1.78620575752575e-05\\
2.61013203451404	1.7534097423777e-05\\
2.616421509296	1.71268766280264e-05\\
2.62271098407797	1.6644782590447e-05\\
2.62900045885993	1.60931631398853e-05\\
2.63528993364189	1.54772169480566e-05\\
2.64157940842385	1.48024264490232e-05\\
2.64786888320581	1.40742849907838e-05\\
2.65415835798777	1.32986424432602e-05\\
2.66044783276973	1.24811231216881e-05\\
2.6667373075517	1.16276678454597e-05\\
2.67302678233366	1.07441592263058e-05\\
2.67931625711562	9.83666905085556e-06\\
2.68560573189758	8.91071977093816e-06\\
2.69189520667954	7.97262691776268e-06\\
2.6981846814615	7.02777833794244e-06\\
2.70447415624347	6.08237314736471e-06\\
2.71076363102543	5.14170460519381e-06\\
2.71705310580739	4.21135337091982e-06\\
2.72334258058935	3.29679096466862e-06\\
2.72963205537131	2.40315785049461e-06\\
2.73592153015327	1.53549262904562e-06\\
2.74221100493523	6.98350049788132e-07\\
2.7485004797172	-1.0331132216379e-07\\
2.75478995449916	-8.65489710122347e-07\\
2.76107942928112	-1.584008714417e-06\\
2.76736890406308	-2.25496842176653e-06\\
2.77365837884504	-2.87499642581679e-06\\
2.779947853627	-3.44079307978973e-06\\
2.78623732840897	-3.94959351979196e-06\\
2.79252680319093	-4.39877112512477e-06\\
2.79881627797289	-4.78590663988143e-06\\
2.80510575275485	-5.10939935338683e-06\\
2.81139522753681	-5.36747756996192e-06\\
2.81768470231877	-5.55885344510898e-06\\
2.82397417710074	-5.68298855796456e-06\\
2.8302636518827	-5.738820618717e-06\\
2.83655312666466	-5.72656790609471e-06\\
2.84284260144662	-5.6461722124368e-06\\
2.84913207622858	-5.49809919903055e-06\\
2.85542155101054	-5.28330201632343e-06\\
2.8617110257925	-5.0028356781695e-06\\
2.86800050057447	-4.65813354821876e-06\\
2.87428997535643	-4.25092366640456e-06\\
2.88057945013839	-3.78349432139657e-06\\
2.88686892492035	-3.25783912558109e-06\\
2.89315839970231	-2.67682844423689e-06\\
2.89944787448427	-2.04324896913022e-06\\
2.90573734926624	-1.36028393171728e-06\\
2.9120268240482	-6.31163857178763e-07\\
2.91831629883016	1.40560587169603e-07\\
2.92460577361212	9.50967660173774e-07\\
2.93089524839408	1.79628113983199e-06\\
2.93718472317604	2.67247014562599e-06\\
2.943474197958	3.57521639671177e-06\\
2.94976367273997	4.50040533905849e-06\\
2.95605314752193	5.44341673958115e-06\\
2.96234262230389	6.39989957562648e-06\\
2.96863209708585	7.36528090783395e-06\\
2.97492157186781	8.33497324492782e-06\\
2.98121104664977	9.30449459701777e-06\\
2.98750052143174	1.02693084045313e-05\\
2.9937899962137	1.12248671939597e-05\\
3.00007947099566	1.21667653729673e-05\\
3.00636894577762	1.30906410049647e-05\\
3.01265842055958	1.39920230139978e-05\\
3.01894789534154	1.4867095160298e-05\\
3.0252373701235	1.57116264745127e-05\\
3.03152684490547	1.65218007168733e-05\\
3.03781631968743	1.72939908225089e-05\\
3.04410579446939	1.80246170202736e-05\\
3.05039526925135	1.87104451470077e-05\\
3.05668474403331	1.93483647308312e-05\\
3.06297421881527	1.99356618395541e-05\\
3.06926369359724	2.0469653463806e-05\\
3.0755531683792	2.0948040400981e-05\\
3.08184264316116	2.13687671930529e-05\\
3.08813211794312	2.1730143998866e-05\\
3.09442159272508	2.20305391849251e-05\\
3.10071106750704	2.22689959628042e-05\\
3.107000542289	2.24444665946066e-05\\
3.11329001707097	2.25564454012783e-05\\
3.11957949185293	2.26045740419067e-05\\
3.12586896663489	2.2589028958464e-05\\
3.13215844141685	2.25100102397846e-05\\
3.13844791619881	2.23683546209941e-05\\
3.14473739098077	2.21649133891333e-05\\
3.15102686576274	2.19009707507212e-05\\
3.1573163405447	2.1578218365903e-05\\
3.16360581532666	2.11981860047672e-05\\
3.16989529010862	2.07633529498708e-05\\
3.17618476489058	2.02759256353602e-05\\
3.18247423967254	1.97385525098071e-05\\
3.1887637144545	1.91543458640808e-05\\
3.19505318923647	1.85262215381954e-05\\
3.20134266401843	1.78576319740387e-05\\
3.20763213880039	1.71522915479727e-05\\
3.21392161358235	1.64137200044934e-05\\
3.22021108836431	1.5645993698854e-05\\
3.22650056314627	1.48531999002444e-05\\
3.23279003792824	1.40395895869005e-05\\
3.2390795127102	1.32095919980202e-05\\
3.24536898749216	1.23674199130619e-05\\
3.25165846227412	1.1517839084263e-05\\
3.25794793705608	1.06652823887998e-05\\
3.26423741183804	9.81450830295216e-06\\
3.27052688662	8.97011523193214e-06\\
3.27681636140197	8.13679980637971e-06\\
3.28310583618393	7.31898944650311e-06\\
3.28939531096589	6.52155540592503e-06\\
3.29568478574785	5.74870500713587e-06\\
3.30197426052981	5.00500573252793e-06\\
3.30826373531177	4.29465580964461e-06\\
3.31455321009374	3.62211721949279e-06\\
3.3208426848757	2.99092425848357e-06\\
3.32713215965766	2.4053570086835e-06\\
3.33342163443962	1.8688751879381e-06\\
3.33971110922158	1.38493487611413e-06\\
3.34600058400354	9.56866642809473e-07\\
3.3522900587855	5.87615431868471e-07\\
3.35857953356747	2.79937012237497e-07\\
3.36486900834943	3.62688297173008e-08\\
3.37115848313139	-1.41128111863509e-07\\
3.37744795791335	-2.50607627094723e-07\\
3.38373743269531	-2.90176103590056e-07\\
3.39002690747727	-2.58876752923243e-07\\
3.39631638225924	-1.55609086505137e-07\\
3.4026058570412	2.01252987608314e-08\\
3.40889533182316	2.68531948677264e-07\\
3.41518480660512	5.89812771067955e-07\\
3.42147428138708	9.83383870334364e-07\\
3.42776375616904	1.44848490890581e-06\\
3.43405323095101	1.98390807781834e-06\\
3.44034270573297	2.58828913501929e-06\\
3.44663218051493	3.25988730764948e-06\\
3.45292165529689	3.99632517655846e-06\\
3.45921113007885	4.79526534036268e-06\\
3.46550060486081	5.65389564144425e-06\\
3.47179007964277	6.56908741802908e-06\\
3.47807955442474	7.5372590799816e-06\\
3.4843690292067	8.5547253547702e-06\\
3.49065850398866	9.61727710091509e-06\\
3.49694797877062	1.07207997643854e-05\\
3.50323745355258	1.18604875751771e-05\\
3.50952692833454	1.30316329887137e-05\\
3.51581640311651	1.42291210067924e-05\\
3.52210587789847	1.54475201270543e-05\\
3.52839535268043	1.66814024851192e-05\\
3.53468482746239	1.79250728251645e-05\\
3.54097430224435	1.91726940101944e-05\\
3.54726377702631	2.04181651497493e-05\\
3.55355325180827	2.16555636143312e-05\\
3.55984272659024	2.28785665967735e-05\\
3.5661322013722	2.40810168179451e-05\\
3.57242167615416	2.5256455046474e-05\\
3.57871115093612	2.63985912170028e-05\\
3.58500062571808	2.75010879704496e-05\\
3.59129010050004	2.85574897134211e-05\\
3.59757957528201	2.95617483061505e-05\\
3.60386905006397	3.05073881463613e-05\\
3.61015852484593	3.13885466312058e-05\\
3.61644799962789	3.21989991789451e-05\\
3.62273747440985	3.29328722727951e-05\\
3.62902694919181	3.35845616064034e-05\\
3.63531642397377	3.41486756951781e-05\\
3.64160589875574	3.46196266036713e-05\\
3.6478953735377	3.49924830516102e-05\\
3.65418484831966	3.52624010702129e-05\\
3.66047432310162	3.54248913936317e-05\\
3.66676379788358	3.54755738953827e-05\\
3.67305327266554	3.54105031874496e-05\\
3.67934274744751	3.52262031810824e-05\\
3.68563222222947	3.49192469002446e-05\\
3.69192169701143	3.44867812600569e-05\\
3.69821117179339	3.3926236028492e-05\\
3.70450064657535	3.32354238707921e-05\\
3.71079012135731	3.24127377098193e-05\\
3.71707959613927	3.14566823362838e-05\\
3.72336907092124	3.0366411920113e-05\\
3.7296585457032	2.91414162347792e-05\\
3.73594802048516	2.77817280220916e-05\\
3.74223749526712	2.62877192653832e-05\\
3.74852697004908	2.46600975515321e-05\\
3.75481644483104	2.29002598644001e-05\\
3.76110591961301	2.10099542528042e-05\\
3.76739539439497	1.89914453585516e-05\\
3.77368486917693	1.68472943187226e-05\\
3.77997434395889	1.45805852298508e-05\\
3.78626381874085	1.21949306048919e-05\\
3.79255329352281	9.69427674135659e-06\\
3.79884276830477	7.08296283846721e-06\\
3.80513224308674	4.36603022535564e-06\\
3.8114217178687	1.54849749378627e-06\\
3.81771119265066	-1.36397284222767e-06\\
3.82400066743262	-4.36515165347373e-06\\
3.83029014221458	-7.44894896342885e-06\\
3.83657961699654	-1.06083307400695e-05\\
3.84286909177851	-1.38364248414291e-05\\
3.84915856656047	-1.71256124303909e-05\\
3.85544804134243	-2.04681828108733e-05\\
3.86173751612439	-2.38563306993456e-05\\
3.86802699090635	-2.72815450443886e-05\\
3.87431646568831	-3.07354566757567e-05\\
3.88060594047027	-3.42092816936201e-05\\
3.88689541525224	-3.76940793103131e-05\\
3.8931848900342	-4.11806586271268e-05\\
3.89947436481616	-4.46599319730012e-05\\
3.90576383959812	-4.81223814858822e-05\\
3.91205331438008	-5.15587225891068e-05\\
3.91834278916204	-5.49592155039136e-05\\
3.92463226394401	-5.8314318266639e-05\\
3.93092173872597	-6.1614511650987e-05\\
3.93721121350793	-6.48500540592067e-05\\
3.94350068828989	-6.80115908835432e-05\\
3.94979016307185	-7.10894978510623e-05\\
3.95607963785381	-7.4074439908145e-05\\
3.96236911263578	-7.69571488490328e-05\\
3.96865858741774	-7.97283710198826e-05\\
3.9749480621997	-8.23792979645077e-05\\
3.98123753698166	-8.49010305046249e-05\\
3.98752701176362	-8.7285227891698e-05\\
3.99381648654558	-8.95234493327735e-05\\
4.00010596132754	-9.16078163299971e-05\\
4.00639543610951	-9.35305079110549e-05\\
4.01268491089147	-9.52842949573096e-05\\
4.01897438567343	-9.68619315244723e-05\\
4.02526386045539	-9.82569258667354e-05\\
4.03155333523735	-9.94630112245432e-05\\
4.03784281001931	-0.000100474257806127\\
4.04413228480128	-0.000101285257358086\\
4.05042175958324	-0.000101891128167608\\
4.0567112343652	-0.000102287103345589\\
4.06300070914716	-0.000102469455256937\\
4.06929018392912	-0.000102434480055535\\
4.07557965871108	-0.000102179084137788\\
4.08186913349304	-0.000101700848858854\\
4.08815860827501	-0.000100997873914821\\
4.09444808305697	-0.000100068695701339\\
4.10073755783893	-9.89122456758196e-05\\
4.10702703262089	-9.75284134483445e-05\\
4.11331650740285	-9.59173989940609e-05\\
4.11960598218481	-9.40799054660602e-05\\
4.12589545696678	-9.20172756195825e-05\\
4.13218493174874	-8.97317868293612e-05\\
4.1384744065307	-8.72255218382634e-05\\
4.14476388131266	-8.45017223127797e-05\\
4.15105335609462	-8.1564050333327e-05\\
4.15734283087658	-7.84167539222835e-05\\
4.16363230565854	-7.5064558586746e-05\\
4.16992178044051	-7.15125765964331e-05\\
4.17621125522247	-6.7766954543913e-05\\
4.18250073000443	-6.38337214695639e-05\\
4.18879020478639	-5.97199991716479e-05\\
4.19507967956835	-5.5433079978684e-05\\
4.20136915435031	-5.09807705384446e-05\\
4.20765862913228	-4.63713436147373e-05\\
4.21394810391424	-4.16137618231005e-05\\
4.2202375786962	-3.67168681805197e-05\\
4.22652705347816	-3.16907676278788e-05\\
4.23281652826012	-2.65450867118489e-05\\
4.23910600304208	-2.1290305085131e-05\\
4.24539547782405	-1.59372025336779e-05\\
4.25168495260601	-1.04967657534871e-05\\
4.25797442738797	-4.98033932672115e-06\\
4.26426390216993	6.00458406552207e-07\\
4.27055337695189	6.23381492914632e-06\\
4.27684285173385	1.19075666589197e-05\\
4.28313232651581	1.76094208654831e-05\\
4.28942180129777	2.33270620810799e-05\\
4.29571127607974	2.90479465547833e-05\\
4.3020007508617	3.47592867910862e-05\\
4.30829022564366	4.044855995744e-05\\
4.31457970042562	4.61028084828286e-05\\
4.32086917520758	5.17092839800171e-05\\
4.32715864998954	5.72553371966933e-05\\
4.33344812477151	6.2728251577937e-05\\
4.33973759955347	6.81153196637752e-05\\
4.34602707433543	7.34040340830688e-05\\
4.35231654911739	7.858212939027e-05\\
4.35860602389935	8.36374510981841e-05\\
4.36489549868131	8.85579102032352e-05\\
4.37118497346328	9.333188245364e-05\\
4.37747444824524	9.79478754743468e-05\\
4.3837639230272	0.000102394716122944\\
4.39005339780916	0.000106661656900542\\
4.39634287259112	0.000110738048533676\\
4.40263234737308	0.000114613957521215\\
4.40892182215505	0.000118279725938919\\
4.41521129693701	0.000121726006909739\\
4.42150077171897	0.000124944000162941\\
4.42779024650093	0.000127925583910837\\
4.43407972128289	0.00013066274823359\\
4.44036919606485	0.000133148437271302\\
4.44665867084681	0.000135375878926425\\
4.45294814562878	0.000137338824060862\\
4.45923762041074	0.000139031974867976\\
4.4655270951927	0.00014045015177544\\
4.47181656997466	0.000141589230224781\\
4.47810604475662	0.000142445357596444\\
4.48439551953858	0.000143015628964349\\
4.49068499432054	0.000143297692375199\\
4.49697446910251	0.000143289733387064\\
4.50326394388447	0.000142990842505242\\
4.50955341866643	0.000142400640470441\\
4.51584289344839	0.000141519332828466\\
4.52213236823035	0.000140348221975728\\
4.52842184301231	0.00013888870489609\\
4.53471131779428	0.000137143399115303\\
4.54100079257624	0.000135115407829289\\
4.5472902673582	0.000132808516354999\\
4.55357974214016	0.000130227019326412\\
4.55986921692212	0.000127376148157055\\
4.56615869170408	0.000124261689052219\\
4.57244816648605	0.000120890015750774\\
4.57873764126801	0.000117268216854427\\
4.58502711604997	0.000113403932118672\\
4.59131659083193	0.000109305601654341\\
4.59760606561389	0.000104981731055886\\
4.60389554039585	0.000100442048278637\\
4.61018501517781	9.56965304794721e-05\\
4.61647448995978	9.07555677258642e-05\\
4.62276396474174	8.5630052126362e-05\\
4.6290534395237	8.03316015662858e-05\\
4.63534291430566	7.4871959441225e-05\\
4.64163238908762	6.92635567247635e-05\\
4.64792186386958	6.35189699096372e-05\\
4.65421133865155	5.76513339183293e-05\\
4.66050081343351	5.16739310114644e-05\\
4.66679028821547	4.5600321755046e-05\\
4.67307976299743	3.94446215068456e-05\\
4.67936923777939	3.32207528117578e-05\\
4.68565871256135	2.69430474872934e-05\\
4.69194818734332	2.06260829145322e-05\\
4.69823766212528	1.42841381602921e-05\\
4.70452713690724	7.93193430581596e-06\\
4.7108166116892	1.58411785378121e-06\\
4.71710608647116	-4.74461558042094e-06\\
4.72339556125312	-1.10397140815621e-05\\
4.72968503603508	-1.7286756701651e-05\\
4.73597451081705	-2.34714043472195e-05\\
4.74226398559901	-2.95793197437888e-05\\
4.74855346038097	-3.5596565794549e-05\\
4.75484293516293	-4.15093236370012e-05\\
4.76113240994489	-4.73041054647183e-05\\
4.76742188472685	-5.29677854501642e-05\\
4.77371135950882	-5.84875688218744e-05\\
4.78000083429078	-6.38509482087102e-05\\
4.78629030907274	-6.90459328325232e-05\\
4.7925797838547	-7.40609284548555e-05\\
4.79886925863666	-7.88848137744935e-05\\
4.80515873341862	-8.35071641631657e-05\\
4.81144820820058	-8.79178533068625e-05\\
4.81773768298255	-9.21075388760073e-05\\
4.82402715776451	-9.6067245976883e-05\\
4.83031663254647	-9.97889928839868e-05\\
4.83660610732843	-0.000103265070720227\\
4.84289558211039	-0.000106488680103212\\
4.84918505689235	-0.000109453600089182\\
4.85547453167431	-0.000112154484668281\\
4.86176400645628	-0.000114586506242631\\
4.86805348123824	-0.000116745766717941\\
4.8743429560202	-0.000118629059215891\\
4.88063243080216	-0.000120233878988074\\
4.88692190558412	-0.000121558645332698\\
4.89321138036608	-0.000122602319606813\\
4.89950085514805	-0.000123364992759889\\
4.90579032993001	-0.000123847388749709\\
4.91207980471197	-0.000124050893646199\\
4.91836927949393	-0.000123977733892389\\
4.92465875427589	-0.000123631054520956\\
4.93094822905785	-0.000123014573546243\\
4.93723770383982	-0.000122132945762132\\
4.94352717862178	-0.000120991671792581\\
4.94981665340374	-0.000119596514196019\\
4.9561061281857	-0.000117954425149946\\
4.96239560296766	-0.000116072804303258\\
4.96868507774962	-0.000113959786176565\\
4.97497455253158	-0.000111624378405395\\
4.98126402731355	-0.00010907565956586\\
4.98755350209551	-0.000106323863292346\\
4.99384297687747	-0.000103379623396904\\
5.00013245165943	-0.00010025390474766\\
5.00642192644139	-9.69584198173834e-05\\
5.01271140122335	-9.35052321437979e-05\\
5.01900087600531	-8.99068636499578e-05\\
5.02529035078728	-8.61761764099356e-05\\
5.03157982556924	-8.23263999336632e-05\\
5.0378693003512	-7.83710729592713e-05\\
5.04415877513316	-7.43242344469763e-05\\
5.05044824991512	-7.01994940754957e-05\\
5.05673772469708	-6.60114601487294e-05\\
5.06302719947905	-6.17743007751415e-05\\
5.06931667426101	-5.75023659621365e-05\\
5.07560614904297	-5.32102731085615e-05\\
5.08189562382493	-4.89123731313157e-05\\
5.08818509860689	-4.462316428544e-05\\
5.09447457338885	-4.03568565161549e-05\\
5.10076404817082	-3.6127589737589e-05\\
5.10705352295278	-3.19493128699833e-05\\
5.11334299773474	-2.78358256764477e-05\\
5.1196324725167	-2.38003640333773e-05\\
5.12592194729866	-1.98561128854635e-05\\
5.13221142208062	-1.60157642312697e-05\\
5.13850089686258	-1.22915016618208e-05\\
5.14479037164455	-8.69505038281204e-06\\
5.15107984642651	-5.23790640727384e-06\\
5.15736932120847	-1.93075538845733e-06\\
5.16365879599043	1.21651646622922e-06\\
5.16994827077239	4.19412208430003e-06\\
5.17623774555435	6.99329484632472e-06\\
5.18252722033631	9.6057929113158e-06\\
5.18881669511828	1.20241938930121e-05\\
5.19510616990024	1.42419867188437e-05\\
5.2013956446822	1.62530823217821e-05\\
5.20768511946416	1.80524639290525e-05\\
5.21397459424612	1.96358769244398e-05\\
5.22026406902808	2.09997733691125e-05\\
5.22655354381005	2.2141709450807e-05\\
5.23284301859201	2.3059904378897e-05\\
5.23913249337397	2.37536360145896e-05\\
5.24542196815593	2.42228261413402e-05\\
5.25171144293789	2.44685979851056e-05\\
5.25800091771985	2.44928587562754e-05\\
5.26429039250182	2.42982468989794e-05\\
5.27057986728378	2.38885395447141e-05\\
5.27686934206574	2.32683232752606e-05\\
5.2831588168477	2.24428717956471e-05\\
5.28944829162966	2.14187411984312e-05\\
5.29573776641162	2.02029182219121e-05\\
5.30202724119358	1.8803440980264e-05\\
5.30831671597555	1.72290606315073e-05\\
5.31460619075751	1.54894028128183e-05\\
5.32089566553947	1.35947316266538e-05\\
5.32718514032143	1.15561601887748e-05\\
5.33347461510339	9.38527591642924e-06\\
5.33976408988535	7.09466849002638e-06\\
5.34605356466732	4.69718224849203e-06\\
5.35234303944928	2.20649621951452e-06\\
5.35863251423124	-3.63361550625996e-07\\
5.3649219890132	-2.99776388601458e-06\\
5.37121146379516	-5.68184054827725e-06\\
5.37750093857712	-8.40005714053405e-06\\
5.38379041335909	-1.11366820192416e-05\\
5.39007988814105	-1.38757436616288e-05\\
5.39636936292301	-1.66011861892912e-05\\
5.40265883770497	-1.92963512404276e-05\\
5.40894831248693	-2.19449050860021e-05\\
5.41523778726889	-2.4530298340153e-05\\
5.42152726205085	-2.70359277578791e-05\\
5.42781673683282	-2.94454453637627e-05\\
5.43410621161478	-3.17423058504573e-05\\
5.44039568639674	-3.39106162527969e-05\\
5.4466851611787	-3.59343100626575e-05\\
5.45297463596066	-3.77979915811011e-05\\
5.45926411074262	-3.94863591282046e-05\\
5.46555358552459	-4.09845540616516e-05\\
5.47184306030655	-4.22784273723664e-05\\
5.47813253508851	-4.33542863902403e-05\\
5.48442200987047	-4.41989682258281e-05\\
5.49071148465243	-4.47999082098249e-05\\
5.49700095943439	-4.51454920948891e-05\\
5.50329043421635	-4.52248696092283e-05\\
5.50957990899832	-4.50276875199052e-05\\
5.51586938378028	-4.45447517449793e-05\\
5.52215885856224	-4.37678108937689e-05\\
5.5284483333442	-4.26893270741857e-05\\
5.53473780812616	-4.13027723880077e-05\\
5.54102728290812	-3.96029809053289e-05\\
5.54731675769009	-3.75855124730151e-05\\
5.55360623247205	-3.52473634848138e-05\\
5.55989570725401	-3.25861747114686e-05\\
5.56618518203597	-2.96011330647161e-05\\
5.57247465681793	-2.62925987044582e-05\\
5.57876413159989	-2.26617730731959e-05\\
5.58505360638185	-1.87115520020598e-05\\
5.59134308116382	-1.44456043926766e-05\\
5.59763255594578	-9.86903160082875e-06\\
5.60392203072774	-4.98824374517426e-06\\
5.6102115055097	1.89402271644212e-07\\
5.61650098029166	5.65487243875396e-06\\
5.62279045507362	1.13987007352989e-05\\
5.62907992985559	1.74093729583547e-05\\
5.63536940463755	2.36746500377194e-05\\
5.64165887941951	3.01811733152135e-05\\
5.64794835420147	3.69142944691703e-05\\
5.65423782898343	4.38584465882741e-05\\
5.66052730376539	5.09967394464184e-05\\
5.66681677854735	5.83116088819224e-05\\
5.67310625332932	6.57842138025444e-05\\
5.67939572811128	7.33949655113975e-05\\
5.68568520289324	8.11232966952957e-05\\
5.6919746776752	8.89478706085356e-05\\
5.69826415245716	9.68465192272561e-05\\
5.70455362723912	0.000104796439700294\\
5.71084310202109	0.000112774227090995\\
5.71713257680305	0.000120755550597096\\
5.72342205158501	0.000128716179460753\\
5.72971152636697	0.000136630724227871\\
5.73600100114893	0.000144473986438243\\
5.74229047593089	0.000152220254676649\\
5.74857995071285	0.000159843679284677\\
5.75486942549482	0.00016731823234295\\
5.76115890027678	0.000174617578522884\\
5.76744837505874	0.000181715908183833\\
5.7737378498407	0.000188587173397536\\
5.78002732462266	0.000195205522686592\\
5.78631679940462	0.000201545464733499\\
5.79260627418659	0.000207581866561668\\
5.79889574896855	0.000213290102692554\\
5.80518522375051	0.00021864594782528\\
5.81147469853247	0.000223625826038187\\
5.81776417331443	0.000228206921747187\\
5.82405364809639	0.000232366986892885\\
5.83034312287835	0.000236084812058834\\
5.83663259766032	0.00023934013552207\\
5.84292207244228	0.00024211359959736\\
5.84921154722424	0.000244386843405664\\
5.8555010220062	0.000246142779360525\\
5.86179049678816	0.00024736571322137\\
5.86807997157012	0.000248040803853655\\
5.87436944635209	0.00024815487813612\\
5.88065892113405	0.000247695912548807\\
5.88694839591601	0.000246653649810469\\
5.89323787069797	0.000245019135036273\\
5.89952734547993	0.000242784928559558\\
5.90581682026189	0.000239945165958488\\
5.91210629504385	0.000236495685385307\\
5.91839576982582	0.000232434242207091\\
5.92468524460778	0.000227759664994664\\
5.93097471938974	0.000222473063331563\\
5.9372641941717	0.000216576998354867\\
5.94355366895366	0.000210075675568078\\
5.94984314373562	0.000202975414140383\\
5.95613261851759	0.000195284002984408\\
5.96242209329955	0.000187011177331442\\
5.96871156808151	0.00017816814215621\\
5.97500104286347	0.000168767903232947\\
5.98129051764543	0.000158825390826678\\
5.98757999242739	0.000148357030411717\\
5.99386946720936	0.000137381081003696\\
6.00015894199132	0.000125917009427212\\
6.00644841677328	0.000113986319774995\\
6.01273789155524	0.000101611851277994\\
6.0190273663372	8.88179674802814e-05\\
6.02531684111916	7.56302870286163e-05\\
6.03160631590112	6.20759565208573e-05\\
6.03789579068309	4.81835122627672e-05\\
6.04418526546505	3.39825492119417e-05\\
6.05047474024701	1.95038555830251e-05\\
6.05676421502897	4.77948196930811e-06\\
6.06305368981093	-1.01579389593098e-05\\
6.06934316459289	-2.52743884630036e-05\\
6.07563263937485	-4.05353530368302e-05\\
6.08192211415682	-5.59056134079583e-05\\
6.08821158893878	-7.13489789632149e-05\\
6.09450106372074	-8.68289171194192e-05\\
6.1007905385027	-0.000102308495115722\\
6.10708001328466	-0.000117750463687116\\
6.11336948806662	-0.000133117155201035\\
6.11965896284859	-0.000148371032992145\\
6.12594843763055	-0.000163474345754366\\
6.13223791241251	-0.000178389414941194\\
6.13852738719447	-0.000193078991287621\\
6.14481686197643	-0.000207505825528642\\
6.15110633675839	-0.000221633075852878\\
6.15739581154036	-0.000235424726270139\\
6.16368528632232	-0.000248845193709712\\
6.16997476110428	-0.000261859375314089\\
6.17626423588624	-0.000274433390586637\\
6.1825537106682	-0.000286534064798616\\
6.18884318545016	-0.000298129252769286\\
6.19513266023212	-0.000309188075334532\\
6.20142213501409	-0.000319680595566751\\
6.20771160979605	-0.00032957837174763\\
6.21400108457801	-0.000338854348228779\\
6.22029055935997	-0.000347482844517799\\
6.22658003414193	-0.000355439795384882\\
6.23286950892389	-0.000362702623533551\\
6.23915898370586	-0.000369250657968223\\
6.24544845848782	-0.000375064890249632\\
6.25173793326978	-0.000380127938115038\\
6.25802740805174	-0.000384424598451005\\
6.2643168828337	-0.00038794164720457\\
6.27060635761566	-0.000390667180909077\\
6.27689583239763	-0.000392592162825167\\
6.28318530717959	-0.000393708862247877\\
};
%\addlegendentry{data1}

\end{axis}
\end{tikzpicture}%
}
\caption{Plot eines Interpolanten über den Rand des Gebietes}
\label{fig:rand}
\end{figure}

Wir stellen zur Vereinfachung zunächst fest, dass es genügt konstante Nullrandwerte in der \ac{PDE} zu betrachten. Sei dafür wieder folgende \ac{PDE} gegeben:
\begin{align*}
Lu(x) &= f(x), x\in \Omega\\
Bu(x) &= g(x), x \in \partial \Omega
\end{align*}
Wir können annehmen, dass eine Funktion $\bar{g} \in C^2(\Omega)$ existiert mit $\bar{g}|_{\partial \Omega} = g$. Damit gilt $u = \bar{u} + \bar{g}$ für eine Funktion $\bar{u}$. Eingesetzt erhalten wir
\begin{align*}
L\bar{u}(x) + L\bar{g}(x) &= f(x) , x \in \Omega\\
B\bar{u} + B\bar{g} &= g(x) , x \in \partial \Omega
\end{align*}
, was äquivalent dazu ist, dass wir folgende \ac{PDE} nach $\bar{u}$ lösen:
\begin{align*}
L\bar{u}(x) &= f(x) + L\bar{g}(x), x \in \Omega\\
\bar{u}(x) &= 0, x \in \partial \Omega
\end{align*}

Die Idee ist jetzt einen Kern bzw. Ansatzfunktionen zu konstruieren, der auf dem Rand von $\Omega$ Null ist. Der Interpolant ist eine Linearkombination aus diesen Funktionen und wird demnach auf dem Rand auch Null sein. Dafür führen wir Gewichtsfunktionen ein, die dann in Verbindung mit einem gegebenen Kern das Geforderte erfüllen werden.
\section{Gewichtsfunktionen}
\begin{definition}
Sei $\Omega \subset \mathbb{R}^n$ offen und beschränkt. Eine Funktion $w:\mathbb{R}^n \rightarrow \mathbb{R}$ heißt Gewichtsfunktion auf $\Omega$, wenn gilt:
\begin{enumerate}
\item $w(x) > 0$ für alle $x \in \Omega$
\item $w(x) = 0$ für alle $x \in \partial \Omega$
\item $w(x) < 0$ für alle $x \in \mathbb{R}^n \setminus \bar{\Omega}$
\end{enumerate}
\end{definition}

\begin{theorem}
\label{thm:Gewicht}
Seien $\Omega_1, \Omega_2 \subset \mathbb{R}^n$ zwei offene und beschränkte Mengen und $w_{1}, w_{2}$ die dazugehörigen Gewichtsfunktionen. Dann gilt:
\begin{enumerate}
\item Für das Komplement $\Omega^\mathrm{C}$: $w = -w_1$
\item Für die Vereinigung $\Omega_1 \cup \Omega_2$: $w = w_1 + w_2 + \sqrt{w_1^2 + w_2^2}$
\item Für den Schnitt $\Omega_1 \cap \Omega_2$: $w = w_1 + w_2 - \sqrt{w_1^2 + w_2^2}$
\end{enumerate}
\end{theorem}
\begin{proof}
\begin{enumerate}
\item
\begin{itemize}
\item
Sei $x \in \Omega^\mathrm{C}$.
\begin{align*}
w(x) = - w_1(x) > 0
\end{align*}
\item
Sei $x \in \partial \Omega^\mathrm{C}$
\begin{align*}
w(x) = -w_1(x) = 0
\end{align*}
\item
Sei $x \in \Omega$
\begin{align*}
w(x) = -w_1(x) < 0
\end{align*}
\end{itemize}
\item
\begin{itemize}
\item
Sei $x \in \Omega_1, x \in \Omega_2$. $\Rightarrow w_1 >0, w_2>0$
\begin{align*}
w = w_1 + w_2 + \sqrt{w_1^2 + w_2^2} > 0
\end{align*}
\item
Sei \ac{oBdA} $x \in \Omega_1, x \notin \bar{\Omega_2}$. $\Rightarrow w_1 > 0, w_2  0$
\begin{align*}
w =  w_1 + w_2 + \sqrt{w_1^2 + w_2^2} > w_1 + w_2 + \underbrace{\sqrt{w_2^2}}_{=|w_2| = -w_2} = w_1+w_2-w_2 = w_1 > 0
\end{align*}
\item
Sei $x \notin \bar{\Omega_1}, x \notin \bar{\Omega_2}$. $\Rightarrow w_1 < 0, w_2 < 0$
\begin{align*}
&w = w_1 + w_2 + \sqrt{w_1^2 + w_2^2} \overset{!}{<} 0\\
&\Leftrightarrow -w_1 - w_2 > \sqrt{w_1^2 + w_2^2}\\
&\Leftrightarrow w_1^2 + \underbrace{2w_1 w_2}_{>0} +w_2^2 > w_1^2 + w_2^2
\end{align*}
\item
Sei \ac{oBdA} $x \in \partial \Omega_1, x \notin \Omega_2$. $\Rightarrow w_1 = 0, w_2 \leq 0$
\begin{align*}
w =  w_1 + w_2 + \sqrt{w_1^2 + w_2^2} = w_2 + \sqrt{w_2^2} = w_2 - w_2 = 0
\end{align*}
\item
Sei \ac{oBdA} $x \in \partial \Omega_1, x \in \Omega_2$. $\Rightarrow w_1 = 0, w_2 > 0$
\begin{align*}
w = w_1 + w_2 + \sqrt{w_1^2 + w_2^2} = w_2 + \sqrt{w_2^2} > 0
\end{align*}
\end{itemize}
\item
\begin{itemize}
\item
Sei $x \in \Omega_1, x \in \Omega_2$. $\Rightarrow w_1 >0, w_2 >0$
\begin{align*}
&w = w_1 + w_2 - \sqrt{w_1^2 + w_2^2} \overset{!}{>}0\\
&\Leftrightarrow w_1 + w_2 > \sqrt{w_1^2 + w_2^2}\\
&\Leftrightarrow (w_1 + w_2)^2 > w_1^2 + w_2^2\\
&\Leftrightarrow w_1^2 + 2w_1 w_2 + w_2^2 > w_1^2 + w_2^2\\
&\Leftrightarrow 2w_1 w_2 >0
\end{align*}
\item
Sei \ac{oBdA} $x \in \Omega_1, x \notin \Omega_2$. $\Rightarrow w_1 >0, w_2 <0$
\begin{align*}
w = w_1 + w_2 - \sqrt{w_1^2 + w_2^2} < w_1 + w_2 - \sqrt{w_1^2} = w_1+w_2-w_1 = w_2 <0
\end{align*}
\item
Sei \ac{oBdA} $x \in \partial \Omega_1, x \in \bar{\Omega_2}$. $w_1 = 0, w_2 > 0$
\begin{align*}
w = w_1 + w_2 - \sqrt{w_1^2+w_2^2} = w_2 - \sqrt{w_2^2} = 0
\end{align*}
\item
Sei \ac{oBdA} $x \in \partial \Omega_1, x \notin \bar{\Omega_2}$. $w_1 = 0, w_2 < 0$
\begin{align*}
w = w_1 + w_2 - \sqrt{w_1^2 + w_2^2} = w_2 - \sqrt{w_2^2} = 2w_2 < 0
\end{align*}
\item
Sei $x \notin \bar{\Omega_1}, x \notin \bar{\Omega_2}$. $\Rightarrow w_1 < 0, w_2 < 0$
\begin{align*}
w = w_1 + w_2 - \sqrt{w_1^2 + w_2^2} < 0
\end{align*}
\end{itemize}
\end{enumerate}
\end{proof}

\begin{example}
Sei $\Omega = [-1,1] \times [-1,1]$. Dann können wir $\Omega$ schreiben als $\Omega = \Omega_1  \cap \Omega_2$ mit $\Omega_1 = [-1,1] \times (- \infty, \infty), \Omega_2 =   (- \infty, \infty) \times [-1,1]$.
Dann sind 
\begin{align*}
w_1(x,y) &= -x^2 +1\\
w_2(x,y) &= -y^2 +1 
\end{align*}
Gewichtsfunktionen auf $\Omega_1$ bzw, $\Omega_2$. Nach Satz \ref{thm:Gewicht} ist dann die Gewichtsfunktion für $\Omega$ gegeben durch:
\begin{align*}
w(x,y) &= w_1(x,y) + w_2(x,y) - \sqrt{w_1(x,y)^2 + w_2(x,y)^2}\\
&= -x^2 +1 -y^2 +1 - \sqrt{(-x^2+1)^2 + (-y^2+1)^2}\\
&= -x^2-y^2+2 - \sqrt{x^4 -2x^2 + y^4 -2y^2+2}
\end{align*}
\end{example}
Wir möchten jetzt einen Kern und Gewichtsfunktion verknüpfen und bekommen damit eine neue Funktion, die auf dem Rand unseres Definitionsgebiets konstant Null ist. Dazu betrachten wir wieder zwei verschiedene Ansätze.

\section{Symmetrische Kollokation}
\begin{theorem}
\label{thm:gewichtKern}
Sei $\Omega$ eine Menge, $K':\Omega \times \Omega \rightarrow \mathbb{R}$ ein \ac{PD} Kern und $g:\Omega \rightarrow \mathbb{R} \setminus \{0\}$ eine Funktion. Dann ist 
\begin{align*}
K(x,y) := g(x)K'(x,y)g(y)
\end{align*}
ein Kern und es gilt für den entsprechenden \ac{RKHR}:
\begin{align*}
\mathcal{H}_{K}(\Omega) = g \mathcal{H}_{K'}(\Omega) := \left\{ gf|f \in \mathcal{H}_K'(\Omega)\right\}
\end{align*}
\end{theorem}
\begin{proof}
Wir zeigen zunächst, dass $\tilde{K}(x,y):= g(x)g(y)$ ein \ac{PD} Kern ist.

Die Symmetrie ist offensichtlich, da
\begin{align*}
\tilde{K}(x,y)= g(x)g(y) = g(y) g(x) = \tilde{K}(y,x)
\end{align*}

Zur positiven Definitheit betrachten wir eine Punktmenge $X_N := \{x_i \in \Omega| 1 \le i \le N \}\subset \Omega$. Wir erhalten für die Kernmatrix
\begin{align*}
A = 
\begin{pmatrix}
g(x_1)g(x_1) & \cdots & g(x_1)g(x_N) \\ 
\vdots & \ddots & \vdots \\ 
g(x_N)g(x_1) & \cdots & g(x_N)g(x_N)
\end{pmatrix} 
=
\begin{pmatrix}
g(x_1) \\ 
\vdots \\ 
g(x_N)
\end{pmatrix}
\begin{pmatrix}
g(x_1) & \cdots & g(x_N)
\end{pmatrix}
:= \bar{g}\bar{g}^T
\end{align*}
und damit für alle $\alpha \neq 0$
\begin{align*}
\alpha^T A \alpha = \alpha^T \left(\bar{g}\bar{g}^T\right)\alpha = \left(\alpha^T \bar{g}\right)\left(\bar{g}^T \alpha\right) = \left\| \bar{g}^T\alpha\right\| \geq 0
\end{align*}
Also ist $\tilde{K}$ ein \ac{PD} Kern.

Nach Satz \ref{thm:Kombi} ist $K(x,y) = \tilde{K}(x,y) K'(x,y) = g(x) K'(x,y) g(y)$ ein \ac{PD} Kern.

Es fehlt noch der zweite Teil des Satzes. Dafür stellen wir zunächst fest, dass für alle $y \in \Omega$
\begin{align*}
K(\cdot,y) = g(\cdot) K'(\cdot,y) g(y) \in g \mathcal{H}_K' (\Omega)
\end{align*}
Als nächstes zeigen wir, dass $\mathcal{H}_K(\Omega)$ tatsächlich ein Hilbertraum ist. Sei dafür
\begin{align*}
s : \mathcal{H}_{K'} (\Omega) &\rightarrow g\mathcal{H}_{K'} (\Omega)\\
f &\mapsto gf
\end{align*}
$s$ ist bijektiv, da $g \neq 0$ ist. Damit können wir auf $\mathcal{H}_K (\Omega)$ ein Skalarprodukt definieren:
\begin{align*}
\left(\cdot, \cdot \right)_{\mathcal{H}_K(\Omega)} : \mathcal{H}_K(\Omega) \times \mathcal{H}_K(\Omega) &\rightarrow \mathbb{R}\\
(gf, gh) &\mapsto \left(s^{-1}(gf), s^{-1}(gh)\right)_{\mathcal{H}_{K'}(\Omega)} = \left( f,h \right)_{\mathcal{H}_{K'}(\Omega)}
\end{align*}
Damit wird $\mathcal{H}_K(\Omega)$ zu einem Hilbertraum. 

Wir zeigen noch die Reproduzierbarkeit auf $\mathcal{H}_K(\Omega)$, dann folgt aus der Eindeutigkeit des Kerns aus Satz \ref{thm:EindeutigkeitKern} die Behauptung. Sei dafür $x \in \Omega$ und $h = gf \in \mathcal{H}_K (\Omega)$.
\begin{align*}
\left(h, K(\cdot,x) \right)_{\mathcal{H}_K(\Omega)} &= \left(gf, gK'(\cdot, x) g(x)\right)_{\mathcal{H}_K(\Omega)}\\
&= g(x) \left( gf, gK'(\cdot, x)\right)_{\mathcal{H}_K(\Omega)}\\
&= g(x) \left( f, K'(\cdot, x)\right)_{\mathcal{H}_{K'}(\Omega)}\\
&= g(x) f(x)\\
&= h(x)
\end{align*}
\end{proof}

Wir haben jetzt einen neuen Kern konstruiert, der auf dem Rand unseres Definitionsgebiets konstant Null ist. Wenn wir jetzt noch zusätzlich annehmen, dass auch die Ableitung der Gewichtsfunktion wieder ein Gewichtsfunktion ist, können wir die Konstruktion aus Kapitel \ref{sec:SymKol} verwenden. 

Sei also $\Omega \subset \mathbb{R}^n$ offen und beschränkt $K'$ ein \ac{PD} Kern, $g$ eine Gewichtsfunktion auf $\Omega$, für die auch dere Ableitung eine Gewichtsfunktion auf $\Omega$ ist, und folgende \ac{PDE} gegeben:
\begin{align*}
Lu(x) &= f(x), x \in \Omega\\
u(x) &= 0 , x \in \partial \Omega
\end{align*}
Für ein $N \in \mathbb{N}$ betrachten wir eine Menge $X_N \subset \Omega^\circ$.

Wir definieren die Menge $\Lambda_N = \left\{ \lambda_1, \dots, \lambda_N\right\}$ mit $\lambda_i =  \delta_{x_i} \circ L$. Diese Funktionale sind im von $K(x,y) := g(x) K'(x,y) g(y)$ erzeugten \ac{RKHR} stetig. Also wählen wir 
\begin{align*}
V_N &= \text{span} \left\{\lambda_1^y K(x,y), \dots, \lambda_N^y K(x,y)\right\}\\
&= \text{span} \left\{(\delta_{x_1} \circ L)^y (g(x) K'(x,y) g(y)), \dots, (\delta_{x_N} \circ L)^y (g(x) K'(x,y) g(y))\right\}
\end{align*}
als Ansatzfunktionen.

Damit erhalten wir folgenden Interpolanten:
\begin{align*}
s_u(x) &= \sum_{j=1}^N \alpha_j \lambda_j^y K(x,y)\\
&= \sum_{j=1}^N \alpha_j (\delta_{x_j} \circ L)^y( g(x)K'(x,y)g(y))
\end{align*}

Die $\alpha_j$ erhält man als Lösung des \ac{LGS} $A\alpha = b$ mit 
\begin{align*}
A_{i,j} &= (\delta_{x_i} \circ L)^x (\delta_{x_j} \circ L)^y (g(x)K'(x,y)g(y))\\
b_i &= f(x_i)
\end{align*}

Die Matrix $A$ ist wieder symmetrisch und positiv definit und das \ac{LGS} ist damit lösbar.
\section{Nicht-Symmetrische Kollokation}
Wie bei der Standardkollokation können wir einen wesentlich simpleren Ansatz wählen. Es sei die gleiche Problemstellung wie gerade gegeben, allerdings haben dieses Mal keine zusätzliche Anforderung an die Ableitung der Gewichtsfunktion. Wir wählen 
\begin{align*}
V_N:= \text{span} \left\{g(x)K'(x,x_1), \dots, g(x)K'(x,x_N)\right\}
\end{align*}
als Ansatzfunktionen und bekommen damit folgenden Interpolanten:
\begin{align*}
s_u (x) = \sum_{j=1}^N \alpha_j g(x)K'(x,x_j)
\end{align*}
Die $\alpha_j$ erhält man als Lösung des \ac{LGS} $A\alpha = b$ mit 
\begin{align*}
A_{i,j} &= (\delta_{x_i} \circ L)^x (g(x) K(x,x_j))\\
b_i &= f(x_i)
\end{align*}

Erneut kann man keine Aussage über die Lösbarkeit des \ac{LGS} treffen.
\chapter{Implementation}
\label{cha:Implementation}


%\include{chapters/literatur}
%\include{chapters/drucken}
%\chapter{Zusammenfassung und Ausblick}
\label{cha:schluss}

In diesem Kapitel fassen wir die wichtigsten Erkenntnisse dieser Arbeit zusammen und zeigen interessante weiterführende Fragen auf.

Wir haben in Kapitel \ref{cha:Grundlagen} und \ref{cha:Standardkollokation} die bekannte Theorie der Kernkollokation vorgestellt, welche wir in Kapitel \ref{cha:Gewichtet} um Gewichtsfunktionen erweitert haben. Diese könnte man noch weiter untersuchen, auch im Hinblick darauf, Gewichtsfunktionen für beliebige Gebiete mit möglichst wenig Singularitäten in ihren Ableitungen zu finden. Als Ansatz dafür bietet sich die in Kapitel \ref{sec:andereGewicht} experimentell vorgestellte Methode an, in der wir eine Gewichtsfunktion nur auf einer Umgebung gegeben hatten.

Daraufhin haben wir in Kapitel \ref{cha:Implementierung} eine Implementierung der theoretisch hergeleiteten Verfahren vorgestellt, welche wir in Kapitel \ref{cha:NumerischeTests} ausgewertet haben. Dort haben wir sehr verschiedene Ergebnisse für unterschiedliche \acp{PDE} erhalten. Insbesondere haben die gewichteten Verfahren teilweise sehr schlechte Ergebnisse geliefert. Diese Beobachtung haben wir auf die Singularitäten der partiellen Ableitungen der Gewichtsfunktionen zurückgeführt.

Wir haben außerdem verschiedene Möglichkeiten der Kollokationspunktwahl untersucht. Dabei haben wir festgestellt, dass eine Greedy-Punktwahl bezüglich der Anzahl der Kollokationspunkte die geschickteste der vorgestellten Wahlmöglichkeiten ist. Hier wäre es noch interessant einen anderen Fehlerschätzer zur Punktwahl zu benutzen, beispielsweise mit den $C^1, C^2, \dots$ Normen. Außerdem wäre eine Greedy-Punktwahl wünschenswert, bei der die Punkte auf dem Rand nicht festgesetzt sind, sondern auch erst nach und nach gesetzt werden. Dafür könnte man auf dem Rand einen zweiten Fehlerschätzer einführen und diesen mithilfe einer Konstante gegen den Fehler im Inneren abwägen.

Wir haben die Kernkollokation zum Großteil nur in zwei Dimensionen getestet, haben aber gesehen, dass einer Umsetzung in höheren Dimensionen nichts im Wege steht. Diesen Teil haben wir allerdings nur kurz angeschnitten und man könnte dort noch weitere Experimente durchführen, um auf eine größere Allgemeingültigkeit zu schließen.

%%%----------------------------------------------------------
\appendix                                            % Anhang 
%%%----------------------------------------------------------

%\chapter{Technische Informationen}
\label{app:TechnischeInfos}
\begin{acronym}[PDE]
\acro{PDE}{partielle Differentialgleichung}
\acrodefplural{PDE}[PDEs]{partielle Differentialgleichungen}
\end{acronym}

\begin{acronym}[PD]
\acro{PD}{positiv definit}
\end{acronym}

\begin{acronym}[SPD]
\acro{SPD}{strikt positiv definit}
\end{acronym}

\begin{acronym}[RKHR]
\acro{RKHR}{reproduzierender Kern Hilbert Raum}
\end{acronym}

\begin{acronym}[LGS]
\acro{LGS}{lineares Gleichungssystem}
\end{acronym}	% Technische Ergänzungen
%\include{back/anhang_b}	% Inhalt der CD-ROM/DVD
%\include{back/anhang_c}	% Chronologische Liste der Änderungen
	% Quelltext dieses Dokuments

%%%----------------------------------------------------------
\listoffigures
\addcontentsline{toc}{chapter}{Abbildungsverzeichnis}
\chapter*{Abkürzungsverzeichnis}
\label{app:abkuerzungen}
\addcontentsline{toc}{chapter}{Abkürzungsverzeichnis}

\begin{acronym}[PDE]
\acro{PDE}{partielle Differentialgleichung}
\acrodefplural{PDE}[PDEs]{partielle Differentialgleichungen}
\end{acronym}

\begin{acronym}[PD]
\acro{PD}{positiv definit}
\end{acronym}

\begin{acronym}[SPD]
\acro{SPD}{strikt positiv definit}
\end{acronym}

\begin{acronym}[RKHR]
\acro{RKHR}{reproduzierender Kern Hilbert Raum}
\end{acronym}

\begin{acronym}[LGS]
\acro{LGS}{lineares Gleichungssystem}
\end{acronym}

\begin{acronym}[oBdA]
\acro{oBdA}{ohne Beschränkung der Allgemeinheit}
\end{acronym}
\MakeBibliography                        % Quellenverzeichnis
%%%----------------------------------------------------------

%%% Messbox zur Druckkontrolle ------------------------------
%\include{back/messbox}

%%%----------------------------------------------------------
\end{document}
%%%----------------------------------------------------------