\chapter{Einleitung}
\label{cha:Einleitung}


Unser Ziel ist es Lösungen von \acp{PDE} zu approximieren. Diese sind allgemein gegeben durch:
\begin{align*}
L u(x) &= f(x), x \in \Omega \\
B u(x) &= g(x), x \in \partial \Omega
\end{align*}
, wobei $\Omega \subset \mathbb{R}^n$, $L$ ein linearer, beschränkter Differentialoperator und $B$ ein linearer, beschränkter Auswertungsoperator ist.\\
Für den größten Teil dieser Arbeit werden wir folgende \ac{PDE} im $\mathbb{R}^2$ betrachten:
\begin{align*}
\Delta u(x) &= f(x), x \in \Omega \\
u(x) &= 0 , x \in \partial \Omega
\end{align*}

Es genügt die Nullrandbedingung zu betrachten, da jede \ac{PDE} auf eine mit Nullrandbedingung umgeformt werden kann.\\
HIER KOMMT DIE BEGRÜNDUNG!!

Wir wollen zur Approximation der \ac{PDE} einen interpolierenden Ansatz wählen. Dazu müssen wir die Interpolation zunächst verallgemeinern.

\begin{definition}
Sei $\Omega \subset \mathbb{R}^n$ eine nicht leere Menge, $\mathcal{H}$ ein Hilbertraum mit Funktionen $f:\Omega \rightarrow \mathbb{R}$, $u \in \mathcal{H}$  und $\Lambda_N := \{\lambda_1, \dots, \lambda_N\} \subset \mathcal{H}'$ eine Menge von linearen, stetigen und linear unabhängigen Funktionalen. Dann ist eine Funktion $s_u \in \mathcal{H}$ der gesuchte Interpolant, wenn gilt, dass
\begin{align*}
\lambda_i(s) = \lambda_i(s_u) , 1\le i \le N
\end{align*}
\end{definition}

\begin{example}
\begin{itemize}
\item
Sei $X_N := \{x_1, \dots, x_N\} \subset \Omega$ eine Menge von Punkten und $\Lambda_N := \{\delta_{x_1}, \dots,\delta_{x_N}\} \subset \mathcal{H}'$ die Punktauswertungsfunktionale $\delta_{x_i}(f) = f(x_i), 1\le i \le N$. (STETIG?!) Dann bekommen wir die Standardinterpolation mit
\begin{align*}
s(x_i) = \delta_{x_i}(s) = \delta_{x_i}(s_u) = s_u(x_i), 1\le i \le N
\end{align*}
\item
Mit $\lambda_i := \delta_{x_i} \circ D^a$ für einen Multiindex $a \in \mathbb{N}_0^d$ erhält man noch zusätzliche Informationen über die Ableitung der Funktion.
\item
Sei eine \ac{PDE} gegeben:
\begin{align*}
L u(x) &= f(x), x \in \Omega \\
B u(x) &= g(x), x \in \partial \Omega
\end{align*}
Sei $X_N \subset \Omega$ eine Menge an Kollokationspunkten. Dann möchten wir, dass $s_u$ die \ac{PDE} in den Punkten $X_N$ erfüllt, also:
\begin{align*}
L s_u(x_i) &= L u(x_i) = f(x_i), x_i \in \Omega \\
B s_u(x_i) &= B u(x_i) = g(x_i), x_i \in \partial \Omega
\end{align*}
\end{itemize}
\end{example}

Wir müssen einen geeigneten Ansatz wählen um das Interpolationsproblem zu lösen, also einen $N$-dimensionalen Unterraum $V_N := \text{span}\{\nu_1, \dots, \nu_N\} \subset \mathcal{H}$ und fordern, dass $s_u \in V_N$, also 
\begin{align*}
s_u(x) := \sum_{j=1}^N \alpha_j \nu_j(x), x \in \Omega
\end{align*}
Also lassen sich die Interpolationsbedingungen schreiben als:
\begin{align*}
\lambda_i(u) = \lambda_i(s_u) = \sum_{j=1}^N \alpha_j \lambda_i(\nu_j)
\end{align*}
Diese Bedingungen lassen sich umschreiben als lineares Gleichungssystem $A_\Lambda \alpha = b$ mit $(A_\Lambda)_{i,j} := \lambda_i(\nu_j), b_i := \lambda_i(u)$.

\section{Standardkollokation}
Wir suchen jetzt nach geeigneten Ansatzfunktionen und einem Hilbertraum, in dem die Auswertungs- und Differentialfunktionale stetig sind. Dies führt uns zur Definition von Kern Funktionen mit denen wir einen Hilbertraum konstruieren werden, der uns das Geforderte liefern wird.

\begin{definition}
Sei $\Omega$ eine nicht leere Menge. Ein reeller Kern auf $\Omega$ ist eine symmetrische Funktion $K: \Omega \times \Omega \rightarrow \mathbb{R}$.\\
Für alle $N \in \mathbb{N}$ und für eine Menge $X_N = \{x_i\}_{i=1}^N$ ist die Kern Matrix (oder Gram'sche Matrix) $A:= A_{K,X_N} \in \mathbb{R}^{N \times N}$  definiert als $A:=[K(x_i, x_j)]_{i,j=1}^N$.\\
Ein Kern $K$ heißt \ac{PD} auf $\Omega$, wenn für alle $N \in \mathbb{N}$ und alle Mengen $X_N$ mit paarweise verschiedenen Elementen $x_{i=1}^N$ gilt, dass die Kern Matrix positiv definit ist. Der Kern $K$ heißt \ac{SPD}, falls die Kern Matrix strikt positiv definit ist.
\end{definition}

\begin{example} Sei $\Omega \subset \mathbb{R}^n$. Dann sind folgende Funktionen Kerne auf $\Omega$:\\
\begin{itemize}
\item $K(x,y) := \exp(-\gamma \|x-y\|),\gamma > 0$
\item $K(x,y) := (x,y)$
\end{itemize}
\end{example}

Wir kommen mit dieser Definition direkt zu den gesuchten Hilberträumen.

\begin{definition}[Reproduzierender Kern Hilbertraum]
Sei $\Omega$ eine nicht leere Menge und $\mathcal{H}$ ein Hilbertraum mit Funktionen $f:\Omega \rightarrow \mathbb{R}$ und Skalarprodut $(\cdot, \cdot)_\mathcal{H}$. Dann nennt man $\mathcal{H}$ einen \ac{RKHR} auf $\Omega$, wenn eine Funktion $K:\Omega \times \Omega \rightarrow \mathbb{R}$ existiert, sodass
\begin{enumerate}
\item $K(\cdot, x) \in \mathcal{H}$ für alle $x \in \Omega$
\item $(f, K(\cdot,x))_\mathcal{H} = f(x)$ für alle $ x \in \Omega$, $f \in \mathcal{H}$
\end{enumerate}
\end{definition}

Bei Interpolationsproblemen haben wir zunächst einen Kern $K$ gegeben und wollen damit eine Funktion approximieren. Also stellt sich die Frage ob zu jedem Kern $K$ ein \ac{RKHR} existiert. Diese wird durch folgenden Satz beantwortet:

\begin{theorem}[Moore, Aronszajn]
Sei $\Omega$ eine nicht leere Menge und $K:\Omega \times \Omega \rightarrow \mathbb{R}$ ein positiv definiter Kern. Dann existiert genau ein \ac{RKHR} $\mathcal{H}_K (\Omega)$ mit reproduzierendem Kern $K$.
\end{theorem}
\begin{proof}
SIEHE SKRIPT!
\end{proof}

Mit diesem Wissen können wir uns erste Eigenschaften von \ac{RKHR} anschauen:

\begin{theorem}
Sei $\Omega$ eine nicht leere Menge und $\mathcal{H}$ ein Hilbertraum mit Funktionen $f: \Omega \rightarrow \mathbb{R}$. Dann gilt:
\begin{enumerate}
\item $\mathcal{H}$ ist genau dann ein \ac{RKHR}, wenn die Auswertungsfunktionale stetig sind.
\item Wenn $\mathcal{H}$ ein \ac{RKHR} mit Kern $K$ ist, dann ist $K(\cdot,x)$ der Riesz-Repräsentant des Funktionals $\delta_x \in \mathcal{H}'$.
\item Ist $\Omega \subset \mathbb{R}^n$ offen und $K \in C^{2k}(\Omega \times \Omega)$ für $k \in \mathbb{N}$, dann ist $\mathcal{H}_K(\Omega) \subset C^k(\Omega)$ und es gilt für alle Multiindizes $a:= (a_1,\dots, a_d)$:
\begin{align*}
D^a f(x):= \partial_{x^{(1)}}^{a_1}\partial_{x^{(2)}}^{a_2} \dots \partial_{x^{(d)}}^{a_d} f(x) = (f, D_2^a K(\cdot,x))_{\mathcal{H}_K(\Omega)}
\end{align*}
\item LINEARE UNABHÄNGIGKEIT
\end{enumerate}
\end{theorem}

\begin{proof}
\begin{enumerate}
\item Für alle $f \in \mathcal{H}$ und alle $x \in \Omega$ gilt:
\begin{align*}
|f(x)| &= |(f, K(\cdot,x))_\mathcal{H}| \le \|f\|_\mathcal{H}\|K(\cdot,x)\|_\mathcal{H}\\
&= \|f\|_\mathcal{H} \sqrt{(K(\cdot,x),K(\cdot,x))_\mathcal{H}} = \|f\|_\mathcal{H} \sqrt{K(x,x)}
\end{align*}
, wobei für die erste und die letzte Gleichung die Reproduzierbarkeit des Kerns benutzt wurde.

Sei $\mathcal{H}$ ein \ac{RKHR}. Dann gilt mit dem eben gezeigten:
\begin{align*}
|\delta_x(f)| &= |f(x)| \le \|f\|_\mathcal{H} \sqrt{K(x,x)}\\
\Leftrightarrow \frac{|\delta_x(f)|}{\|f\|_\mathcal{H}} &\le \sqrt{K(x,x)}
\end{align*}
Also ist $\delta_x$ beschränkt und damit stetig.

Für die andere Richtung nehmen wir an, dass $\delta_x  \in \mathcal{H}'$ für alle $x \in \Omega$. Also existiert ein Riesz-Repräsentant $\nu_{\delta_x} \in \mathcal{H}$. Definieren wir $K(\cdot,x):= \nu_{\delta_x}$, dann ist $K$ ein Kern. Es ist klar, dass $K(\cdot,x) \in \mathcal{H}$ und nach der Definition des Riesz-Repräsentanten gilt:
\begin{align*}
(f, K(\cdot,x))_\mathcal{H} = (f, \nu_{\delta_x})_\mathcal{H} = \delta_x(f) = f(x)
\end{align*}
\item Die Behauptung folgt sofort aus der Reproduzierbarkeit von $K$, da $(f, K(\cdot,x))_\mathcal{H}= f(x)$ für alle $x \in \Omega$ und alle $f \in \mathcal{H}$ gilt.
\item
Den Beweis, dass $\mathcal{H}_K(\Omega) \subset C^k(\Omega)$ werden wir hier auslassen.

REST VOM BEWEIS FEHLT AUCH NOCH!!
\item AUCH DER BEWEIS FEHLT
\end{enumerate}
\end{proof}

Damit haben wir alle nötigen Werkzeuge um die Interpolation durchzuführen. Wir haben Ansatzfunktionen $K$, den dazugehörigen Hilbertraum $\mathcal{H}_K(\Omega)$ und die Stetigkeit und lineare Unabhängigkeit aller benötigten Operatoren. Jetzt müssen wir nur noch einen geeigneten Ansatz wählen.
\subsection{Symmetrische Kollokation}
\subsection{Nicht-Symmetrische Kollokation}