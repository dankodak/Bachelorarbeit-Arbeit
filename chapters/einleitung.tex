\chapter{Einleitung}
\label{cha:Einleitung}


Unser Ziel ist es Lösungen von \acp{PDE} zu approximieren. Diese sind allgemein gegeben durch:
\begin{align*}
L u(x) &= f(x), x \in \Omega \\
B u(x) &= g(x), x \in \partial \Omega
\end{align*}
, wobei $\Omega \subset \mathbb{R}^n$, $L$ ein linearer, beschränkter Differentialoperator und $B$ ein linearer, beschränkter Auswertungsoperator ist.\\
Für den größten Teil dieser Arbeit werden wir folgende \ac{PDE} im $\mathbb{R}^2$ betrachten:
\begin{align*}
\Delta u(x) &= f(x), x \in \Omega \\
u(x) &= 0 , x \in \partial \Omega
\end{align*}

Es genügt die Nullrandbedingung zu betrachten, da jede \ac{PDE} auf eine mit Nullrandbedingung umgeformt werden kann.\\
HIER KOMMT DIE BEGRÜNDUNG!!

Wir wollen zur Approximation der \ac{PDE} einen interpolierenden Ansatz wählen. Dazu müssen wir die Interpolation zunächst verallgemeinern.

\begin{definition}
Sei $\Omega \subset \mathbb{R}^n$ eine nicht leere Menge, $\mathcal{H}$ ein Hilbertraum mit Funktionen $f:\Omega \rightarrow \mathbb{R}$, $u \in \mathcal{H}$  und $\Lambda_N := \{\lambda_1, \dots, \lambda_N\} \subset \mathcal{H}'$ eine Menge von linearen, stetigen und linear unabhängigen Funktionalen. Dann ist eine Funktion $s_u \in \mathcal{H}$ der gesuchte Interpolant von $u$, wenn gilt, dass
\begin{align*}
\lambda_i(u) = \lambda_i(s_u) , 1\le i \le N
\end{align*}
\end{definition}

\begin{example}
\begin{itemize}
\item
Sei $\Omega \subset \mathbb{R}^d$ ,$X_N := \{x_1, \dots, x_N\} \subset \Omega$ eine Menge von Punkten und $\mathcal{H}$ ein Hilbertraum mit Funktionen , in dem die Punktauswertungfunktionale $\delta_{x_i}(f) = f(x_i), 1\le i \le N$  stetig sind. Dann bekommen wir die Standardinterpolation mit $\Lambda_N := \{\delta_{x_1}, \dots,\delta_{x_N}\} \subset \mathcal{H}'$.
\begin{align*}
s(x_i) = \delta_{x_i}(s) = \delta_{x_i}(s_u) = s_u(x_i), 1\le i \le N
\end{align*}
\item
Mit $\lambda_i := \delta_{x_i} \circ D^a$ für einen Multiindex $a \in \mathbb{N}_0^d$ erhält man noch zusätzliche Informationen über die Ableitung der Funktion.
\item
Sei eine \ac{PDE} gegeben:
\begin{align*}
L u(x) &= f(x), x \in \Omega \\
B u(x) &= g(x), x \in \partial \Omega
\end{align*}
Sei $X_N \subset \Omega$ eine Menge an Kollokationspunkten. Dann möchten wir, dass $s_u$ die \ac{PDE} in den Punkten $X_N$ erfüllt, also:
\begin{align*}
L s_u(x_i) &= L u(x_i) = f(x_i), x_i \in \Omega \\
B s_u(x_i) &= B u(x_i) = g(x_i), x_i \in \partial \Omega
\end{align*}
\end{itemize}
\end{example}

Wir müssen einen geeigneten Ansatz wählen um das Interpolationsproblem zu lösen, also einen $N$-dimensionalen Unterraum $V_N := \text{span}\{\nu_1, \dots, \nu_N\} \subset \mathcal{H}$ und fordern, dass $s_u \in V_N$, also 
\begin{align*}
s_u(x) := \sum_{j=1}^N \alpha_j \nu_j(x), x \in \Omega, \alpha \in \mathbb{R}^N
\end{align*}
Also lassen sich die Interpolationsbedingungen schreiben als:
\begin{align*}
\lambda_i(u) = \lambda_i(s_u) = \sum_{j=1}^N \alpha_j \lambda_i(\nu_j)
\end{align*}
Diese lassen sich auch als lineares Gleichungssystem $A_\Lambda \alpha = b$ schreiben  mit $(A_\Lambda)_{i,j} := \lambda_i(\nu_j), b_i := \lambda_i(u)$.



