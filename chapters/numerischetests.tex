\chapter{Numerische Tests}
\label{cha:NumerischeTests}

In diesem Kapitel möchten wir die vorgestellten Verfahren verschiedenen numerischen Tests unterziehen. Dabei wollen wir hauptsächlich den Fehler der numerischen Lösung zur analytischen untersuchen, aber auch die Kondition der Kollokationsmatrizen, die Parameterwahl und die Laufzeit betrachten. Zusätzlich werden wir die Verfahren mit der finiten Elemente Methode (\acs{FEM}) vergleichen.

Dafür betrachten wir die Poisson-Gleichung mit Dirichlet-Randbedingung auf $\Omega = [-1,1] \times [-1,1] \subset \mathbb{R}^n$:
\begin{align*}
- \Delta u &= 2\pi^2 \sin(\pi x)\sin(\pi y)&&, (x,y) \in \Omega\\
u &= 0&&, (x,y) \in \partial \Omega
\end{align*}
mit analytischer Lösung $u(x,y) = \sin(\pi x)\sin(\pi y)$.

Diese werden wir mithilfe der Kernkollokation numerisch lösen. Dafür wählen wir den Gauß Kern aus Beispiel \ref{ex:Kern}.

\section{Fehler}
Wir werden die Kollokationspunkte, so wie die Testpunkte, zunächst, wie in Abbildung \ref{fig:Kollok} gezeigt, in einem Gitter anordnen.
\begin{figure}[h]
\centering
\resizebox {.8\columnwidth} {!} {
% This file was created by matlab2tikz.
%
%The latest updates can be retrieved from
%  http://www.mathworks.com/matlabcentral/fileexchange/22022-matlab2tikz-matlab2tikz
%where you can also make suggestions and rate matlab2tikz.
%
\definecolor{mycolor1}{rgb}{0.00000,0.44700,0.74100}%
%
\begin{tikzpicture}

\begin{axis}[%
width=4.in,
height=4.in,
at={(0.983in,0.681in)},
scale only axis,
xmin=-1,
xmax=1,
ymin=-1,
ymax=1,
axis background/.style={fill=white},
axis x line*=bottom,
axis y line*=left,
legend style={legend cell align=left, align=left, draw=white!15!black}
]
\addplot [color=red, draw=none, mark=+, mark options={solid, red}]
  table[row sep=crcr]{%
-0.777777777777778	-0.777777777777778\\
-0.555555555555556	-0.777777777777778\\
-0.333333333333333	-0.777777777777778\\
-0.111111111111111	-0.777777777777778\\
0.111111111111111	-0.777777777777778\\
0.333333333333333	-0.777777777777778\\
0.555555555555556	-0.777777777777778\\
0.777777777777778	-0.777777777777778\\
-0.777777777777778	-0.555555555555556\\
-0.555555555555556	-0.555555555555556\\
-0.333333333333333	-0.555555555555556\\
-0.111111111111111	-0.555555555555556\\
0.111111111111111	-0.555555555555556\\
0.333333333333333	-0.555555555555556\\
0.555555555555556	-0.555555555555556\\
0.777777777777778	-0.555555555555556\\
-0.777777777777778	-0.333333333333333\\
-0.555555555555556	-0.333333333333333\\
-0.333333333333333	-0.333333333333333\\
-0.111111111111111	-0.333333333333333\\
0.111111111111111	-0.333333333333333\\
0.333333333333333	-0.333333333333333\\
0.555555555555556	-0.333333333333333\\
0.777777777777778	-0.333333333333333\\
-0.777777777777778	-0.111111111111111\\
-0.555555555555556	-0.111111111111111\\
-0.333333333333333	-0.111111111111111\\
-0.111111111111111	-0.111111111111111\\
0.111111111111111	-0.111111111111111\\
0.333333333333333	-0.111111111111111\\
0.555555555555556	-0.111111111111111\\
0.777777777777778	-0.111111111111111\\
-0.777777777777778	0.111111111111111\\
-0.555555555555556	0.111111111111111\\
-0.333333333333333	0.111111111111111\\
-0.111111111111111	0.111111111111111\\
0.111111111111111	0.111111111111111\\
0.333333333333333	0.111111111111111\\
0.555555555555556	0.111111111111111\\
0.777777777777778	0.111111111111111\\
-0.777777777777778	0.333333333333333\\
-0.555555555555556	0.333333333333333\\
-0.333333333333333	0.333333333333333\\
-0.111111111111111	0.333333333333333\\
0.111111111111111	0.333333333333333\\
0.333333333333333	0.333333333333333\\
0.555555555555556	0.333333333333333\\
0.777777777777778	0.333333333333333\\
-0.777777777777778	0.555555555555556\\
-0.555555555555556	0.555555555555556\\
-0.333333333333333	0.555555555555556\\
-0.111111111111111	0.555555555555556\\
0.111111111111111	0.555555555555556\\
0.333333333333333	0.555555555555556\\
0.555555555555556	0.555555555555556\\
0.777777777777778	0.555555555555556\\
-0.777777777777778	0.777777777777778\\
-0.555555555555556	0.777777777777778\\
-0.333333333333333	0.777777777777778\\
-0.111111111111111	0.777777777777778\\
0.111111111111111	0.777777777777778\\
0.333333333333333	0.777777777777778\\
0.555555555555556	0.777777777777778\\
0.777777777777778	0.777777777777778\\
};
\addlegendentry{Kollokationspunkte}

\addplot [color=blue, draw=none, mark=asterisk, mark options={solid, blue}]
  table[row sep=crcr]{%
-0.5	-0.5\\
0	-0.5\\
0.5	-0.5\\
-0.5	0\\
0	0\\
0.5	0\\
-0.5	0.5\\
0	0.5\\
0.5	0.5\\
};
\addlegendentry{Testpunkte}

\addplot [color=mycolor1, forget plot]
  table[row sep=crcr]{%
-1	-1\\
-1	1\\
1	1\\
1	-1\\
-1	-1\\
};
\end{axis}
\end{tikzpicture}%
}
\caption{Kollokationspunkte}
\label{fig:Kollok}
\end{figure}
Als Maß des Fehlers unserer numerischen Lösung wollen wir zunächst den maximalen absoluten Fehler zur analytischen Lösung berechnen, also
\begin{align*}
\text{error} = \max_{x \in \Omega} |u(x) - s_u (x)|,
\end{align*}
wobei $s_u$ die numerische Lösung bezeichnet.

Damit können wir uns anschauen, wie sich der Fehler bei Veränderung der Anzahl der Kollokationspunkte verhält.

In den folgenden vier Abbildungen \ref{fig:standard error abs n-sym}, \ref{fig:standard error abs sym}, \ref{fig:weighted error abs n-sym}, \ref{fig:weighted error abs sym} ist der Fehler der vier verschiedenen Verfahren dargestellt.
\begin{figure}[H]
\centering
\resizebox {.8\columnwidth} {!} {
% This file was created by matlab2tikz.
%
%The latest updates can be retrieved from
%  http://www.mathworks.com/matlabcentral/fileexchange/22022-matlab2tikz-matlab2tikz
%where you can also make suggestions and rate matlab2tikz.
%
\definecolor{mycolor1}{rgb}{0.00000,0.44700,0.74100}%
%
\begin{tikzpicture}

\begin{axis}[%
width=4.521in,
height=3.566in,
at={(0.758in,0.481in)},
scale only axis,
xmin=0,
xmax=350,
xlabel style={font=\color{white!15!black}},
xlabel={Anzahl der Kollokationspunkte},
ymode=log,
ymin=1e-06,
ymax=0.1,
yminorticks=true,
ylabel style={font=\color{white!15!black}},
ylabel={error},
axis background/.style={fill=white},
title style={font=\bfseries},
%title={},
legend style={legend cell align=left, align=left, draw=white!15!black}
]
\addplot [color=mycolor1]
  table[row sep=crcr]{%
4	0.0320644506260388\\
9	0.0306339761398731\\
16	0.00355243728244914\\
25	0.0017737871617638\\
36	0.000178468537807508\\
49	9.79655766377152e-05\\
64	3.72607179916912e-05\\
81	1.98171067870567e-05\\
100	3.87155205178874e-06\\
121	5.71971388607651e-06\\
144	1.20896339392967e-06\\
169	1.05487836936369e-06\\
196	2.43333894801856e-06\\
225	1.15948445597591e-06\\
256	3.56903262012029e-06\\
289	1.50488678509267e-06\\
324	2.19234618849956e-06\\
};
\addlegendentry{data1}

\end{axis}
\end{tikzpicture}%
}
\caption{Standardkollokation Nicht-Symmetrisch}
\label{fig:standard error abs n-sym}
\end{figure}
\begin{figure}[H]
\centering
\resizebox {.8\columnwidth} {!} {
% This file was created by matlab2tikz.
%
%The latest updates can be retrieved from
%  http://www.mathworks.com/matlabcentral/fileexchange/22022-matlab2tikz-matlab2tikz
%where you can also make suggestions and rate matlab2tikz.
%
\definecolor{mycolor1}{rgb}{0.00000,0.44700,0.74100}%
%
\begin{tikzpicture}

\begin{axis}[%
width=4.521in,
height=3.566in,
at={(0.758in,0.481in)},
scale only axis,
xmin=0,
xmax=6000,
xlabel style={font=\color{white!15!black}},
xlabel={Anzahl der Kollokationspunkte},
ymode=log,
ymin=1e-08,
ymax=1,
yminorticks=true,
ylabel style={font=\color{white!15!black}},
ylabel={Maximaler absoluter Fehler},
axis background/.style={fill=white},
%title style={font=\bfseries},
%title={error plot},
legend style={legend cell align=left, align=left, draw=white!15!black}
]
\addplot [color=mycolor1]
  table[row sep=crcr]{%
9	0.999748271191593\\
16	0.107726668816776\\
25	0.0258124696186204\\
36	0.00287159725441316\\
49	0.00362599497445987\\
64	0.000275935163883759\\
81	0.00021918327793681\\
100	2.48216260381184e-05\\
121	1.53681446218856e-05\\
144	3.1884711230723e-06\\
169	2.07364346599404e-06\\
196	2.17226477314258e-06\\
225	3.82458048009404e-07\\
256	3.59086256618291e-07\\
289	9.11238147272009e-07\\
324	8.53447753842995e-07\\
361	1.29068967375662e-06\\
400	3.13765663895182e-07\\
441	1.10766025934739e-07\\
484	4.31307005194226e-07\\
529	9.47970708806145e-08\\
576	2.12821862091706e-07\\
625	7.90592467714291e-07\\
676	1.91874217403409e-07\\
729	7.67803314552506e-07\\
784	1.40958309802208e-07\\
841	1.60128464252868e-06\\
900	6.95889448287801e-08\\
961	4.01987755416222e-07\\
1024	1.08639425788759e-07\\
1089	1.38214763745204e-07\\
1156	1.72387506935934e-07\\
1225	1.03612470492287e-07\\
1296	4.576342060858e-08\\
1369	1.98436127196722e-07\\
1444	6.35113151403743e-07\\
1521	5.06464954974639e-08\\
1600	1.52946966314182e-07\\
1681	4.09538019885414e-07\\
1764	5.5073665183869e-08\\
1849	1.66687766992024e-07\\
1936	1.06468657223857e-07\\
2025	5.50868686666206e-07\\
2116 6.99989677332979e-08\\
2209	1.01204194941085e-07\\
2304	5.08060809312205e-07\\
2401	2.75578998398807e-07\\
2500	1.35024995649713e-07\\
2601	4.28100931787467e-07\\
2704	5.52255797536816e-07\\
2809	3.7940928332425e-07\\
2916	9.61568953350422e-08\\
3025	1.89729150917861e-07\\
3136	1.21232752281486e-07\\
3249	2.27521430473665e-07\\
3364	9.72793279263584e-08\\
3481	2.35286509470134e-07\\
3600 1.68344388165598e-07\\
3721	1.9990815880444e-07\\
3844 1.23043994684074e-07\\
3969 6.10004760037697e-08\\
4096	8.13971327007224e-08\\
4225	6.43811171041619e-08\\
4356	8.52963331077206e-08\\
4489	4.8430353877249e-07\\
4624	9.93239147595304e-08\\
4761	1.89865310695758e-07\\
4900	1.6717591381013e-07\\
5041	6.36370970363842e-08\\
5184	7.78480268026627e-08\\
5329	7.94766711331718e-08\\
5476	1.1111985875889e-07\\
5625	2.43472415561996e-08\\
5776	7.04574397714097e-08\\
5929	5.41579672358461e-08\\
6084	1.31851039864017e-07\\
6241	5.68597048888897e-08\\
%6400	8.04885743610484e-08\\
};
%\addlegendentry{data1}

\end{axis}
\end{tikzpicture}%
}
\caption{Standardkollokation Symmetrisch}
\label{fig:standard error abs sym}
\end{figure}
\begin{figure}[H]
\centering
\resizebox {.8\columnwidth} {!} {
% This file was created by matlab2tikz.
%
%The latest updates can be retrieved from
%  http://www.mathworks.com/matlabcentral/fileexchange/22022-matlab2tikz-matlab2tikz
%where you can also make suggestions and rate matlab2tikz.
%
\definecolor{mycolor1}{rgb}{0.00000,0.44700,0.74100}%
%
\begin{tikzpicture}

\begin{axis}[%
width=4.521in,
height=3.566in,
at={(0.758in,0.481in)},
scale only axis,
xmin=0,
xmax=7000,
xlabel style={font=\color{white!15!black}},
xlabel={Anzahl der Kollokationspunkte},
ymode=log,
ymin=0.0001,
ymax=0.193128640997656,
yminorticks=true,
ylabel style={font=\color{white!15!black}},
ylabel={error},
axis background/.style={fill=white},
title style={font=\bfseries},
%title={error plot},
legend style={legend cell align=left, align=left, draw=white!15!black}
]
\addplot [color=mycolor1]
  table[row sep=crcr]{%
4	0.193128640997656\\
9	0.15429679379079\\
16	0.0460246958638726\\
25	0.024694615782175\\
36	0.0116949267612725\\
49	0.0184723650297368\\
64	0.00770494559446124\\
81	0.0104689630595075\\
100	0.00499062042985399\\
121	0.00680676632427402\\
144	0.00441697660141732\\
169	0.00654900714194975\\
196	0.00713433015692551\\
225	0.00945377019006362\\
256	0.00654279211806434\\
289	0.00934061416420038\\
324	0.00526506022360749\\
361	0.00547512455131141\\
400	0.00471635220966081\\
441	0.0052646417558769\\
484	0.0064283434045869\\
529	0.00351755836284226\\
576	0.00318572688981794\\
625	0.00403782594092178\\
676	0.00235093474471624\\
729	0.00239851854988768\\
784	0.00214868444943638\\
841	0.00231141172606663\\
900	0.00187068367159898\\
961	0.00194908698272984\\
1024	0.00187034403725673\\
1089	0.00196613577438514\\
1156	0.00197757612474177\\
1225	0.00162168131795074\\
1296	0.00194955851515578\\
1369	0.00110672807043146\\
1444	0.00121259506195763\\
1521	0.00112274907748577\\
1600	0.0012696257620108\\
1681	0.00137868742885077\\
1764	0.00111074476850058\\
1849	0.00137953380990466\\
1936	0.000865284123169485\\
2025	0.000867278698563693\\
2116	0.020303348108163\\
2209	0.00129708911054426\\
2304	0.000754430130699894\\
2401	0.000867203879304771\\
2500	0.00063335379648965\\
2601	0.000638937607292727\\
2704	0.00362228553536667\\
2809	0.000667173227936095\\
2916	0.00121605777456142\\
3025	0.00065387456677554\\
3136	0.000689886271300929\\
3249	0.00157462688673092\\
3364	0.000467246718358549\\
3481	0.00050288660343495\\
3600	0.000522592209090961\\
3721	0.0478898221125074\\
3844	0.000547093177643932\\
3969	0.000502654248024297\\
4096	0.000640314395829842\\
4225	0.000423147807886189\\
4356	0.000613751598949008\\
4489	0.00266914109814746\\
4624	0.00037973442375187\\
4761	0.000336686943717282\\
4900	0.000333247510963787\\
5041	0.000622394932566736\\
5184	0.000314427678619867\\
5329	0.000566880506456827\\
5476	0.000305234483597394\\
5625	0.00144050943677986\\
5776	0.000944749817815887\\
5929	0.000349251699575004\\
6084	0.00471299033041693\\
};
\addlegendentry{data1}

\end{axis}
\end{tikzpicture}%
}
\caption{Gewichtete Kollokation Nicht-Symmetrisch}
\label{fig:weighted error abs n-sym}
\end{figure}
\begin{figure}[H]
\centering
\resizebox {.8\columnwidth} {!} {
% This file was created by matlab2tikz.
%
%The latest updates can be retrieved from
%  http://www.mathworks.com/matlabcentral/fileexchange/22022-matlab2tikz-matlab2tikz
%where you can also make suggestions and rate matlab2tikz.
%
\definecolor{mycolor1}{rgb}{0.00000,0.44700,0.74100}%
%
\begin{tikzpicture}

\begin{axis}[%
width=4.521in,
height=3.566in,
at={(0.758in,0.481in)},
scale only axis,
xmin=0,
xmax=2400,
xlabel style={font=\color{white!15!black}},
xlabel={Anzahl der Kollokationspunkte},
ymode=log,
ymin=0.000283657173492734,
ymax=1,
yminorticks=true,
ylabel style={font=\color{white!15!black}},
ylabel={Maximaler absoluter Fehler},
axis background/.style={fill=white},
%title style={font=\bfseries},
%title={error plot},
legend style={legend cell align=left, align=left, draw=white!15!black}
]
\addplot [color=mycolor1]
  table[row sep=crcr]{%
1	1\\
4	0.395035224833815\\
9	0.279941526477746\\
16	0.00757742098063802\\
25	0.15279986779413\\
36	0.0181867597368573\\
49	0.0174726143157827\\
64	0.0059766539555961\\
81	0.0072259913307462\\
100	0.00640403437702161\\
121	0.00442626411214436\\
144	0.00532310723804046\\
169	0.00387310797335394\\
196	0.00407996506526073\\
225	0.00368820159673228\\
256	0.00301329641393755\\
289	0.00417435519798021\\
324	0.00255717984883069\\
361	0.00228090820312338\\
400	0.00351398681638995\\
441	0.0026029916068597\\
484	0.00179911655500439\\
529	0.0018987095444248\\
576	0.00263945817673591\\
625	0.00199283880609186\\
676	0.00143566232698034\\
729	0.00144940556062869\\
784	0.0023538107588279\\
841	0.00181963638460381\\
900	0.00131781663419642\\
961	0.00106571464305072\\
1024	0.00104064262918289\\
1089	0.00114222240624102\\
1156	0.00145230809792884\\
1225	0.00114740561792823\\
1296	0.000866938209172496\\
1369	0.000726834193691045\\
1444	0.000675689848845577\\
1521	0.000472648830725779\\
1600	0.00101085556294689\\
1681	0.000823352891871659\\
1764	0.000653814864999019\\
1849	0.0005307186107327\\
1936	0.000418057941285993\\
2025	0.000338538094624863\\
2116	0.00028389945236007\\
2209	0.000283657173492734\\
2304	0.000662089880460209\\
};
%\addlegendentry{data1}

\end{axis}
\end{tikzpicture}%
}
\caption{Gewichtete Kollokation Symmetrisch}
\label{fig:weighted error abs sym}
\end{figure}

Wir stellen als erstes fest, dass alle Verfahren vernünftige Ergebnisse liefern und konvergieren. Unsere theoretische Herleitung war demnach also sinnvoll.

Zum besseren Vergleich der Verfahren stellen wir in Tabelle \ref{tab:Vergleich Fehler} die Fehler nochmal in einer Tabelle dar.

\begin{table}[H]
\centering
\begin{tabular}{c|c|c|c|c}
%\hline 
 & $n = 9$ & $n = 100$ & $n = 1024$ & $n = 6084$ \\ 
\hline 
Standard N-Sym & \num{0.99974} & \num{2.18205e-04} & \num{2.61592e-06} & \num{9.07630e-06} \\ 
%\hline 
Standard Sym & \num{0.99974} & \num{2.48216e-05} & \num{1.08639e-07} & \num{1.31851e-07} \\ 
%\hline 
Gewichtet N-Sym & \num{0.37643} & \num{8.20819e-03} & \num{1.87889e-03} & \num{3.09914e-04} \\ 
%\hline 
Gewichtet Sym & \num{0.27994} & \num{6.40403e-03} & \num{1.04064e-03} & • \\ 
%\hline 
\end{tabular} 
\caption{Vergleich der Verfahren}
\label{tab:Vergleich Fehler}
\end{table}

Wir erkennen, dass alle Verfahren sehr schnell gute Ergebnisse liefern. Die beiden Standardverfahren liefern bereits mit nahezu $100$ Kollokationspunkten ihre besten Ergebnisse und verbessern sich danach nur noch wenig. Im Gegensatz dazu stehen die gewichteten Verfahren, die zwar auch schon am Anfang gute Ergebnisse liefern, sich dann aber auch mit mehr Kollokationspunkten weiter verbessern, wenn auch ziemlich langsam. Gesamt erreichen die gewichteten Verfahren aber selbst mit $6000$ Kollokationspunkten nicht die Genauigkeit der Standardverfahren. Im Vergleich der symmetrischen und nicht-symmetrischen Verfahren schneiden beide Male die symmetrischen Verfahren leicht besser ab.

Wir möchten nun überprüfen, ob die gewichtete Kollokation ihren Sinn erfüllt, d.h. ob die Lösung auf dem Rand auch wirklich Null ist. Dafür plotten wir eine mit dem Standardverfahren erstellte Lösung (Abbildung \ref{fig:rand standard}) und eine mit dem gewichteten Verfahren erstellte Lösung (Abbildung \ref{fig:rand weighted}) über einen Teil des Randes.

\begin{figure}[H]
\centering
\resizebox {.75\columnwidth} {!} {
% This file was created by matlab2tikz.
%
%The latest updates can be retrieved from
%  http://www.mathworks.com/matlabcentral/fileexchange/22022-matlab2tikz-matlab2tikz
%where you can also make suggestions and rate matlab2tikz.
%
\definecolor{mycolor1}{rgb}{0.00000,0.44700,0.74100}%
%
\begin{tikzpicture}

\begin{axis}[%
width=4.521in,
height=3.566in,
at={(0.758in,0.481in)},
scale only axis,
xmin=-1,
xmax=1,
ymin=-2.5e-06,
ymax=2e-06,
axis background/.style={fill=white},
legend style={legend cell align=left, align=left, draw=white!15!black}
]
\addplot [color=mycolor1]
  table[row sep=crcr]{%
-1	-5.58793544769287e-07\\
-0.97979797979798	-2.90572643280029e-07\\
-0.95959595959596	7.45058059692383e-08\\
-0.939393939393939	1.9371509552002e-07\\
-0.919191919191919	-2.98023223876953e-08\\
-0.898989898989899	2.01165676116943e-07\\
-0.878787878787879	5.58793544769287e-07\\
-0.858585858585859	5.51342964172363e-07\\
-0.838383838383838	5.88595867156982e-07\\
-0.818181818181818	4.09781932830811e-07\\
-0.797979797979798	4.54485416412354e-07\\
-0.777777777777778	6.63101673126221e-07\\
-0.757575757575758	7.67409801483154e-07\\
-0.737373737373737	3.05473804473877e-07\\
-0.717171717171717	7.15255737304688e-07\\
-0.696969696969697	4.76837158203125e-07\\
-0.676767676767677	8.5681676864624e-07\\
-0.656565656565657	6.85453414916992e-07\\
-0.636363636363636	1.06543302536011e-06\\
-0.616161616161616	6.25848770141602e-07\\
-0.595959595959596	7.30156898498535e-07\\
-0.575757575757576	7.22706317901611e-07\\
-0.555555555555556	1.10268592834473e-06\\
-0.535353535353535	1.08033418655396e-06\\
-0.515151515151515	1.37090682983398e-06\\
-0.494949494949495	1.41561031341553e-06\\
-0.474747474747475	1.33365392684937e-06\\
-0.454545454545455	1.46776437759399e-06\\
-0.434343434343434	1.31130218505859e-06\\
-0.414141414141414	1.61677598953247e-06\\
-0.393939393939394	1.49756669998169e-06\\
-0.373737373737374	1.84029340744019e-06\\
-0.353535353535353	1.57207250595093e-06\\
-0.333333333333333	1.564621925354e-06\\
-0.313131313131313	1.11758708953857e-06\\
-0.292929292929293	1.35600566864014e-06\\
-0.272727272727273	1.13248825073242e-06\\
-0.252525252525252	1.02072954177856e-06\\
-0.232323232323232	6.18398189544678e-07\\
-0.212121212121212	8.64267349243164e-07\\
-0.191919191919192	8.27014446258545e-07\\
-0.171717171717172	2.23517417907715e-08\\
-0.151515151515151	4.61935997009277e-07\\
-0.131313131313131	2.23517417907715e-08\\
-0.111111111111111	3.05473804473877e-07\\
-0.0909090909090909	2.83122062683105e-07\\
-0.0707070707070707	-3.65078449249268e-07\\
-0.0505050505050505	-1.78813934326172e-07\\
-0.0303030303030303	-4.4703483581543e-08\\
-0.0101010101010101	-3.20374965667725e-07\\
0.0101010101010102	-2.08616256713867e-07\\
0.0303030303030303	1.78813934326172e-07\\
0.0505050505050506	-3.27825546264648e-07\\
0.0707070707070707	1.34110450744629e-07\\
0.0909090909090908	-2.23517417907715e-08\\
0.111111111111111	3.39001417160034e-07\\
0.131313131313131	6.07222318649292e-07\\
0.151515151515152	2.83122062683105e-07\\
0.171717171717172	5.69969415664673e-07\\
0.191919191919192	8.67992639541626e-07\\
0.212121212121212	9.16421413421631e-07\\
0.232323232323232	6.59376382827759e-07\\
0.252525252525253	1.11758708953857e-06\\
0.272727272727273	8.77305865287781e-07\\
0.292929292929293	6.6123902797699e-07\\
0.313131313131313	1.0412186384201e-06\\
0.333333333333333	1.48266553878784e-06\\
0.353535353535354	6.14207237958908e-07\\
0.373737373737374	9.53208655118942e-07\\
0.393939393939394	4.51225787401199e-07\\
0.414141414141414	6.00237399339676e-07\\
0.434343434343434	2.46800482273102e-07\\
0.454545454545455	4.94532287120819e-07\\
0.474747474747475	1.49011611938477e-08\\
0.494949494949495	3.87430191040039e-07\\
0.515151515151515	3.83704900741577e-07\\
0.535353535353535	4.24683094024658e-07\\
0.555555555555556	3.16649675369263e-07\\
0.575757575757576	4.58210706710815e-07\\
0.595959595959596	2.75671482086182e-07\\
0.616161616161616	6.29574060440063e-07\\
0.636363636363636	6.78002834320068e-07\\
0.656565656565657	8.45640897750854e-07\\
0.676767676767677	1.00582838058472e-06\\
0.696969696969697	1.00955367088318e-06\\
0.717171717171717	1.02072954177856e-06\\
0.737373737373737	1.39325857162476e-06\\
0.757575757575758	1.16229057312012e-06\\
0.777777777777778	1.39325857162476e-06\\
0.797979797979798	1.36345624923706e-06\\
0.818181818181818	1.30385160446167e-06\\
0.838383838383838	1.03563070297241e-06\\
0.858585858585859	1.08778476715088e-06\\
0.878787878787879	7.45058059692383e-07\\
0.898989898989899	2.23517417907715e-08\\
0.919191919191919	-6.70552253723145e-08\\
0.939393939393939	-7.07805156707764e-07\\
0.95959595959596	-1.26659870147705e-06\\
0.97979797979798	-1.51246786117554e-06\\
1	-2.25752592086792e-06\\
};
%\addlegendentry{data1}

\end{axis}
\end{tikzpicture}%
}
\caption{Rand bei Standardkollokation}
\label{fig:rand standard}
\end{figure}
\begin{figure}[H]
\centering
\resizebox {.75\columnwidth} {!} {
% This file was created by matlab2tikz.
%
%The latest updates can be retrieved from
%  http://www.mathworks.com/matlabcentral/fileexchange/22022-matlab2tikz-matlab2tikz
%where you can also make suggestions and rate matlab2tikz.
%
\definecolor{mycolor1}{rgb}{0.00000,0.44700,0.74100}%
%
\begin{tikzpicture}

\begin{axis}[%
width=4.521in,
height=3.566in,
at={(0.758in,0.481in)},
scale only axis,
xmin=-1,
xmax=1,
ymin=-4e-16,
ymax=4e-16,
axis background/.style={fill=white},
legend style={legend cell align=left, align=left, draw=white!15!black}
]
\addplot [color=mycolor1]
  table[row sep=crcr]{%
-1	0\\
-0.97979797979798	6.36054028626163e-17\\
-0.95959595959596	1.09459696635666e-16\\
-0.939393939393939	-1.96007741793587e-16\\
-0.919191919191919	1.4191549568074e-16\\
-0.898989898989899	-3.71767520242687e-16\\
-0.878787878787879	0\\
-0.858585858585859	0\\
-0.838383838383838	-3.13322411811114e-16\\
-0.818181818181818	0\\
-0.797979797979798	0\\
-0.777777777777778	2.13524025908097e-16\\
-0.757575757575758	0\\
-0.737373737373737	0\\
-0.717171717171717	0\\
-0.696969696969697	0\\
-0.676767676767677	0\\
-0.656565656565657	0\\
-0.636363636363636	3.07819254967142e-16\\
-0.616161616161616	0\\
-0.595959595959596	0\\
-0.575757575757576	0\\
-0.555555555555556	0\\
-0.535353535353535	0\\
-0.515151515151515	0\\
-0.494949494949495	0\\
-0.474747474747475	0\\
-0.454545454545455	0\\
-0.434343434343434	0\\
-0.414141414141414	0\\
-0.393939393939394	3.20741719714877e-16\\
-0.373737373737374	0\\
-0.353535353535353	0\\
-0.333333333333333	0\\
-0.313131313131313	0\\
-0.292929292929293	0\\
-0.272727272727273	0\\
-0.252525252525252	0\\
-0.232323232323232	0\\
-0.212121212121212	0\\
-0.191919191919192	0\\
-0.171717171717172	0\\
-0.151515151515151	0\\
-0.131313131313131	0\\
-0.111111111111111	0\\
-0.0909090909090909	0\\
-0.0707070707070707	0\\
-0.0505050505050505	0\\
-0.0303030303030303	0\\
-0.0101010101010101	0\\
0.0101010101010102	0\\
0.0303030303030303	0\\
0.0505050505050506	0\\
0.0707070707070707	0\\
0.0909090909090908	0\\
0.111111111111111	0\\
0.131313131313131	0\\
0.151515151515152	0\\
0.171717171717172	0\\
0.191919191919192	0\\
0.212121212121212	0\\
0.232323232323232	0\\
0.252525252525253	0\\
0.272727272727273	0\\
0.292929292929293	0\\
0.313131313131313	0\\
0.333333333333333	0\\
0.353535353535354	0\\
0.373737373737374	0\\
0.393939393939394	0\\
0.414141414141414	0\\
0.434343434343434	0\\
0.454545454545455	0\\
0.474747474747475	0\\
0.494949494949495	0\\
0.515151515151515	0\\
0.535353535353535	0\\
0.555555555555556	0\\
0.575757575757576	0\\
0.595959595959596	0\\
0.616161616161616	0\\
0.636363636363636	0\\
0.656565656565657	0\\
0.676767676767677	0\\
0.696969696969697	0\\
0.717171717171717	0\\
0.737373737373737	0\\
0.757575757575758	0\\
0.777777777777778	0\\
0.797979797979798	0\\
0.818181818181818	0\\
0.838383838383838	-1.56661204759499e-16\\
0.858585858585859	0\\
0.878787878787879	0\\
0.898989898989899	3.71767481297e-16\\
0.919191919191919	2.83830927614092e-16\\
0.939393939393939	1.96007644623201e-16\\
0.95959595959596	-1.09459556870653e-16\\
0.97979797979798	-5.08839393738675e-17\\
1	0\\
};
%\addlegendentry{data1}

\end{axis}
\end{tikzpicture}%
}
\caption{Rand bei gewichteter Kollokation}
\label{fig:rand weighted}
\end{figure}

Bei der Standardkollokation sind deutlich Schwankungen über den Rand erkennbar, wohingegen der Rand bei der gewichteten Kollokation bis auf kleine numerische Ungenauigkeiten tatsächlich Null ist. Das Ziel der gewichteten Kollokation wurde also erreicht.
%\section{Kondition}
%\section{Laufzeit}