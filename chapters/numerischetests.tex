\chapter{Numerische Tests}
\label{cha:NumerischeTests}

In diesem Kapitel möchten wir die vorgestellten Verfahren verschiedenen numerischen Tests unterziehen. Dabei wollen wir hauptsächlich den Fehler der numerischen Lösung zur analytischen untersuchen, aber auch die Kondition der Kollokationsmatrizen, die Parameterwahl und die Laufzeit betrachten. Zusätzlich werden wir die Verfahren mit der finiten Elemente Methode (\acs{FEM}) vergleichen.

Dafür betrachten wir die Poisson-Gleichung mit Dirichlet-Randbedingung auf $\Omega = [-1,1] \times [-1,1] \subset \mathbb{R}^n$:
\begin{align*}
- \Delta u &= 2\pi^2 \sin(\pi x)\sin(\pi y)&&, (x,y) \in \Omega\\
u &= 0&&, (x,y) \in \partial \Omega
\end{align*}
mit analytischer Lösung $u(x,y) = \sin(\pi x)\sin(\pi y)$.

Diese werden wir numerisch lösen.

\section{Fehler}
Wir werden die Kollokationspunkte, so wie die Testpunkte, zunächst, wie in Abbildung \ref{fig:Kollok} gezeigt, in einem Gitter anordnen.
\begin{figure}[h]
\centering
\resizebox {.8\columnwidth} {!} {
% This file was created by matlab2tikz.
%
%The latest updates can be retrieved from
%  http://www.mathworks.com/matlabcentral/fileexchange/22022-matlab2tikz-matlab2tikz
%where you can also make suggestions and rate matlab2tikz.
%
\definecolor{mycolor1}{rgb}{0.00000,0.44700,0.74100}%
%
\begin{tikzpicture}

\begin{axis}[%
width=4.in,
height=4.in,
at={(0.983in,0.681in)},
scale only axis,
xmin=-1,
xmax=1,
ymin=-1,
ymax=1,
axis background/.style={fill=white},
axis x line*=bottom,
axis y line*=left,
legend style={legend cell align=left, align=left, draw=white!15!black}
]
\addplot [color=red, draw=none, mark=+, mark options={solid, red}]
  table[row sep=crcr]{%
-0.777777777777778	-0.777777777777778\\
-0.555555555555556	-0.777777777777778\\
-0.333333333333333	-0.777777777777778\\
-0.111111111111111	-0.777777777777778\\
0.111111111111111	-0.777777777777778\\
0.333333333333333	-0.777777777777778\\
0.555555555555556	-0.777777777777778\\
0.777777777777778	-0.777777777777778\\
-0.777777777777778	-0.555555555555556\\
-0.555555555555556	-0.555555555555556\\
-0.333333333333333	-0.555555555555556\\
-0.111111111111111	-0.555555555555556\\
0.111111111111111	-0.555555555555556\\
0.333333333333333	-0.555555555555556\\
0.555555555555556	-0.555555555555556\\
0.777777777777778	-0.555555555555556\\
-0.777777777777778	-0.333333333333333\\
-0.555555555555556	-0.333333333333333\\
-0.333333333333333	-0.333333333333333\\
-0.111111111111111	-0.333333333333333\\
0.111111111111111	-0.333333333333333\\
0.333333333333333	-0.333333333333333\\
0.555555555555556	-0.333333333333333\\
0.777777777777778	-0.333333333333333\\
-0.777777777777778	-0.111111111111111\\
-0.555555555555556	-0.111111111111111\\
-0.333333333333333	-0.111111111111111\\
-0.111111111111111	-0.111111111111111\\
0.111111111111111	-0.111111111111111\\
0.333333333333333	-0.111111111111111\\
0.555555555555556	-0.111111111111111\\
0.777777777777778	-0.111111111111111\\
-0.777777777777778	0.111111111111111\\
-0.555555555555556	0.111111111111111\\
-0.333333333333333	0.111111111111111\\
-0.111111111111111	0.111111111111111\\
0.111111111111111	0.111111111111111\\
0.333333333333333	0.111111111111111\\
0.555555555555556	0.111111111111111\\
0.777777777777778	0.111111111111111\\
-0.777777777777778	0.333333333333333\\
-0.555555555555556	0.333333333333333\\
-0.333333333333333	0.333333333333333\\
-0.111111111111111	0.333333333333333\\
0.111111111111111	0.333333333333333\\
0.333333333333333	0.333333333333333\\
0.555555555555556	0.333333333333333\\
0.777777777777778	0.333333333333333\\
-0.777777777777778	0.555555555555556\\
-0.555555555555556	0.555555555555556\\
-0.333333333333333	0.555555555555556\\
-0.111111111111111	0.555555555555556\\
0.111111111111111	0.555555555555556\\
0.333333333333333	0.555555555555556\\
0.555555555555556	0.555555555555556\\
0.777777777777778	0.555555555555556\\
-0.777777777777778	0.777777777777778\\
-0.555555555555556	0.777777777777778\\
-0.333333333333333	0.777777777777778\\
-0.111111111111111	0.777777777777778\\
0.111111111111111	0.777777777777778\\
0.333333333333333	0.777777777777778\\
0.555555555555556	0.777777777777778\\
0.777777777777778	0.777777777777778\\
};
\addlegendentry{Kollokationspunkte}

\addplot [color=blue, draw=none, mark=asterisk, mark options={solid, blue}]
  table[row sep=crcr]{%
-0.5	-0.5\\
0	-0.5\\
0.5	-0.5\\
-0.5	0\\
0	0\\
0.5	0\\
-0.5	0.5\\
0	0.5\\
0.5	0.5\\
};
\addlegendentry{Testpunkte}

\addplot [color=mycolor1, forget plot]
  table[row sep=crcr]{%
-1	-1\\
-1	1\\
1	1\\
1	-1\\
-1	-1\\
};
\end{axis}
\end{tikzpicture}%
}
\caption{Kollokationspunkte}
\label{fig:Kollok}
\end{figure}
Als Maß des Fehlers unserer numerischen Lösung wollen wir zunächst den maximalen absoluten Fehler zur analytischen Lösung berechnen, also
\begin{align*}
error = \max_{x \in \Omega} |u(x) - s_u (x)|,
\end{align*}
wobei $s_u$ die numerische Lösung bezeichnet.

Damit können wir uns anschauen, wie sich der Fehler bei Veränderung der Anzahl der Kollokationspunkte verhält.
\begin{figure}[h]
\centering
\resizebox {\columnwidth} {!} {
% This file was created by matlab2tikz.
%
%The latest updates can be retrieved from
%  http://www.mathworks.com/matlabcentral/fileexchange/22022-matlab2tikz-matlab2tikz
%where you can also make suggestions and rate matlab2tikz.
%
\definecolor{mycolor1}{rgb}{0.00000,0.44700,0.74100}%
%
\begin{tikzpicture}

\begin{axis}[%
width=4.521in,
height=3.566in,
at={(0.758in,0.481in)},
scale only axis,
xmin=0,
xmax=350,
xlabel style={font=\color{white!15!black}},
xlabel={Anzahl der Kollokationspunkte},
ymode=log,
ymin=1e-06,
ymax=0.1,
yminorticks=true,
ylabel style={font=\color{white!15!black}},
ylabel={error},
axis background/.style={fill=white},
title style={font=\bfseries},
%title={},
legend style={legend cell align=left, align=left, draw=white!15!black}
]
\addplot [color=mycolor1]
  table[row sep=crcr]{%
4	0.0320644506260388\\
9	0.0306339761398731\\
16	0.00355243728244914\\
25	0.0017737871617638\\
36	0.000178468537807508\\
49	9.79655766377152e-05\\
64	3.72607179916912e-05\\
81	1.98171067870567e-05\\
100	3.87155205178874e-06\\
121	5.71971388607651e-06\\
144	1.20896339392967e-06\\
169	1.05487836936369e-06\\
196	2.43333894801856e-06\\
225	1.15948445597591e-06\\
256	3.56903262012029e-06\\
289	1.50488678509267e-06\\
324	2.19234618849956e-06\\
};
\addlegendentry{data1}

\end{axis}
\end{tikzpicture}%
}
\caption{Standardkollokation Nicht-Symmetrisch}
\label{fig:standard error abs n-sym}
\end{figure}

\begin{figure}[h]
\centering
\resizebox {\columnwidth} {!} {
% This file was created by matlab2tikz.
%
%The latest updates can be retrieved from
%  http://www.mathworks.com/matlabcentral/fileexchange/22022-matlab2tikz-matlab2tikz
%where you can also make suggestions and rate matlab2tikz.
%
\definecolor{mycolor1}{rgb}{0.00000,0.44700,0.74100}%
%
\begin{tikzpicture}

\begin{axis}[%
width=4.521in,
height=3.566in,
at={(0.758in,0.481in)},
scale only axis,
xmin=0,
xmax=7000,
xlabel style={font=\color{white!15!black}},
xlabel={Anzahl der Kollokationspunkte},
ymode=log,
ymin=0.0001,
ymax=0.193128640997656,
yminorticks=true,
ylabel style={font=\color{white!15!black}},
ylabel={error},
axis background/.style={fill=white},
title style={font=\bfseries},
%title={error plot},
legend style={legend cell align=left, align=left, draw=white!15!black}
]
\addplot [color=mycolor1]
  table[row sep=crcr]{%
4	0.193128640997656\\
9	0.15429679379079\\
16	0.0460246958638726\\
25	0.024694615782175\\
36	0.0116949267612725\\
49	0.0184723650297368\\
64	0.00770494559446124\\
81	0.0104689630595075\\
100	0.00499062042985399\\
121	0.00680676632427402\\
144	0.00441697660141732\\
169	0.00654900714194975\\
196	0.00713433015692551\\
225	0.00945377019006362\\
256	0.00654279211806434\\
289	0.00934061416420038\\
324	0.00526506022360749\\
361	0.00547512455131141\\
400	0.00471635220966081\\
441	0.0052646417558769\\
484	0.0064283434045869\\
529	0.00351755836284226\\
576	0.00318572688981794\\
625	0.00403782594092178\\
676	0.00235093474471624\\
729	0.00239851854988768\\
784	0.00214868444943638\\
841	0.00231141172606663\\
900	0.00187068367159898\\
961	0.00194908698272984\\
1024	0.00187034403725673\\
1089	0.00196613577438514\\
1156	0.00197757612474177\\
1225	0.00162168131795074\\
1296	0.00194955851515578\\
1369	0.00110672807043146\\
1444	0.00121259506195763\\
1521	0.00112274907748577\\
1600	0.0012696257620108\\
1681	0.00137868742885077\\
1764	0.00111074476850058\\
1849	0.00137953380990466\\
1936	0.000865284123169485\\
2025	0.000867278698563693\\
2116	0.020303348108163\\
2209	0.00129708911054426\\
2304	0.000754430130699894\\
2401	0.000867203879304771\\
2500	0.00063335379648965\\
2601	0.000638937607292727\\
2704	0.00362228553536667\\
2809	0.000667173227936095\\
2916	0.00121605777456142\\
3025	0.00065387456677554\\
3136	0.000689886271300929\\
3249	0.00157462688673092\\
3364	0.000467246718358549\\
3481	0.00050288660343495\\
3600	0.000522592209090961\\
3721	0.0478898221125074\\
3844	0.000547093177643932\\
3969	0.000502654248024297\\
4096	0.000640314395829842\\
4225	0.000423147807886189\\
4356	0.000613751598949008\\
4489	0.00266914109814746\\
4624	0.00037973442375187\\
4761	0.000336686943717282\\
4900	0.000333247510963787\\
5041	0.000622394932566736\\
5184	0.000314427678619867\\
5329	0.000566880506456827\\
5476	0.000305234483597394\\
5625	0.00144050943677986\\
5776	0.000944749817815887\\
5929	0.000349251699575004\\
6084	0.00471299033041693\\
};
\addlegendentry{data1}

\end{axis}
\end{tikzpicture}%
}
\caption{Gewichtete Kollokation Nicht-Symmetrisch}
\label{fig:weighted error abs n-sym}
\end{figure}




\section{Kondition}
\section{Laufzeit}