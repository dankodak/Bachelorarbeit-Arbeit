\chapter{Standardkollokation}
\label{cha:Standardkollokation}

\section{Symmetrische Kollokation}
\label{sec:SymKol}
Sei wieder $\Omega \subset \mathbb{R}^d$ offen und beschränkt, $L$ und $B$ lineare Differentialoperatoren der Ordnung $k$ und $l$, $K \in C^{2\max\{l,k\}}(\Omega \times \Omega)$  ein \gls{SPD} Kern und eine \gls{PDE} wie in \eqref{eq:PDE} gegeben. Für ein $N \in \mathbb{N}$ betrachten wir die Menge $X_N = \left\{ x_i \right\}_{i=1}^N \subset \widebar{\Omega}$, die wir in $N_{in} > 0$ Punkte im Inneren und $N_{bd} > 0$ Punkte auf dem Rand aufteilen. Also haben wir die beiden Mengen
\begin{align*}
X_{in} &= X_N \cap \Omega\\
X_{bd} &= X_N \cap \partial \Omega
\end{align*}
Wir definieren die Menge $\Lambda_N = \{\lambda_1, \dots, \lambda_N\}$ an linearen Funktionalen mit
\begin{align*}
\lambda_i =
\begin{cases}
\delta_{x_i} \circ L & x_i \in X_{in}\\
\delta_{x_i} \circ B & x_i \in X_{bd}
\end{cases}
\end{align*}
Wir wissen aus Satz \ref{stetig}, dass in $\mathcal{H}_K(\Omega)$ alle $\lambda_i$ stetig und aus Satz \ref{linUn}, dass sie linear unabhängig sind. Als Ansatzfunktionen, also den Unterraum $V_N \subset \mathcal{H}_K(\Omega)$, wählen wir die Riesz Repräsentanten der $\lambda_i$:
\begin{align*}
V_N &= \text{span} \{\lambda_1^y K(x,y), \dots , \lambda_N^y K(x,y)\}\\
&= \text{span} \{\{(\delta_{x_1} \circ L)^y K(x,y), \dots, (\delta_{x_{N_{in}}} \circ L)^y K(x,y)\}\\
&\cup \{ (\delta_{x_{N_{in} + 1}} \circ B)^y K(x,y), \dots, (\delta_{x_{N}} \circ B)^y K(x,y)\}\}\\
&=: \text{span} \{\nu_1, \dots, \nu_N\}
\end{align*}

Damit bekommen wir folgenden Interpolanten:
\begin{align*}
s_u(x) &= \sum_{j=1}^N \alpha_j \lambda_j^y K(x,y)\\
&= \sum_{j=1}^{N_{in}} \alpha_j (\delta_{x_j} \circ L)^y K(x,y) + \sum_{j=N_{in}+1}^{N} \alpha_j (\delta_{x_j} \circ B)^y K(x,y)
\end{align*}
Die $\alpha_j$ erhält man als Lösung des \ac{LGS} $A \alpha = b$ mit $A_{i,j} := (\nu_i,\nu_j)_{\mathcal{H}_K}$, da
\begin{align*}
\langle \lambda_i, s_u\rangle = \langle \lambda_i, \sum_{j=1}^N \alpha_j \nu_j \rangle = \sum_{j=1}^N \alpha_j \langle \lambda_i,\nu_j\rangle \overset{\ref{Riesz}}{=} \sum_{j=1}^N \alpha_j \left(\nu_j, \nu_i\right),
\end{align*}
also
\begin{align*}
\begin{pmatrix}
A_{L,L} & A_{L,B} \\ 
A_{L,B}^T & A_{B,B}
\end{pmatrix} 
\alpha =
\begin{pmatrix}
b_L \\ 
b_B
\end{pmatrix} 
\end{align*}
mit
\begin{align*}
(A_{L,L})_{i,j} &= (\delta_{x_i} \circ L)^x(\delta_{x_j} \circ L)^y K(x,y),x_i, x_j \in X_{in}\\
(A_{L,B})_{i,j} &= (\delta_{x_i} \circ L)^x(\delta_{x_j} \circ B)^y K(x,y),x_i \in X_{in}, x_j \in X_{bd} \\
(A_{B,B})_{i,j} &= (\delta_{x_i} \circ B)^x(\delta_{x_j} \circ B)^y K(x,y), x_i, x_j \in X_{bd}
\end{align*}
und
\begin{align*}
(b_L)_i &= f(x_i), x_i \in X_{in}\\
(b_B)_i &= g(x_i), x_i \in X_{bd}
\end{align*}
Das \ac{LGS} ist lösbar, da A offensichtlich symmetrisch und positiv definit ist, da:
\begin{align*}
\alpha^T A \alpha = \sum_{i,j = 1}^N \alpha_i \alpha_j (\nu_i, \nu_j)_{\mathcal{H}_K} = \left(\sum_{i=1}^N \alpha_i \nu_i, \sum_{j=1}^N \alpha_j \nu_j \right)_{\mathcal{H}_K} = \left\| \sum_{i=1}^N \alpha_i \nu_i \right\|_{\mathcal{H}_K}^2 > 0
\end{align*}
Für die letzte Abschätzung benutzen wir die lineare Unabhängigkeit der Funktionale aus Satz \ref{linUn}.
\section{Nicht-Symmetrische Kollokation}
\label{sec:NSymKol}
Sei die gleiche Problemstellung wie im vorherigen Kapitel gegeben. Wir wählen jedoch einen anderen Unterraum $V_N$ für die Ansatzfunktionen:
\begin{align*}
V_N := \text{span} \{K(x,x_1), \dots, K(x,x_N) \}
\end{align*}
Damit bekommen wir folgenden Interpolanten:
\begin{align*}
s_u(x) = \sum_{j=1}^N \alpha_j K(x,x_j)
\end{align*}
Die $\alpha_i$ erhält man wieder als Lösung eines \ac{LGS} $A \alpha = b$ mit
\begin{align*}
A := \begin{pmatrix}
A_L \\ 
A_B
\end{pmatrix} 
\end{align*}
mit
\begin{align*}
(A_L)_{i,j} &= (\delta_{x_i} \circ L)^x K(x,x_j), x_i \in X_{in}, x_j \in X_N\\
(A_B)_{i,j} &= (\delta_{x_i} \circ B)^x K(x,x_j), x_i \in X_{bd}, x_j \in X_N
\end{align*}
und
$b$ wie im vorherigen Abschnitt.

Bei diesem Ansatz kann jedoch nicht garantiert werden, dass die Matrix $A$ invertierbar ist und damit, dass das \ac{LGS} auch lösbar ist. Allerdings ist die Konstruktion eines Gegenbeispiels aufwendig und es ist in der Praxis unwahrscheinlich, dass man auf einen solchen Fall trifft. Man findet ein Beispiel in \textcite{Hon.2001}.
