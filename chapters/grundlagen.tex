\chapter{Kerne und reproduzierende Kern Hilberträume}
\label{cha:Grundlagen}

Die Kernkollokation ist ein Verfahren, welches auf der Idee der Interpolation beruht. Da wir bei \acp{PDE} auch Ableitungen betrachten müssen, reicht die einfache Punktauswertung für unsere Problemstellung nicht mehr aus, weswegen wir eine verallgemeinerte Form der Interpolation benötigen.

\begin{definition}
Sei $\Omega \subset \mathbb{R}^d$ eine nicht leere Menge, $\mathcal{H}$ ein Hilbertraum bestehend aus Funktionen $f:\Omega \rightarrow \mathbb{R}$, $\mathcal{H}'$ der dazugehörige Dualraum, $u \in \mathcal{H}$  und $\Lambda_N := \{\lambda_1, \dots, \lambda_N\} \subset \mathcal{H}'$ eine Menge von linearen, stetigen und linear unabhängigen Funktionalen. Dann ist eine Funktion $s_u \in \mathcal{H}$ ein verallgemeinerter Interpolant von $u$, wenn gilt, dass
\begin{align*}
\langle \lambda_i,u \rangle = \langle \lambda_i,s_u \rangle , 1\le i \le N,
\end{align*}
wobei wir $\langle \lambda_i, u \rangle := \lambda_i(u)$ für die Anwendung des Funktionals schreiben.
\end{definition}

\begin{example}
$\mbox{}$
\begin{itemize}
\item
Sei $\Omega \subset \mathbb{R}^d$, $X_N := \{x_1, \dots, x_N\} \subset \Omega$ eine Menge von Punkten, $\mathcal{H}$ ein Hilbertraum bestehend aus Funktionen $f:\Omega \rightarrow \mathbb{R}$, in dem die Punktauswertungfunktionale $\delta_{x_i}(f) := f(x_i), 1\le i \le N$  stetig sind und $u \in \mathcal{H}$. Dann bekommen wir die Standardinterpolation mit $\Lambda_N := \{\delta_{x_1}, \dots,\delta_{x_N}\} \subset \mathcal{H}'$:
\begin{align*}
s(x_i) = \langle \delta_{x_i},u \rangle = \langle \delta_{x_i},s_u \rangle = s_u(x_i), 1\le i \le N
\end{align*}
\item
Für einen Multiindex $a \in \mathbb{N}_0^d$ sei $D^a$ ein linearer partieller Differentialoperator. Dann erhält man mit $\lambda_i := \delta_{x_i} \circ D^a$ die Auswertungsfunktionale der Ableitung, mit welchen wir interpolieren können.
\item
Sei eine \ac{PDE} mit Lösung $u \in C^k(\Omega) \cap C^0(\widebar \Omega)$ gegeben:
\begin{align}\label{eq:PDE}
\begin{split}
L u(x) &= f(x), x \in \Omega \\
B u(x) &= g(x), x \in \partial \Omega,
\end{split}
\end{align}
wobei $L$ ein linearer Differentialoperator und $B$ ein linearer Randwertoperator ist.
Sei $X_N \subset \Omega$ eine Menge an Punkten, genant Kollokationspunkte. Dann möchten wir, dass $s_u$ die \ac{PDE} in den Punkten $X_N$ erfüllt, also:
\begin{align*}
L s_u(x_i) &= L u(x_i) = f(x_i), x_i \in \Omega\\
B s_u(x_i) &= B u(x_i) = g(x_i), x_i \in \partial \Omega
\end{align*}
\end{itemize}
\end{example}

Wir müssen einen geeigneten diskreten Ansatz wählen um das Interpolationsproblem numerisch zu lösen. Hierfür wählen wir einen $N$-dimensionalen Unterraum $V_N := \text{span}\{\nu_1, \dots, \nu_N\} \subset \mathcal{H}$ und fordern, dass $s_u \in V_N$, also 
\begin{align*}
s_u(x) := \sum_{j=1}^N \alpha_j \nu_j(x), x \in \Omega, \alpha \in \mathbb{R}^N.
\end{align*}
Damit können wir die Interpolationsbedingungen schreiben als:
\begin{align*}
\langle \lambda_i,u \rangle = \langle \lambda_i,s_u \rangle = \sum_{j=1}^N \alpha_j \langle \lambda_i,\nu_j \rangle
\end{align*}
Diese lassen sich auch als lineares Gleichungssystem $A_\Lambda \alpha = b$ schreiben  mit $(A_\Lambda)_{i,j} := \langle \lambda_i,\nu_j \rangle, b_i := \langle \lambda_i,u \rangle$.

Wir suchen jetzt nach geeigneten Ansatzfunktionen und einem Hilbertraum, in dem die Auswertungs- und Differentialfunktionale stetig sind. Dies führt uns zur Definition von Kern Funktionen, mit denen wir einen Hilbertraum konstruieren werden, der das Geforderte liefern wird.

\begin{definition}
\label{Kern}
Sei $\Omega$ eine beliebige nicht leere Menge. Ein reeller Kern auf $\Omega$ ist eine symmetrische Funktion $K: \Omega \times \Omega \rightarrow \mathbb{R}$.

Für alle $N \in \mathbb{N}$ und für eine Menge $X_N = \{x_i\}_{i=1}^N \subset \Omega$ ist die Kernmatrix (oder Gram'sche Matrix) $A:= A_{K,X_N} \in \mathbb{R}^{N \times N}$  definiert als $A:=[K(x_i, x_j)]_{i,j=1}^N$.

Ein Kern $K$ heißt \ac{PD} auf $\Omega$, wenn für alle $N \in \mathbb{N}$ und alle Mengen $X_N$ mit paarweise verschiedenen Elementen $\left\{x_i\right\}_{i=1}^N \subset \Omega$ gilt, dass die Kernmatrix positiv semidefinit ist. Der Kern $K$ heißt \ac{SPD}, falls die Kernmatrix für alle solche $N$ und $X_N$ positiv definit ist.
\end{definition}

\begin{theorem}
\label{thm:Kombi}
Sei $\Omega$ eine beliebige nicht leere Menge, $K_1, K_2:\Omega \rightarrow \mathbb{R}$ zwei \ac{PD} Kerne auf $\Omega$ und $a \geq 0$. Dann sind folgende Funktionen wieder \ac{PD} Kerne auf $\Omega$:
\begin{enumerate}
\item
$K(x,y) := K_1(x,y) + K_2(x,y)$
\item
$K(x,y) := aK_1 (x,y)$
\item
$K(x,y) := K_1(x,y)K_2(x,y)$
\end{enumerate}
\end{theorem}
\begin{proof}
Die Symmetrie ist in allen Fällen offensichtlich. Wir betrachten daher nur die positive Definitheit.

Sei $X_N \subset \Omega$ eine Menge mit paarweise verschiedenen Punkten $\left\{x_i\right\}_{i=1}^N$.
\begin{enumerate}
\item
Für die Kernmatrix von $K$ gilt:
\begin{align*}
A_K &= 
\begin{pmatrix}
K(x_1, x_1) & \cdots & K(x_1, x_N) \\ 
\vdots & \ddots & \vdots \\ 
K(x_N, x_1) & \cdots & K(x_N, x_N)
\end{pmatrix} \\
&=
\begin{pmatrix}
K_1(x_1, x_1) & \cdots & K_1(x_1, x_N) \\ 
\vdots & \ddots & \vdots \\ 
K_1(x_N, x_1) & \cdots & K_1(x_N, x_N)
\end{pmatrix} 
+
\begin{pmatrix}
K_2(x_1, x_1) & \cdots & K_2(x_1, x_N) \\ 
\vdots & \ddots & \vdots \\ 
K_2(x_N, x_1) & \cdots & K_2(x_N, x_N)
\end{pmatrix} \\
&= A_{K_1} + A_{K_2}
\end{align*}
Wir erhalten also für ein beliebiges $\alpha \neq 0$
\begin{align*}
\alpha^T A_K \alpha &= \alpha^T \left( A_{K_1} + A_{K_2} \right) \alpha \\
&=\underbrace{\alpha^T A_{K_1} \alpha}_{\geq 0} + \underbrace{\alpha^T A_{K_2} \alpha}_{\geq 0} \geq 0
\end{align*}
\item
Für ein beliebiges $\alpha \neq 0$ gilt
\begin{align*}
\alpha^T A_K \alpha = \alpha^T a A_{K_1} \alpha = a \alpha^T A_{K_1} \alpha \geq 0
\end{align*}
\item
Wir betrachten wieder die Kernmatrix.
\begin{align*}
A_K &= 
\begin{pmatrix}
K_1(x_1,x_1)K_2(x_1, x_1) & \cdots & K_1(x_1,x_N)K_2(x_1, x_N) \\ 
\vdots & \ddots & \vdots \\ 
K_1(x_N,x_1)K_2(x_N, x_1) & \cdots & K_1(x_N,x_N)K_2(x_N, x_N)
\end{pmatrix} \\
&= 
\begin{pmatrix}
K_1(x_1,x_1) & \cdots & K_1(x_1,x_N) \\ 
\vdots & \ddots & \vdots \\ 
K_1(x_N,x_1) & \cdots & K_1(x_N,x_N)
\end{pmatrix}
\circ
\begin{pmatrix}
K_2(x_1, x_1) & \cdots & K_2(x_1, x_N) \\ 
\vdots & \ddots & \vdots \\ 
K_2(x_N, x_1) & \cdots & K_2(x_N, x_N)
\end{pmatrix},
\end{align*}
wobei $\circ$ das Hadamard-Produkt der beiden Matrizen bezeichnet. 
Die beiden letzten Matrizen sind positiv semidefinit und damit nach dem Satz von Schur \cite{.30.07.2018} auch das Produkt der beiden.
\end{enumerate}
\end{proof}

\begin{example}
\label{ex:Kern}
Sei $\Omega \subset \mathbb{R}^n$ und $\gamma \in \mathbb{R}^+$. Dann sind folgende Funktionen \ac{PD} Kerne auf $\Omega$:
\begin{enumerate}
\item Skalarprodukt: $K(x,y) := \gamma^{-1} (x,y)$
\item Gauß Kern: $K(x,y) := \exp\left(-\gamma \|x-y\|^2\right)$ ist sogar \ac{SPD}
\end{enumerate}
\end{example}

\begin{proof}
$\mbox{}$
\begin{enumerate}
\item
Die Symmetrie folgt aus der Symmetrie des Skalarprodukts. Die Kernmatrix entspricht der Gram Matrix des Skalarprodukts. Diese ist aufgrund der positiven Definitheit des Skalarprodukt positiv definit. 
\item
Einen Beweis dafür findet man in \textcite[Theorem 6.10]{Wendland.2005}.
\end{enumerate}
\end{proof}

Die Definition der Kerneführt uns direkt zu den gesuchten Hilberträumen. Diese sind zunächst ohne Bezug zu Kernen definiert, wir werden aber feststellen, dass sie eng miteinander verknüpft sind.

\begin{definition}[Reproduzierender Kern Hilbertraum]
Sei $\Omega$ eine beliebige nicht leere Menge und $\mathcal{H}$ ein Hilbertraum bestehend aus Funktionen $f:\Omega \rightarrow \mathbb{R}$ und Skalarprodukt $(\cdot, \cdot)_\mathcal{H}$. Dann nennt man $\mathcal{H}$ einen reproduzierenden Kern Hilbertraum (\acs{RKHR})\glsunset{RKHR} auf $\Omega$, wenn eine Funktion $K:\Omega \times \Omega \rightarrow \mathbb{R}$ existiert, sodass
\begin{enumerate}
\item $K(\cdot, x) \in \mathcal{H}$ für alle $x \in \Omega$
\item $(f, K(\cdot,x))_\mathcal{H} = f(x)$ für alle $ x \in \Omega$, $f \in \mathcal{H}$ (Reproduzierbarkeit)
\end{enumerate}
Man nennt $K$ den reproduzierenden Kern von $\mathcal{H}$.
\end{definition}

Dass $K$ tatsächlich ein Kern nach Definition \ref{Kern} ist, zeigt folgender Satz.
\begin{theorem}
\label{thm:EindeutigkeitKern}
Sei $\mathcal{H}$ ein \gls{RKHR} mit reproduzierendem Kern $K$. Dann ist $K$ ein Kern, eindeutig und \ac{PD}.
\end{theorem}

\begin{proof}
Wir folgen dem Beweis in \textcite[Theorem 3.6]{Santin.2017}.

Wir zeigen zunächst, dass $K$ tatsächlich ein Kern ist.
\begin{align*}
K(x,y) &= \left( K(\cdot,y), K(x, \cdot)\right)_\mathcal{H} &&\text{(Reproduzierbarkeit)}\\
&= \left(K(x,\cdot), K(\cdot,y)\right)_\mathcal{H}\\
&= K(y,x) &&\text{(Reproduzierbarkeit)}
\end{align*}

Sei $X_N := \left\{ x_i \right\}_{i=1}^N \subset \Omega$ eine Menge von $N$ paarweise verschiedenen Punkten und $\alpha \in \mathbb{R}^N, \alpha \neq 0$. Dann gilt:
\begin{align*}
\alpha^T A \alpha &= \sum_{i,j=1}^N \alpha_i \alpha_j K(x_i, x_j)\\
&= \sum_{i,j=1}^N \alpha_i \alpha_j \left(K(\cdot, x_i), K(\cdot,x_j)\right)_\mathcal{H}\\
&= \left(\sum_{i=1}^N \alpha_i K(\cdot,x_i), \sum_{j=1}^N \alpha_j K(\cdot, x_j)\right)_\mathcal{H}\\
&= \left\| \sum_{i=1}^N \alpha_i K(\cdot,x_i)\right\|_\mathcal{H}^2 \geq 0
\end{align*}
$K$ ist somit \ac{PD}.

Seien jetzt $K_1, K_2$ zwei Kerne auf $\mathcal{H}$. Dann gilt für alle $x,y \in \Omega$:
\begin{align*}
K_1(x,y) &= (K_1(\cdot,y), K_2(x, \cdot))_\mathcal{H} &&\text{(Reproduzierbarkeit von }K_1\text{)}\\
&= K_2(x,y) &&\text{(Reproduzierbarkeit von }K_2\text{)}
\end{align*}
Also ist $K$ eindeutig.
\end{proof}

Bei Interpolationsproblemen kommen wir jedoch aus der anderen Richtung und haben Ansatzfunktionen, also einen Kern $K$, gegeben und wollen damit eine Funktion approximieren. Demnach stellt sich die Frage ob zu jedem Kern $K$ ein \ac{RKHR} existiert. Diese wird durch folgenden Satz beantwortet:

\begin{theorem}
Sei $\Omega$ eine beliebige nicht leere Menge und $K:\Omega \times \Omega \rightarrow \mathbb{R}$ ein positiv definiter Kern. Dann existiert genau ein \ac{RKHR} $\mathcal{H}_K (\Omega)$ mit reproduzierendem Kern $K$.
\end{theorem}
\begin{proof}
Einen Beweis findet man in \textcite[Kap. 10.2]{Wendland.2005}. 

Man betrachtet dort zunächst den $\text{span} \{K(\cdot,y), y \in \Omega \}$ und stellt fest, dass dieser mit einem geeigneten Innenprodukt ein Prähilbertraum ist. Der Abschluss dessen ist der gesuchte \ac{RKHR}.
\end{proof}

Wir wollen an einen Satz aus der Funktionalanalysis erinnern, den wir noch oft benötigen werden.

\begin{theorem}[Fréchet-Riesz]
Sei $\mathcal{H}$ ein Hilbertraum und $\lambda \in \mathcal{H}'$ ein beschränktes lineares Funktional. Dann existiert ein eindeutig bestimmtes Element $\nu_\lambda \in \mathcal{H}$, so dass für alle $x \in \mathcal{H}$ gilt:
\begin{align*}
\langle \lambda, x \rangle = \left( x, \nu_\lambda \right)_\mathcal{H}
\end{align*}
Wir nennen $\nu_\lambda$ den Riesz-Repräsentanten von $\lambda$.
\end{theorem}

Zur Wohldefiniertheit unserer Interpolation benötigen wir die Stetigkeit aller verwendeten Funktionale. Zunächst betrachten wir die Punktauswertungsfunktionale.

\begin{theorem}
\label{stetig}
Sei $\Omega$ eine beliebige nicht leere Menge und $\mathcal{H}$ ein Hilbertraum bestehend aus Funktionen $f: \Omega \rightarrow \mathbb{R}$. Dann gilt:
\begin{enumerate}
\item \label{stetig1} $\mathcal{H}$ ist genau dann ein \ac{RKHR}, wenn die Auswertungsfunktionale stetig sind.
\item \label{stetig2} Wenn $\mathcal{H}$ ein \ac{RKHR} mit reproduzierendem Kern $K$ ist, dann ist $K(\cdot,x)$ der Riesz-Repräsentant des Funktionals $\delta_x \in \mathcal{H}'$.
\end{enumerate}
\end{theorem}

\begin{proof}
Wir folgen dem Beweis in \textcite[Proposition 3.8]{Santin.2017}.
\begin{enumerate}
\item 
Sei $\mathcal{H}$ ein \ac{RKHR}. Für alle $f \in \mathcal{H}$ und alle $x \in \Omega$ gilt:
\begin{align*}
|\langle \delta_x,f \rangle | &= |f(x)|\\ 
&= |(f, K(\cdot,x))_\mathcal{H}| &&\text{(Reproduzierbarkeit)}\\
&\le \|f\|_\mathcal{H}\|K(\cdot,x)\|_\mathcal{H} &&\text{(Cauchy Schwarz)}\\
&= \|f\|_\mathcal{H} \sqrt{(K(\cdot,x),K(\cdot,x))_\mathcal{H}}\\
&= \|f\|_\mathcal{H} \sqrt{K(x,x)} &&\text{(Reproduzierbarkeit)}\\
\Leftrightarrow \frac{|\langle \delta_x,f \rangle|}{\|f\|_\mathcal{H}} &\le \sqrt{K(x,x)}
\end{align*}

Also ist $\delta_x$ beschränkt und damit stetig.

Für die andere Richtung nehmen wir an, dass $\delta_x  \in \mathcal{H}'$ für alle $x \in \Omega$. Also existiert ein Riesz-Repräsentant $\nu_{\delta_x} \in \mathcal{H}$. Definieren wir $K(\cdot,x):= \nu_{\delta_x}$, dann ist $K$ ein reproduzierender Kern. Es ist klar, dass $K(\cdot,x) \in \mathcal{H}$ und nach der Definition des Riesz-Repräsentanten gilt:
\begin{align*}
(f, K(\cdot,x))_\mathcal{H} = (f, \nu_{\delta_x})_\mathcal{H} = \langle \delta_x,f \rangle = f(x)
\end{align*}
\item Die Behauptung folgt sofort aus der Reproduzierbarkeit von $K$, da $(f, K(\cdot,x))_\mathcal{H}= f(x)$ für alle $x \in \Omega$ und alle $f \in \mathcal{H}$ gilt.
\end{enumerate}
\end{proof}

Wir haben gesehen, dass in einem \ac{RKHR} $\mathcal{H}_K$ die Auswertungsfunktionale stetig sind. Da wir uns mit Differentialgleichungen beschäftigen, wollen wir auch Ableitungen auswerten. Dafür ist es nötig, dass diese ebenfalls in $\mathcal{H}_K$ liegen.

\begin{theorem}
Sei $k \in \mathbb{N}$. Angenommen $\Omega \subset \mathbb{R}^d$ ist offen, K ist \ac{SPD} auf $\Omega$ und $K \in C^{2k}(\Omega \times \Omega)$. Dann gilt für alle Multiindizes $a \in \mathbb{N}_0^d$ mit $|a| \le k$ und alle $x \in \Omega$, dass $D_2^a K(\cdot , x) \in \mathcal{H}_K(\Omega)$, wobei der tiefgestellte Index bedeutet, dass der Operator auf das zweite Argument angewandt wird.

Außerdem gilt für alle $f \in \mathcal{H}_K(\Omega)$:
\begin{align*}
D^a f(x) = \left(f,D_2^a K(\cdot,x)\right)_{\mathcal{H}_K(\Omega)}
\end{align*}
und damit dass $\lambda := \delta_x \circ D^a$ stetig ist.
\end{theorem}

\begin{proof}
Einen Beweis des ersten Teils findet man im Vorlesungsskript von \textcite[Proposition 7.13]{Santin.2017} und einen Beweis des zweiten Teils in \textcite[Proposition 3.14]{Santin.2017}. Der Beweis der Stetigkeit von $\lambda := \delta_x \circ D^a$ verläuft komplett analog zum Beweis von \ref{stetig}.\ref{stetig1}.
\end{proof}

In Satz \ref{stetig} haben wir gesehen, wie der Riesz-Repräsentant des Auswertungsfunktionals aussieht. Dies wollen wir jetzt auf alle Funktionale verallgemeinern.

\begin{theorem}
\label{Riesz}
Sei $K$ ein \ac{PD} Kern auf $\Omega \neq \emptyset$ und $\lambda \in \mathcal{H}_K (\Omega)'$. Dann ist $\lambda^y K(\cdot,y) \in \mathcal{H}_k(\Omega)$ und es gilt $\langle \lambda,f \rangle = \left(f,\lambda^y K(\cdot,y)\right)_{\mathcal{H}_K(\Omega)}$ für alle $f \in \mathcal{H}_K(\Omega)$, wobei der hochgestellte Index bedeutet, dass das Funktional auf die zweite Komponente angewandt wird. Es ist also $\lambda^y K(\cdot,y)$ der Riesz-Repräsentant von $\lambda$.
\end{theorem}

\begin{proof}
Wir folgen dem Beweis in \textcite[Proposition 7.8]{Santin.2017}.

Da $\lambda \in \mathcal{H}_K(\Omega)$, existiert ein Riesz-Repräsentant $\nu_\lambda \in \mathcal{H}_K(\Omega)$ mit $\langle \lambda ,f \rangle = \left(f, \nu_\lambda\right)_{\mathcal{H}_K(\Omega)}$. Außerdem ist $f_x(y) := K(x,y)$ für alle $x \in \Omega$ eine Funktion in $\mathcal{H}_K (\Omega)$. Damit bekommen wir:
\begin{align*}
\langle \lambda, K(x,\cdot) \rangle &= \langle \lambda,f_x \rangle\\ 
&= \left(f_x, \nu_\lambda\right)_{\mathcal{H}_K (\Omega)} &&\text{(Riesz-Repräsentant)}\\ 
&= \left(K(\cdot,x), \nu_\lambda\right)_{\mathcal{H}_K (\Omega)}\\ 
&= \nu_\lambda(x) &&\text{(Reproduzierbarkeit)}
\end{align*}
Damit gilt $\nu_\lambda(\cdot) = \lambda^y K(\cdot,y)$ und damit ist $\lambda^yK(\cdot,y)$ der Riesz-Repräsentant von $\lambda$. Da $\nu_\lambda \in \mathcal{H}_K (\Omega)$ gilt auch $\lambda^y K(\cdot,y) \in \mathcal{H}_K (\Omega)$.
\end{proof}

Jetzt fehlt nur noch die lineare Unabhängigkeit aller verwendeten Funktionale. Zunächst betrachten wir die der Auswertungsfunktionale:
\begin{theorem}
Sei $\Omega$ eine beliebige nicht leere Menge und $\mathcal{H}$ ein \ac{RKHR} mit Kern $K$. Dann sind $\{\delta_x,x\in \Omega\}$ genau dann linear unabhängig, wenn $K$ \ac{SPD} ist.
\end{theorem}

\begin{proof}
Wir folgen dem Beweis in \textcite[Proposition 3.8]{Santin.2017}.

Seien $\lambda_1, \dots, \lambda_n \in \mathcal{H}'$ und $\nu_{\lambda_1},\dots, \nu_{\lambda_n} \in \mathcal{H}$ die dazugehörigen Riesz Repräsentanten. Diese sind linear abhängig, wenn ein $\alpha \in \mathbb{R}^n$ und $\alpha \neq 0$ existiert mit $\lambda := \sum_{i=1}^n \alpha_i \lambda_i = 0$, also dass $\langle \lambda,f \rangle = 0$ für alle $f \in \mathcal{H}$. Das gilt genau dann, wenn die Riesz Repräsentanten linear abhängig sind, da
\begin{align*}
0 = \langle \lambda,f \rangle = \sum_{i=1}^n \alpha_i \langle \lambda_i,f \rangle = \sum_{i=1}^n \alpha_i \left( \nu_{\lambda_i},f\right)_\mathcal{H} = \left( \sum_{i=1}^n \alpha_i \nu_{\lambda_i}, f \right)_\mathcal{H}
\end{align*}

Also gilt nach \ref{stetig}.\ref{stetig2}, dass $\{\delta_x,x\in \Omega\}$ genau dann linear unabhängig sind, wenn $\{K(\cdot,x) , x \in \Omega\}$ linear unabhängig sind.

Um die strikte positive Definitheit nachzuweisen, betrachten wir die Matrix $A=[K(x_i, x_j)]_{i,j=1}^N$ für paarweise verschiedene Punkte $x_i, 1 \le i \le N$. Sei also $\beta \in \mathbb{R}^n, \beta \neq 0$. Dann gilt:
\begin{align*}
\beta^T A \beta &= \sum_{i,j=1}^n \beta_i \beta_j K(x_i, x_j)\\
&= \sum_{i,j=1}^n \beta_i  \beta_j \left(K(\cdot, x_i),K(\cdot,x_j)\right)_\mathcal{H}\\
&= \left( \sum_{i=1}^n \beta_i K(\cdot,x_i),\sum_{j=1}^n \beta_j K(\cdot, x_j) \right)_\mathcal{H}\\
&= \left\| \sum_{i=1}^n \beta_i K(\cdot, x_i) \right\|_\mathcal{H}^2 > 0
\end{align*}
Für die strikte Ungleichung benötigen wir die lineare Unabhängigkeit. Also gilt, dass K \ac{SPD} ist, wenn $\{\delta_x,x\in \Omega\}$ linear unabhängig sind.

Die andere Richtung folgt genauso aus der strikten Ungleichung.
\end{proof}

Jetzt betrachten wir die lineare Unabhängigkeit der Auswertungen der Ableitungen:

\begin{theorem}
\label{linUn}
Sei $K$ ein translationsinvarianter Kern auf $\mathbb{R}^d$, also $K(x,y) = \Phi (x-y)$ für alle $x,y \in \mathbb{R}^d$. Sei $k \in \mathbb{N}$ und angenommen, dass $\Phi \in L_1(\mathbb{R}^d) \cap C^{2k}(\mathbb{R}^d)$. Sei $a_1, \dots, a_N \in \mathbb{N}_0^d$ mit $|a_i| \le k$ und sei $X_N \subset \mathbb{R}^d$. Angenommen, dass $a_i \neq a_j$, wenn $x_i = x_j$, dann sind die Funktionale $\Lambda_N := \{\lambda_1, \dots, \lambda_N\}, \lambda_i := \delta_{x_i} \circ D^{a_i}$ linear unabhängig in $\mathcal{H}_K(\mathbb{R}^d)'$.
\end{theorem}

\begin{proof}
Den Beweis findet man in \textcite[Theorem 16.4]{Wendland.2005}.
\end{proof}


Damit haben wir alle nötigen Werkzeuge, um die Interpolation durchzuführen zu können. Wir haben einen Kern $K$, den dazugehörigen Hilbertraum $\mathcal{H}_K(\Omega)$, sowie die Stetigkeit und lineare Unabhängigkeit aller benötigten Operatoren. Jetzt müssen wir nur noch einen geeigneten Ansatz wählen.