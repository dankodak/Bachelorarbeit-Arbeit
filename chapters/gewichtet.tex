\chapter{Gewichtete Kollokation}
\label{cha:Gewichtet}
\section{Motivation für die gewichtete Kollokation}
\label{sec:motivation}
Die Standardkollokation hat, egal ob symmetrisch oder nicht-symmetrisch, das Problem, dass wir Punkte im Inneren und auf dem Rand unseres Definitionsbereichs benötigen. Dies macht zum einen die Implementierung etwas komplexer, da man dabei beide Mengen beachten muss, zum anderen werden die Werte auf dem Rand nicht zwingend genau angenommen. In Abbildung \ref{fig:rand} ist die approximierte Lösung einer \ac{PDE} mit Nullrandwerten über den Rand geplottet. Man erkennt deutlich, wo die Stützstellen der Ansatzfunktionen liegen und auch die Schwankungen zwischen den Stützstellen.
\begin{figure}[ht]
\centering
\resizebox {\columnwidth} {!} {
% This file was created by matlab2tikz.
%
%The latest updates can be retrieved from
%  http://www.mathworks.com/matlabcentral/fileexchange/22022-matlab2tikz-matlab2tikz
%where you can also make suggestions and rate matlab2tikz.
%
\definecolor{mycolor1}{rgb}{0.00000,0.44700,0.74100}%
%
\begin{tikzpicture}

\begin{axis}[%
width=7.072in,
height=5.348in,
at={(1.186in,0.722in)},
scale only axis,
xmin=0,
xmax=6.28,
xlabel style={font=\color{white!15!black}},
xlabel={$\text{0 }\leq\text{ x }\leq\text{ 2}\pi$},
ymin=-0.0004,
ymax=0.0004,
axis background/.style={fill=white},
title style={font=\bfseries},
%title={Interpolant auf dem Rand},
legend style={legend cell align=left, align=left, draw=white!15!black}
]
\addplot [color=mycolor1]
  table[row sep=crcr]{%
0	-0.000393708935007453\\
0.00628947478196155	-0.000394012065953575\\
0.0125789495639231	-0.00039349839425995\\
0.0188684243458846	-0.000392166712117614\\
0.0251578991278462	-0.000390017979952972\\
0.0314473739098077	-0.000387055137252901\\
0.0377368486917693	-0.000383283342671348\\
0.0440263234737308	-0.000378709810320288\\
0.0503157982556924	-0.000373343711544294\\
0.0566052730376539	-0.000367196509614587\\
0.0628947478196155	-0.00036028142858413\\
0.069184222601577	-0.000352613940776791\\
0.0754736973835386	-0.000344211162882857\\
0.0817631721655001	-0.000335092398017878\\
0.0880526469474617	-0.000325278619129676\\
0.0943421217294232	-0.000314792534481967\\
0.100631596511385	-0.000303658871416701\\
0.106921071293346	-0.000291903648758307\\
0.113210546075308	-0.000279554711596575\\
0.119500020857269	-0.000266641320195049\\
0.125789495639231	-0.000253194153629011\\
0.132078970421193	-0.000239245124248555\\
0.138368445203154	-0.000224827683268813\\
0.144657919985116	-0.00020997631145292\\
0.150947394767077	-0.000194726271729451\\
0.157236869549039	-0.000179114154889248\\
0.163526344331	-0.000163177239301149\\
0.169815819112962	-0.000146953450894216\\
0.176105293894923	-0.000130481563246576\\
0.182394768676885	-0.00011380073192413\\
0.188684243458846	-9.69505636021495e-05\\
0.194973718240808	-7.99710833234712e-05\\
0.20126319302277	-6.29022579232696e-05\\
0.207552667804731	-4.57844107586425e-05\\
0.213842142586693	-2.86577451333869e-05\\
0.220131617368654	-1.15622060548048e-05\\
0.226421092150616	5.46227602171712e-06\\
0.232710566932577	2.23766037379391e-05\\
0.239000041714539	3.91414723708294e-05\\
0.2452895164965	5.57185849174857e-05\\
0.251578991278462	7.20702264516149e-05\\
0.257868466060423	8.81596679391805e-05\\
0.264157940842385	0.00010395065692137\\
0.270447415624347	0.000119408257887699\\
0.276736890406308	0.000134498444822384\\
0.28302636518827	0.000149188454088289\\
0.289315839970231	0.000163446577062132\\
0.295605314752193	0.000177242742211092\\
0.301894789534154	0.000190547987585887\\
0.308184264316116	0.000203334984689718\\
0.314473739098077	0.000215577936614864\\
0.320763213880039	0.000227252574404702\\
0.327052688662	0.000238336404436268\\
0.333342163443962	0.000248808530159295\\
0.339631638225924	0.000258650041359942\\
0.345921113007885	0.000267843421170255\\
0.352210587789847	0.000276373484666692\\
0.358500062571808	0.000284226509393193\\
0.36478953735377	0.000291390984784812\\
0.371079012135731	0.000297857099212706\\
0.377368486917693	0.000303616954624886\\
0.383657961699654	0.000308664813928772\\
0.389947436481616	0.000312996562570333\\
0.396236911263578	0.000316610166919418\\
0.402526386045539	0.000319505583320279\\
0.408815860827501	0.000321684521622956\\
0.415105335609462	0.000323150648910087\\
0.421394810391424	0.000323909414873924\\
0.427684285173385	0.000323968160955701\\
0.433973759955347	0.000323335927532753\\
0.440263234737308	0.000322023490298307\\
0.44655270951927	0.000320043323881691\\
0.452842184301231	0.000317409565468552\\
0.459131659083193	0.000314137963869143\\
0.465421133865155	0.000310245530272368\\
0.471710608647116	0.000305751047562808\\
0.478000083429078	0.000300674422760494\\
0.484289558211039	0.000295036999887088\\
0.490579032993001	0.000288861327135237\\
0.496868507774962	0.000282171018625377\\
0.503157982556924	0.000274990867183078\\
0.509447457338885	0.000267346385953715\\
0.515736932120847	0.000259264532360248\\
0.522026406902809	0.000250772467552451\\
0.52831588168477	0.000241898360400228\\
0.534605356466732	0.000232670863624662\\
0.540894831248693	0.000223119306610897\\
0.547184306030655	0.000213273564440897\\
0.553473780812616	0.00020316343579907\\
0.559763255594578	0.000192819294170476\\
0.566052730376539	0.000182271698577097\\
0.572342205158501	0.000171551266248571\\
0.578631679940462	0.000160688236064743\\
0.584921154722424	0.000149713174323551\\
0.591210629504386	0.000138656268973136\\
0.597500104286347	0.000127547315059928\\
0.603789579068309	0.000116416063974611\\
0.61007905385027	0.0001052915067703\\
0.616368528632232	9.42021906666923e-05\\
0.622658003414193	8.31760225992184e-05\\
0.628947478196155	7.2240403824253e-05\\
0.635236952978116	6.14216151006985e-05\\
0.641526427760078	5.07455260958523e-05\\
0.647815902542039	4.0236995118903e-05\\
0.654105377324001	2.99196362902876e-05\\
0.660394852105963	1.98164016183e-05\\
0.666684326887924	9.94890069705434e-06\\
0.672973801669886	3.38244717568159e-07\\
0.679263276451847	-8.99644146556966e-06\\
0.685552751233809	-1.80367132998072e-05\\
0.69184222601577	-2.67653012997471e-05\\
0.698131700797732	-3.51666349160951e-05\\
0.704421175579693	-4.32259403169155e-05\\
0.710710650361655	-5.09299716213718e-05\\
0.717000125143616	-5.82669163122773e-05\\
0.723289599925578	-6.52262424409855e-05\\
0.72957907470754	-7.1798396675149e-05\\
0.735868549489501	-7.79757319833152e-05\\
0.742158024271463	-8.37517181935254e-05\\
0.748447499053424	-8.9121163910022e-05\\
0.754736973835386	-9.40803074627183e-05\\
0.761026448617347	-9.86263585218694e-05\\
0.767315923399309	-0.000102758283901494\\
0.77360539818127	-0.000106476229120744\\
0.779894872963232	-0.000109781634819228\\
0.786184347745193	-0.000112676974822534\\
0.792473822527155	-0.000115165879833512\\
0.798763297309117	-0.000117253712232923\\
0.805052772091078	-0.000118946365546435\\
0.81134224687304	-0.000120251144835493\\
0.817631721655001	-0.000121176337415818\\
0.823921196436963	-0.000121731209219433\\
0.830210671218924	-0.000121925884741358\\
0.836500146000886	-0.000121771696285577\\
0.842789620782847	-0.000121280401799595\\
0.849079095564809	-0.000120464988867752\\
0.855368570346771	-0.000119338987133233\\
0.861658045128732	-0.000117916471936041\\
0.867947519910694	-0.000116212358989287\\
0.874236994692655	-0.000114242084237048\\
0.880526469474617	-0.000112021360109793\\
0.886815944256578	-0.00010956665937556\\
0.89310541903854	-0.000106894618511433\\
0.899394893820501	-0.000104022379673552\\
0.905684368602463	-0.000100967004982522\\
0.911973843384424	-9.77460804278962e-05\\
0.918263318166386	-9.43772865866777e-05\\
0.924552792948348	-9.08780348254368e-05\\
0.930842267730309	-8.72660457389429e-05\\
0.937131742512271	-8.35590799397323e-05\\
0.943421217294232	-7.97745888121426e-05\\
0.949710692076194	-7.59296672185883e-05\\
0.956000166858155	-7.20416828698944e-05\\
0.962289641640117	-6.81272285874002e-05\\
0.968579116422078	-6.4202984503936e-05\\
0.97486859120404	-6.02850086579565e-05\\
0.981158065986001	-5.63889771001413e-05\\
0.987447540767963	-5.25300965819042e-05\\
0.993737015549925	-4.87230900034774e-05\\
1.00002649033189	-4.49821927759331e-05\\
1.00631596511385	-4.13208254030906e-05\\
1.01260543989581	-3.77520154870581e-05\\
1.01889491467777	-3.42881030519493e-05\\
1.02518438945973	-3.09405186271761e-05\\
1.03147386424169	-2.77202889265027e-05\\
1.03776333902366	-2.46375311689917e-05\\
1.04405281380562	-2.17015513044316e-05\\
1.05034228858758	-1.89209495147225e-05\\
1.05663176336954	-1.63036238518544e-05\\
1.0629212381515	-1.38563773361966e-05\\
1.06921071293346	-1.15856710181106e-05\\
1.07550018771542	-9.49654713622294e-06\\
1.08178966249739	-7.59366957936436e-06\\
1.08807913727935	-5.88080729357898e-06\\
1.09436861206131	-4.36064874520525e-06\\
1.10065808684327	-3.03528941003606e-06\\
1.10694756162523	-1.90603896044195e-06\\
1.11323703640719	-9.73159330897033e-07\\
1.11952651118916	-2.36239429796115e-07\\
1.12581598597112	3.05681169265881e-07\\
1.13210546075308	6.54828909318894e-07\\
1.13839493553504	8.13659426057711e-07\\
1.144684410317	7.85727024776861e-07\\
1.15097388509896	5.75211743125692e-07\\
1.15726335988092	1.86799297807738e-07\\
1.16355283466289	-3.74093360733241e-07\\
1.16984230944485	-1.10112887341529e-06\\
1.17613178422681	-1.98798079509288e-06\\
1.18242125900877	-3.02712942357175e-06\\
1.18871073379073	-4.21097865910269e-06\\
1.19500020857269	-5.53120844415389e-06\\
1.20128968335466	-6.97901487001218e-06\\
1.20757915813662	-8.54538302519359e-06\\
1.21386863291858	-1.0220734111499e-05\\
1.22015810770054	-1.19952492241282e-05\\
1.2264475824825	-1.38586947286967e-05\\
1.23273705726446	-1.58005641424097e-05\\
1.23902653204642	-1.78102236532141e-05\\
1.24531600682839	-1.98766610992607e-05\\
1.25160548161035	-2.19890935113654e-05\\
1.25789495639231	-2.41362176893745e-05\\
1.26418443117427	-2.63067922787741e-05\\
1.27047390595623	-2.84896887023933e-05\\
1.27676338073819	-3.06735928461421e-05\\
1.28305285552016	-3.28471214743331e-05\\
1.28934233030212	-3.49995425494853e-05\\
1.29563180508408	-3.71196765627246e-05\\
1.30192127986604	-3.9196565921884e-05\\
1.308210754648	-4.12196459365077e-05\\
1.31450022942996	-4.31786575063597e-05\\
1.32078970421193	-4.50631123385392e-05\\
1.32707917899389	-4.68633334094193e-05\\
1.33336865377585	-4.85696873511188e-05\\
1.33965812855781	-5.01730864925776e-05\\
1.34594760333977	-5.16646323376335e-05\\
1.35223707812173	-5.30358556716237e-05\\
1.35852655290369	-5.4278858442558e-05\\
1.36481602768566	-5.53860518266447e-05\\
1.37110550246762	-5.63505564059597e-05\\
1.37739497724958	-5.71654818486422e-05\\
1.38368445203154	-5.78250801481772e-05\\
1.3899739268135	-5.832387614646e-05\\
1.39626340159546	-5.86567075515632e-05\\
1.40255287637743	-5.88193215662614e-05\\
1.40884235115939	-5.88078546570614e-05\\
1.41513182594135	-5.86191526963376e-05\\
1.42142130072331	-5.82506690989248e-05\\
1.42771077550527	-5.77002501813695e-05\\
1.43400025028723	-5.69664771319367e-05\\
1.44028972506919	-5.60486150789075e-05\\
1.44657919985116	-5.49463838979136e-05\\
1.45286867463312	-5.36601728526875e-05\\
1.45915814941508	-5.21911788382567e-05\\
1.46544762419704	-5.0540802476462e-05\\
1.471737098979	-4.87112884002272e-05\\
1.47802657376096	-4.67054960608948e-05\\
1.48431604854293	-4.45268378825858e-05\\
1.49060552332489	-4.21791810367722e-05\\
1.49689499810685	-3.96669129258953e-05\\
1.50318447288881	-3.69951631000731e-05\\
1.50947394767077	-3.41695922543295e-05\\
1.51576342245273	-3.1196053896565e-05\\
1.52205289723469	-2.80812673736364e-05\\
1.52834237201666	-2.48320975515526e-05\\
1.53463184679862	-2.1456100512296e-05\\
1.54092132158058	-1.7961130652111e-05\\
1.54721079636254	-1.43554425449111e-05\\
1.5535002711445	-1.06476618384477e-05\\
1.55978974592646	-6.84698898112401e-06\\
1.56607922070843	-2.96249345410615e-06\\
1.57236869549039	9.95991285890341e-07\\
1.57865817027235	5.01876274938695e-06\\
1.58494764505431	9.09559457795694e-06\\
1.59123711983627	1.32160348584875e-05\\
1.59752659461823	1.7369762645103e-05\\
1.60381606940019	2.15460568142589e-05\\
1.61010554418216	2.57339488598518e-05\\
1.61639501896412	2.99228195217438e-05\\
1.62268449374608	3.41015947924461e-05\\
1.62897396852804	3.82593025278766e-05\\
1.63526344331	4.23851088271476e-05\\
1.64155291809196	4.64679251308553e-05\\
1.64784239287393	5.04971467307769e-05\\
1.65413186765589	5.44618342246395e-05\\
1.66042134243785	5.83514265599661e-05\\
1.66671081721981	6.21555482211988e-05\\
1.67300029200177	6.58638928143773e-05\\
1.67928976678373	6.94662703608628e-05\\
1.68557924156569	7.29531675460748e-05\\
1.69186871634766	7.63148054829799e-05\\
1.69815819112962	7.95421656221151e-05\\
1.70444766591158	8.26263349154033e-05\\
1.71073714069354	8.55587531987112e-05\\
1.7170266154755	8.8331235019723e-05\\
1.72331609025746	9.09360496734735e-05\\
1.72960556503943	9.33659466682002e-05\\
1.73589503982139	9.56139992922544e-05\\
1.74218451460335	9.76737574092112e-05\\
1.74847398938531	9.95393202174455e-05\\
1.75476346416727	0.000101205234386725\\
1.76105293894923	0.000102666785096517\\
1.76734241373119	0.000103919439425226\\
1.77363188851316	0.000104959326563403\\
1.77992136329512	0.000105783252365654\\
1.78621083807708	0.000106388437416172\\
1.79250031285904	0.000106772633444052\\
1.798789787641	0.000106934534414904\\
1.80507926242296	0.000106873092590831\\
1.81136873720493	0.000106587718619267\\
1.81765821198689	0.000106078823591815\\
1.82394768676885	0.000105347007774981\\
1.83023716155081	0.000104393733636243\\
1.83652663633277	0.000103221005701926\\
1.84281611111473	0.00010183125777985\\
1.8491055858967	0.000100227567600086\\
1.85539506067866	9.84136495389976e-05\\
1.86168453546062	9.63936763582751e-05\\
1.86797401024258	9.41724283620715e-05\\
1.87426348502454	9.17551187740173e-05\\
1.8805529598065	8.914764111978e-05\\
1.88684243458846	8.63561690493952e-05\\
1.89313190937043	8.33875637908932e-05\\
1.89942138415239	8.02488684712444e-05\\
1.90571085893435	7.69479629525449e-05\\
1.91200033371631	7.34927962184884e-05\\
1.91828980849827	6.98919393471442e-05\\
1.92457928328023	6.61540398141369e-05\\
1.9308687580622	6.22884836047888e-05\\
1.93715823284416	5.83046821702737e-05\\
1.94344770762612	5.42125089850742e-05\\
1.94973718240808	5.00218338856939e-05\\
1.95602665719004	4.57432215625886e-05\\
1.962316131972	4.13869420299307e-05\\
1.96860560675396	3.69638692063745e-05\\
1.97489508153593	3.24847387673799e-05\\
1.98118455631789	2.79603882518131e-05\\
1.98747403109985	2.34020953939762e-05\\
1.99376350588181	1.8820705008693e-05\\
2.00005298066377	1.42274911922868e-05\\
2.00634245544573	9.63345883064903e-06\\
2.0126319302277	5.04965646541677e-06\\
2.01892140500966	4.87110810354352e-07\\
2.02521087979162	-4.04336606152356e-06\\
2.03150035457358	-8.53098026709631e-06\\
2.03778982935554	-1.29652471514419e-05\\
2.0440793041375	-1.73356565937866e-05\\
2.05036877891946	-2.16321386687923e-05\\
2.05665825370143	-2.58447325904854e-05\\
2.06294772848339	-2.99639614240732e-05\\
2.06923720326535	-3.39804137183819e-05\\
2.07552667804731	-3.78851327695884e-05\\
2.08181615282927	-4.16695438616443e-05\\
2.08810562761123	-4.53252832812723e-05\\
2.0943951023932	-4.88446457893588e-05\\
2.10068457717516	-5.22201444255188e-05\\
2.10697405195712	-5.54448270122521e-05\\
2.11326352673908	-5.85123889322858e-05\\
2.11955300152104	-6.14164455328137e-05\\
2.125842476303	-6.41516307950951e-05\\
2.13213195108496	-6.67131753289141e-05\\
2.13842142586693	-6.90963388478849e-05\\
2.14471090064889	-7.12971777829807e-05\\
2.15100037543085	-7.3312327003805e-05\\
2.15728985021281	-7.51389343349729e-05\\
2.16357932499477	-7.67745696066413e-05\\
2.16986879977673	-7.82177376095206e-05\\
2.1761582745587	-7.94670668256003e-05\\
2.18244774934066	-8.05218551249709e-05\\
2.18873722412262	-8.13820515759289e-05\\
2.19502669890458	-8.20482055132743e-05\\
2.20131617368654	-8.25213937787339e-05\\
2.2076056484685	-8.28030970296822e-05\\
2.21389512325046	-8.28954180178698e-05\\
2.22018459803243	-8.28008414828219e-05\\
2.22647407281439	-8.25228198664263e-05\\
2.23276354759635	-8.20647765067406e-05\\
2.23905302237831	-8.14308950793929e-05\\
2.24534249716027	-8.06256612122525e-05\\
2.25163197194223	-7.96542408352252e-05\\
2.2579214467242	-7.85219526733272e-05\\
2.26421092150616	-7.72346938902047e-05\\
2.27050039628812	-7.57988491386641e-05\\
2.27678987107008	-7.42208612791728e-05\\
2.28307934585204	-7.2507780714659e-05\\
2.289368820634	-7.06668179191183e-05\\
2.29565829541597	-6.87055726302788e-05\\
2.30194777019793	-6.66319101583213e-05\\
2.30823724497989	-6.44537431071512e-05\\
2.31452671976185	-6.21793733444065e-05\\
2.32081619454381	-5.98172446188983e-05\\
2.32710566932577	-5.73759316466749e-05\\
2.33339514410773	-5.48641619388945e-05\\
2.3396846188897	-5.22906884725671e-05\\
2.34597409367166	-4.96643442602362e-05\\
2.35226356845362	-4.69939732283819e-05\\
2.35855304323558	-4.42884957010392e-05\\
2.36484251801754	-4.15566901210696e-05\\
2.3711319927995	-3.8807382225059e-05\\
2.37742146758147	-3.60492194886319e-05\\
2.38371094236343	-3.32908311975189e-05\\
2.39000041714539	-3.05405628751032e-05\\
2.39628989192735	-2.78067018371075e-05\\
2.40257936670931	-2.50973898801021e-05\\
2.40886884149127	-2.24203540710732e-05\\
2.41515831627323	-1.97832341655158e-05\\
2.4214477910552	-1.71933352248743e-05\\
2.42773726583716	-1.46577258419711e-05\\
2.43402674061912	-1.21830562420655e-05\\
2.44031621540108	-9.77573290583678e-06\\
2.44660569018304	-7.4416930146981e-06\\
2.452895164965	-5.18685919814743e-06\\
2.45918463974697	-3.01620093523525e-06\\
2.46547411452893	-9.34855052037165e-07\\
2.47176358931089	1.05287836049683e-06\\
2.47805306409285	2.94285928248428e-06\\
2.48434253887481	4.73142426926643e-06\\
2.49063201365677	6.41553197056055e-06\\
2.49692148843873	7.99232657300308e-06\\
2.5032109632207	9.45954525377601e-06\\
2.50950043800266	1.08154126792215e-05\\
2.51578991278462	1.20583208627068e-05\\
2.52207938756658	1.31874694488943e-05\\
2.52836886234854	1.42023600346874e-05\\
2.5346583371305	1.51028434629552e-05\\
2.54094781191247	1.58894945343491e-05\\
2.54723728669443	1.65628007380292e-05\\
2.55352676147639	1.71242463693488e-05\\
2.55981623625835	1.75754212250467e-05\\
2.56610571104031	1.79182425199542e-05\\
2.57239518582227	1.81552895810455e-05\\
2.57868466060423	1.82891562872101e-05\\
2.5849741353862	1.83231677510776e-05\\
2.59126361016816	1.82608418981545e-05\\
2.59755308495012	1.81056784640532e-05\\
2.60384255973208	1.78620575752575e-05\\
2.61013203451404	1.7534097423777e-05\\
2.616421509296	1.71268766280264e-05\\
2.62271098407797	1.6644782590447e-05\\
2.62900045885993	1.60931631398853e-05\\
2.63528993364189	1.54772169480566e-05\\
2.64157940842385	1.48024264490232e-05\\
2.64786888320581	1.40742849907838e-05\\
2.65415835798777	1.32986424432602e-05\\
2.66044783276973	1.24811231216881e-05\\
2.6667373075517	1.16276678454597e-05\\
2.67302678233366	1.07441592263058e-05\\
2.67931625711562	9.83666905085556e-06\\
2.68560573189758	8.91071977093816e-06\\
2.69189520667954	7.97262691776268e-06\\
2.6981846814615	7.02777833794244e-06\\
2.70447415624347	6.08237314736471e-06\\
2.71076363102543	5.14170460519381e-06\\
2.71705310580739	4.21135337091982e-06\\
2.72334258058935	3.29679096466862e-06\\
2.72963205537131	2.40315785049461e-06\\
2.73592153015327	1.53549262904562e-06\\
2.74221100493523	6.98350049788132e-07\\
2.7485004797172	-1.0331132216379e-07\\
2.75478995449916	-8.65489710122347e-07\\
2.76107942928112	-1.584008714417e-06\\
2.76736890406308	-2.25496842176653e-06\\
2.77365837884504	-2.87499642581679e-06\\
2.779947853627	-3.44079307978973e-06\\
2.78623732840897	-3.94959351979196e-06\\
2.79252680319093	-4.39877112512477e-06\\
2.79881627797289	-4.78590663988143e-06\\
2.80510575275485	-5.10939935338683e-06\\
2.81139522753681	-5.36747756996192e-06\\
2.81768470231877	-5.55885344510898e-06\\
2.82397417710074	-5.68298855796456e-06\\
2.8302636518827	-5.738820618717e-06\\
2.83655312666466	-5.72656790609471e-06\\
2.84284260144662	-5.6461722124368e-06\\
2.84913207622858	-5.49809919903055e-06\\
2.85542155101054	-5.28330201632343e-06\\
2.8617110257925	-5.0028356781695e-06\\
2.86800050057447	-4.65813354821876e-06\\
2.87428997535643	-4.25092366640456e-06\\
2.88057945013839	-3.78349432139657e-06\\
2.88686892492035	-3.25783912558109e-06\\
2.89315839970231	-2.67682844423689e-06\\
2.89944787448427	-2.04324896913022e-06\\
2.90573734926624	-1.36028393171728e-06\\
2.9120268240482	-6.31163857178763e-07\\
2.91831629883016	1.40560587169603e-07\\
2.92460577361212	9.50967660173774e-07\\
2.93089524839408	1.79628113983199e-06\\
2.93718472317604	2.67247014562599e-06\\
2.943474197958	3.57521639671177e-06\\
2.94976367273997	4.50040533905849e-06\\
2.95605314752193	5.44341673958115e-06\\
2.96234262230389	6.39989957562648e-06\\
2.96863209708585	7.36528090783395e-06\\
2.97492157186781	8.33497324492782e-06\\
2.98121104664977	9.30449459701777e-06\\
2.98750052143174	1.02693084045313e-05\\
2.9937899962137	1.12248671939597e-05\\
3.00007947099566	1.21667653729673e-05\\
3.00636894577762	1.30906410049647e-05\\
3.01265842055958	1.39920230139978e-05\\
3.01894789534154	1.4867095160298e-05\\
3.0252373701235	1.57116264745127e-05\\
3.03152684490547	1.65218007168733e-05\\
3.03781631968743	1.72939908225089e-05\\
3.04410579446939	1.80246170202736e-05\\
3.05039526925135	1.87104451470077e-05\\
3.05668474403331	1.93483647308312e-05\\
3.06297421881527	1.99356618395541e-05\\
3.06926369359724	2.0469653463806e-05\\
3.0755531683792	2.0948040400981e-05\\
3.08184264316116	2.13687671930529e-05\\
3.08813211794312	2.1730143998866e-05\\
3.09442159272508	2.20305391849251e-05\\
3.10071106750704	2.22689959628042e-05\\
3.107000542289	2.24444665946066e-05\\
3.11329001707097	2.25564454012783e-05\\
3.11957949185293	2.26045740419067e-05\\
3.12586896663489	2.2589028958464e-05\\
3.13215844141685	2.25100102397846e-05\\
3.13844791619881	2.23683546209941e-05\\
3.14473739098077	2.21649133891333e-05\\
3.15102686576274	2.19009707507212e-05\\
3.1573163405447	2.1578218365903e-05\\
3.16360581532666	2.11981860047672e-05\\
3.16989529010862	2.07633529498708e-05\\
3.17618476489058	2.02759256353602e-05\\
3.18247423967254	1.97385525098071e-05\\
3.1887637144545	1.91543458640808e-05\\
3.19505318923647	1.85262215381954e-05\\
3.20134266401843	1.78576319740387e-05\\
3.20763213880039	1.71522915479727e-05\\
3.21392161358235	1.64137200044934e-05\\
3.22021108836431	1.5645993698854e-05\\
3.22650056314627	1.48531999002444e-05\\
3.23279003792824	1.40395895869005e-05\\
3.2390795127102	1.32095919980202e-05\\
3.24536898749216	1.23674199130619e-05\\
3.25165846227412	1.1517839084263e-05\\
3.25794793705608	1.06652823887998e-05\\
3.26423741183804	9.81450830295216e-06\\
3.27052688662	8.97011523193214e-06\\
3.27681636140197	8.13679980637971e-06\\
3.28310583618393	7.31898944650311e-06\\
3.28939531096589	6.52155540592503e-06\\
3.29568478574785	5.74870500713587e-06\\
3.30197426052981	5.00500573252793e-06\\
3.30826373531177	4.29465580964461e-06\\
3.31455321009374	3.62211721949279e-06\\
3.3208426848757	2.99092425848357e-06\\
3.32713215965766	2.4053570086835e-06\\
3.33342163443962	1.8688751879381e-06\\
3.33971110922158	1.38493487611413e-06\\
3.34600058400354	9.56866642809473e-07\\
3.3522900587855	5.87615431868471e-07\\
3.35857953356747	2.79937012237497e-07\\
3.36486900834943	3.62688297173008e-08\\
3.37115848313139	-1.41128111863509e-07\\
3.37744795791335	-2.50607627094723e-07\\
3.38373743269531	-2.90176103590056e-07\\
3.39002690747727	-2.58876752923243e-07\\
3.39631638225924	-1.55609086505137e-07\\
3.4026058570412	2.01252987608314e-08\\
3.40889533182316	2.68531948677264e-07\\
3.41518480660512	5.89812771067955e-07\\
3.42147428138708	9.83383870334364e-07\\
3.42776375616904	1.44848490890581e-06\\
3.43405323095101	1.98390807781834e-06\\
3.44034270573297	2.58828913501929e-06\\
3.44663218051493	3.25988730764948e-06\\
3.45292165529689	3.99632517655846e-06\\
3.45921113007885	4.79526534036268e-06\\
3.46550060486081	5.65389564144425e-06\\
3.47179007964277	6.56908741802908e-06\\
3.47807955442474	7.5372590799816e-06\\
3.4843690292067	8.5547253547702e-06\\
3.49065850398866	9.61727710091509e-06\\
3.49694797877062	1.07207997643854e-05\\
3.50323745355258	1.18604875751771e-05\\
3.50952692833454	1.30316329887137e-05\\
3.51581640311651	1.42291210067924e-05\\
3.52210587789847	1.54475201270543e-05\\
3.52839535268043	1.66814024851192e-05\\
3.53468482746239	1.79250728251645e-05\\
3.54097430224435	1.91726940101944e-05\\
3.54726377702631	2.04181651497493e-05\\
3.55355325180827	2.16555636143312e-05\\
3.55984272659024	2.28785665967735e-05\\
3.5661322013722	2.40810168179451e-05\\
3.57242167615416	2.5256455046474e-05\\
3.57871115093612	2.63985912170028e-05\\
3.58500062571808	2.75010879704496e-05\\
3.59129010050004	2.85574897134211e-05\\
3.59757957528201	2.95617483061505e-05\\
3.60386905006397	3.05073881463613e-05\\
3.61015852484593	3.13885466312058e-05\\
3.61644799962789	3.21989991789451e-05\\
3.62273747440985	3.29328722727951e-05\\
3.62902694919181	3.35845616064034e-05\\
3.63531642397377	3.41486756951781e-05\\
3.64160589875574	3.46196266036713e-05\\
3.6478953735377	3.49924830516102e-05\\
3.65418484831966	3.52624010702129e-05\\
3.66047432310162	3.54248913936317e-05\\
3.66676379788358	3.54755738953827e-05\\
3.67305327266554	3.54105031874496e-05\\
3.67934274744751	3.52262031810824e-05\\
3.68563222222947	3.49192469002446e-05\\
3.69192169701143	3.44867812600569e-05\\
3.69821117179339	3.3926236028492e-05\\
3.70450064657535	3.32354238707921e-05\\
3.71079012135731	3.24127377098193e-05\\
3.71707959613927	3.14566823362838e-05\\
3.72336907092124	3.0366411920113e-05\\
3.7296585457032	2.91414162347792e-05\\
3.73594802048516	2.77817280220916e-05\\
3.74223749526712	2.62877192653832e-05\\
3.74852697004908	2.46600975515321e-05\\
3.75481644483104	2.29002598644001e-05\\
3.76110591961301	2.10099542528042e-05\\
3.76739539439497	1.89914453585516e-05\\
3.77368486917693	1.68472943187226e-05\\
3.77997434395889	1.45805852298508e-05\\
3.78626381874085	1.21949306048919e-05\\
3.79255329352281	9.69427674135659e-06\\
3.79884276830477	7.08296283846721e-06\\
3.80513224308674	4.36603022535564e-06\\
3.8114217178687	1.54849749378627e-06\\
3.81771119265066	-1.36397284222767e-06\\
3.82400066743262	-4.36515165347373e-06\\
3.83029014221458	-7.44894896342885e-06\\
3.83657961699654	-1.06083307400695e-05\\
3.84286909177851	-1.38364248414291e-05\\
3.84915856656047	-1.71256124303909e-05\\
3.85544804134243	-2.04681828108733e-05\\
3.86173751612439	-2.38563306993456e-05\\
3.86802699090635	-2.72815450443886e-05\\
3.87431646568831	-3.07354566757567e-05\\
3.88060594047027	-3.42092816936201e-05\\
3.88689541525224	-3.76940793103131e-05\\
3.8931848900342	-4.11806586271268e-05\\
3.89947436481616	-4.46599319730012e-05\\
3.90576383959812	-4.81223814858822e-05\\
3.91205331438008	-5.15587225891068e-05\\
3.91834278916204	-5.49592155039136e-05\\
3.92463226394401	-5.8314318266639e-05\\
3.93092173872597	-6.1614511650987e-05\\
3.93721121350793	-6.48500540592067e-05\\
3.94350068828989	-6.80115908835432e-05\\
3.94979016307185	-7.10894978510623e-05\\
3.95607963785381	-7.4074439908145e-05\\
3.96236911263578	-7.69571488490328e-05\\
3.96865858741774	-7.97283710198826e-05\\
3.9749480621997	-8.23792979645077e-05\\
3.98123753698166	-8.49010305046249e-05\\
3.98752701176362	-8.7285227891698e-05\\
3.99381648654558	-8.95234493327735e-05\\
4.00010596132754	-9.16078163299971e-05\\
4.00639543610951	-9.35305079110549e-05\\
4.01268491089147	-9.52842949573096e-05\\
4.01897438567343	-9.68619315244723e-05\\
4.02526386045539	-9.82569258667354e-05\\
4.03155333523735	-9.94630112245432e-05\\
4.03784281001931	-0.000100474257806127\\
4.04413228480128	-0.000101285257358086\\
4.05042175958324	-0.000101891128167608\\
4.0567112343652	-0.000102287103345589\\
4.06300070914716	-0.000102469455256937\\
4.06929018392912	-0.000102434480055535\\
4.07557965871108	-0.000102179084137788\\
4.08186913349304	-0.000101700848858854\\
4.08815860827501	-0.000100997873914821\\
4.09444808305697	-0.000100068695701339\\
4.10073755783893	-9.89122456758196e-05\\
4.10702703262089	-9.75284134483445e-05\\
4.11331650740285	-9.59173989940609e-05\\
4.11960598218481	-9.40799054660602e-05\\
4.12589545696678	-9.20172756195825e-05\\
4.13218493174874	-8.97317868293612e-05\\
4.1384744065307	-8.72255218382634e-05\\
4.14476388131266	-8.45017223127797e-05\\
4.15105335609462	-8.1564050333327e-05\\
4.15734283087658	-7.84167539222835e-05\\
4.16363230565854	-7.5064558586746e-05\\
4.16992178044051	-7.15125765964331e-05\\
4.17621125522247	-6.7766954543913e-05\\
4.18250073000443	-6.38337214695639e-05\\
4.18879020478639	-5.97199991716479e-05\\
4.19507967956835	-5.5433079978684e-05\\
4.20136915435031	-5.09807705384446e-05\\
4.20765862913228	-4.63713436147373e-05\\
4.21394810391424	-4.16137618231005e-05\\
4.2202375786962	-3.67168681805197e-05\\
4.22652705347816	-3.16907676278788e-05\\
4.23281652826012	-2.65450867118489e-05\\
4.23910600304208	-2.1290305085131e-05\\
4.24539547782405	-1.59372025336779e-05\\
4.25168495260601	-1.04967657534871e-05\\
4.25797442738797	-4.98033932672115e-06\\
4.26426390216993	6.00458406552207e-07\\
4.27055337695189	6.23381492914632e-06\\
4.27684285173385	1.19075666589197e-05\\
4.28313232651581	1.76094208654831e-05\\
4.28942180129777	2.33270620810799e-05\\
4.29571127607974	2.90479465547833e-05\\
4.3020007508617	3.47592867910862e-05\\
4.30829022564366	4.044855995744e-05\\
4.31457970042562	4.61028084828286e-05\\
4.32086917520758	5.17092839800171e-05\\
4.32715864998954	5.72553371966933e-05\\
4.33344812477151	6.2728251577937e-05\\
4.33973759955347	6.81153196637752e-05\\
4.34602707433543	7.34040340830688e-05\\
4.35231654911739	7.858212939027e-05\\
4.35860602389935	8.36374510981841e-05\\
4.36489549868131	8.85579102032352e-05\\
4.37118497346328	9.333188245364e-05\\
4.37747444824524	9.79478754743468e-05\\
4.3837639230272	0.000102394716122944\\
4.39005339780916	0.000106661656900542\\
4.39634287259112	0.000110738048533676\\
4.40263234737308	0.000114613957521215\\
4.40892182215505	0.000118279725938919\\
4.41521129693701	0.000121726006909739\\
4.42150077171897	0.000124944000162941\\
4.42779024650093	0.000127925583910837\\
4.43407972128289	0.00013066274823359\\
4.44036919606485	0.000133148437271302\\
4.44665867084681	0.000135375878926425\\
4.45294814562878	0.000137338824060862\\
4.45923762041074	0.000139031974867976\\
4.4655270951927	0.00014045015177544\\
4.47181656997466	0.000141589230224781\\
4.47810604475662	0.000142445357596444\\
4.48439551953858	0.000143015628964349\\
4.49068499432054	0.000143297692375199\\
4.49697446910251	0.000143289733387064\\
4.50326394388447	0.000142990842505242\\
4.50955341866643	0.000142400640470441\\
4.51584289344839	0.000141519332828466\\
4.52213236823035	0.000140348221975728\\
4.52842184301231	0.00013888870489609\\
4.53471131779428	0.000137143399115303\\
4.54100079257624	0.000135115407829289\\
4.5472902673582	0.000132808516354999\\
4.55357974214016	0.000130227019326412\\
4.55986921692212	0.000127376148157055\\
4.56615869170408	0.000124261689052219\\
4.57244816648605	0.000120890015750774\\
4.57873764126801	0.000117268216854427\\
4.58502711604997	0.000113403932118672\\
4.59131659083193	0.000109305601654341\\
4.59760606561389	0.000104981731055886\\
4.60389554039585	0.000100442048278637\\
4.61018501517781	9.56965304794721e-05\\
4.61647448995978	9.07555677258642e-05\\
4.62276396474174	8.5630052126362e-05\\
4.6290534395237	8.03316015662858e-05\\
4.63534291430566	7.4871959441225e-05\\
4.64163238908762	6.92635567247635e-05\\
4.64792186386958	6.35189699096372e-05\\
4.65421133865155	5.76513339183293e-05\\
4.66050081343351	5.16739310114644e-05\\
4.66679028821547	4.5600321755046e-05\\
4.67307976299743	3.94446215068456e-05\\
4.67936923777939	3.32207528117578e-05\\
4.68565871256135	2.69430474872934e-05\\
4.69194818734332	2.06260829145322e-05\\
4.69823766212528	1.42841381602921e-05\\
4.70452713690724	7.93193430581596e-06\\
4.7108166116892	1.58411785378121e-06\\
4.71710608647116	-4.74461558042094e-06\\
4.72339556125312	-1.10397140815621e-05\\
4.72968503603508	-1.7286756701651e-05\\
4.73597451081705	-2.34714043472195e-05\\
4.74226398559901	-2.95793197437888e-05\\
4.74855346038097	-3.5596565794549e-05\\
4.75484293516293	-4.15093236370012e-05\\
4.76113240994489	-4.73041054647183e-05\\
4.76742188472685	-5.29677854501642e-05\\
4.77371135950882	-5.84875688218744e-05\\
4.78000083429078	-6.38509482087102e-05\\
4.78629030907274	-6.90459328325232e-05\\
4.7925797838547	-7.40609284548555e-05\\
4.79886925863666	-7.88848137744935e-05\\
4.80515873341862	-8.35071641631657e-05\\
4.81144820820058	-8.79178533068625e-05\\
4.81773768298255	-9.21075388760073e-05\\
4.82402715776451	-9.6067245976883e-05\\
4.83031663254647	-9.97889928839868e-05\\
4.83660610732843	-0.000103265070720227\\
4.84289558211039	-0.000106488680103212\\
4.84918505689235	-0.000109453600089182\\
4.85547453167431	-0.000112154484668281\\
4.86176400645628	-0.000114586506242631\\
4.86805348123824	-0.000116745766717941\\
4.8743429560202	-0.000118629059215891\\
4.88063243080216	-0.000120233878988074\\
4.88692190558412	-0.000121558645332698\\
4.89321138036608	-0.000122602319606813\\
4.89950085514805	-0.000123364992759889\\
4.90579032993001	-0.000123847388749709\\
4.91207980471197	-0.000124050893646199\\
4.91836927949393	-0.000123977733892389\\
4.92465875427589	-0.000123631054520956\\
4.93094822905785	-0.000123014573546243\\
4.93723770383982	-0.000122132945762132\\
4.94352717862178	-0.000120991671792581\\
4.94981665340374	-0.000119596514196019\\
4.9561061281857	-0.000117954425149946\\
4.96239560296766	-0.000116072804303258\\
4.96868507774962	-0.000113959786176565\\
4.97497455253158	-0.000111624378405395\\
4.98126402731355	-0.00010907565956586\\
4.98755350209551	-0.000106323863292346\\
4.99384297687747	-0.000103379623396904\\
5.00013245165943	-0.00010025390474766\\
5.00642192644139	-9.69584198173834e-05\\
5.01271140122335	-9.35052321437979e-05\\
5.01900087600531	-8.99068636499578e-05\\
5.02529035078728	-8.61761764099356e-05\\
5.03157982556924	-8.23263999336632e-05\\
5.0378693003512	-7.83710729592713e-05\\
5.04415877513316	-7.43242344469763e-05\\
5.05044824991512	-7.01994940754957e-05\\
5.05673772469708	-6.60114601487294e-05\\
5.06302719947905	-6.17743007751415e-05\\
5.06931667426101	-5.75023659621365e-05\\
5.07560614904297	-5.32102731085615e-05\\
5.08189562382493	-4.89123731313157e-05\\
5.08818509860689	-4.462316428544e-05\\
5.09447457338885	-4.03568565161549e-05\\
5.10076404817082	-3.6127589737589e-05\\
5.10705352295278	-3.19493128699833e-05\\
5.11334299773474	-2.78358256764477e-05\\
5.1196324725167	-2.38003640333773e-05\\
5.12592194729866	-1.98561128854635e-05\\
5.13221142208062	-1.60157642312697e-05\\
5.13850089686258	-1.22915016618208e-05\\
5.14479037164455	-8.69505038281204e-06\\
5.15107984642651	-5.23790640727384e-06\\
5.15736932120847	-1.93075538845733e-06\\
5.16365879599043	1.21651646622922e-06\\
5.16994827077239	4.19412208430003e-06\\
5.17623774555435	6.99329484632472e-06\\
5.18252722033631	9.6057929113158e-06\\
5.18881669511828	1.20241938930121e-05\\
5.19510616990024	1.42419867188437e-05\\
5.2013956446822	1.62530823217821e-05\\
5.20768511946416	1.80524639290525e-05\\
5.21397459424612	1.96358769244398e-05\\
5.22026406902808	2.09997733691125e-05\\
5.22655354381005	2.2141709450807e-05\\
5.23284301859201	2.3059904378897e-05\\
5.23913249337397	2.37536360145896e-05\\
5.24542196815593	2.42228261413402e-05\\
5.25171144293789	2.44685979851056e-05\\
5.25800091771985	2.44928587562754e-05\\
5.26429039250182	2.42982468989794e-05\\
5.27057986728378	2.38885395447141e-05\\
5.27686934206574	2.32683232752606e-05\\
5.2831588168477	2.24428717956471e-05\\
5.28944829162966	2.14187411984312e-05\\
5.29573776641162	2.02029182219121e-05\\
5.30202724119358	1.8803440980264e-05\\
5.30831671597555	1.72290606315073e-05\\
5.31460619075751	1.54894028128183e-05\\
5.32089566553947	1.35947316266538e-05\\
5.32718514032143	1.15561601887748e-05\\
5.33347461510339	9.38527591642924e-06\\
5.33976408988535	7.09466849002638e-06\\
5.34605356466732	4.69718224849203e-06\\
5.35234303944928	2.20649621951452e-06\\
5.35863251423124	-3.63361550625996e-07\\
5.3649219890132	-2.99776388601458e-06\\
5.37121146379516	-5.68184054827725e-06\\
5.37750093857712	-8.40005714053405e-06\\
5.38379041335909	-1.11366820192416e-05\\
5.39007988814105	-1.38757436616288e-05\\
5.39636936292301	-1.66011861892912e-05\\
5.40265883770497	-1.92963512404276e-05\\
5.40894831248693	-2.19449050860021e-05\\
5.41523778726889	-2.4530298340153e-05\\
5.42152726205085	-2.70359277578791e-05\\
5.42781673683282	-2.94454453637627e-05\\
5.43410621161478	-3.17423058504573e-05\\
5.44039568639674	-3.39106162527969e-05\\
5.4466851611787	-3.59343100626575e-05\\
5.45297463596066	-3.77979915811011e-05\\
5.45926411074262	-3.94863591282046e-05\\
5.46555358552459	-4.09845540616516e-05\\
5.47184306030655	-4.22784273723664e-05\\
5.47813253508851	-4.33542863902403e-05\\
5.48442200987047	-4.41989682258281e-05\\
5.49071148465243	-4.47999082098249e-05\\
5.49700095943439	-4.51454920948891e-05\\
5.50329043421635	-4.52248696092283e-05\\
5.50957990899832	-4.50276875199052e-05\\
5.51586938378028	-4.45447517449793e-05\\
5.52215885856224	-4.37678108937689e-05\\
5.5284483333442	-4.26893270741857e-05\\
5.53473780812616	-4.13027723880077e-05\\
5.54102728290812	-3.96029809053289e-05\\
5.54731675769009	-3.75855124730151e-05\\
5.55360623247205	-3.52473634848138e-05\\
5.55989570725401	-3.25861747114686e-05\\
5.56618518203597	-2.96011330647161e-05\\
5.57247465681793	-2.62925987044582e-05\\
5.57876413159989	-2.26617730731959e-05\\
5.58505360638185	-1.87115520020598e-05\\
5.59134308116382	-1.44456043926766e-05\\
5.59763255594578	-9.86903160082875e-06\\
5.60392203072774	-4.98824374517426e-06\\
5.6102115055097	1.89402271644212e-07\\
5.61650098029166	5.65487243875396e-06\\
5.62279045507362	1.13987007352989e-05\\
5.62907992985559	1.74093729583547e-05\\
5.63536940463755	2.36746500377194e-05\\
5.64165887941951	3.01811733152135e-05\\
5.64794835420147	3.69142944691703e-05\\
5.65423782898343	4.38584465882741e-05\\
5.66052730376539	5.09967394464184e-05\\
5.66681677854735	5.83116088819224e-05\\
5.67310625332932	6.57842138025444e-05\\
5.67939572811128	7.33949655113975e-05\\
5.68568520289324	8.11232966952957e-05\\
5.6919746776752	8.89478706085356e-05\\
5.69826415245716	9.68465192272561e-05\\
5.70455362723912	0.000104796439700294\\
5.71084310202109	0.000112774227090995\\
5.71713257680305	0.000120755550597096\\
5.72342205158501	0.000128716179460753\\
5.72971152636697	0.000136630724227871\\
5.73600100114893	0.000144473986438243\\
5.74229047593089	0.000152220254676649\\
5.74857995071285	0.000159843679284677\\
5.75486942549482	0.00016731823234295\\
5.76115890027678	0.000174617578522884\\
5.76744837505874	0.000181715908183833\\
5.7737378498407	0.000188587173397536\\
5.78002732462266	0.000195205522686592\\
5.78631679940462	0.000201545464733499\\
5.79260627418659	0.000207581866561668\\
5.79889574896855	0.000213290102692554\\
5.80518522375051	0.00021864594782528\\
5.81147469853247	0.000223625826038187\\
5.81776417331443	0.000228206921747187\\
5.82405364809639	0.000232366986892885\\
5.83034312287835	0.000236084812058834\\
5.83663259766032	0.00023934013552207\\
5.84292207244228	0.00024211359959736\\
5.84921154722424	0.000244386843405664\\
5.8555010220062	0.000246142779360525\\
5.86179049678816	0.00024736571322137\\
5.86807997157012	0.000248040803853655\\
5.87436944635209	0.00024815487813612\\
5.88065892113405	0.000247695912548807\\
5.88694839591601	0.000246653649810469\\
5.89323787069797	0.000245019135036273\\
5.89952734547993	0.000242784928559558\\
5.90581682026189	0.000239945165958488\\
5.91210629504385	0.000236495685385307\\
5.91839576982582	0.000232434242207091\\
5.92468524460778	0.000227759664994664\\
5.93097471938974	0.000222473063331563\\
5.9372641941717	0.000216576998354867\\
5.94355366895366	0.000210075675568078\\
5.94984314373562	0.000202975414140383\\
5.95613261851759	0.000195284002984408\\
5.96242209329955	0.000187011177331442\\
5.96871156808151	0.00017816814215621\\
5.97500104286347	0.000168767903232947\\
5.98129051764543	0.000158825390826678\\
5.98757999242739	0.000148357030411717\\
5.99386946720936	0.000137381081003696\\
6.00015894199132	0.000125917009427212\\
6.00644841677328	0.000113986319774995\\
6.01273789155524	0.000101611851277994\\
6.0190273663372	8.88179674802814e-05\\
6.02531684111916	7.56302870286163e-05\\
6.03160631590112	6.20759565208573e-05\\
6.03789579068309	4.81835122627672e-05\\
6.04418526546505	3.39825492119417e-05\\
6.05047474024701	1.95038555830251e-05\\
6.05676421502897	4.77948196930811e-06\\
6.06305368981093	-1.01579389593098e-05\\
6.06934316459289	-2.52743884630036e-05\\
6.07563263937485	-4.05353530368302e-05\\
6.08192211415682	-5.59056134079583e-05\\
6.08821158893878	-7.13489789632149e-05\\
6.09450106372074	-8.68289171194192e-05\\
6.1007905385027	-0.000102308495115722\\
6.10708001328466	-0.000117750463687116\\
6.11336948806662	-0.000133117155201035\\
6.11965896284859	-0.000148371032992145\\
6.12594843763055	-0.000163474345754366\\
6.13223791241251	-0.000178389414941194\\
6.13852738719447	-0.000193078991287621\\
6.14481686197643	-0.000207505825528642\\
6.15110633675839	-0.000221633075852878\\
6.15739581154036	-0.000235424726270139\\
6.16368528632232	-0.000248845193709712\\
6.16997476110428	-0.000261859375314089\\
6.17626423588624	-0.000274433390586637\\
6.1825537106682	-0.000286534064798616\\
6.18884318545016	-0.000298129252769286\\
6.19513266023212	-0.000309188075334532\\
6.20142213501409	-0.000319680595566751\\
6.20771160979605	-0.00032957837174763\\
6.21400108457801	-0.000338854348228779\\
6.22029055935997	-0.000347482844517799\\
6.22658003414193	-0.000355439795384882\\
6.23286950892389	-0.000362702623533551\\
6.23915898370586	-0.000369250657968223\\
6.24544845848782	-0.000375064890249632\\
6.25173793326978	-0.000380127938115038\\
6.25802740805174	-0.000384424598451005\\
6.2643168828337	-0.00038794164720457\\
6.27060635761566	-0.000390667180909077\\
6.27689583239763	-0.000392592162825167\\
6.28318530717959	-0.000393708862247877\\
};
%\addlegendentry{data1}

\end{axis}
\end{tikzpicture}%
}
\caption{Plot eines Interpolanten über den Rand des Gebietes}
\label{fig:rand}
\end{figure}

Wir stellen zur Vereinfachung zunächst fest, dass es bei einer \ac{PDE} mit Dirichlet-Randwerten genügt, konstante Nullrandwerte  zu betrachten. Dafür sei eine \gls{PDE} wie in \eqref{eq:PDE} mit $B=I$, wobei $I$ die Identität bezeichnet, gegeben. Wir nehmen an, dass eine Funktion $\bar{g} \in C^k(\Omega) \cap C^0 (\widebar \Omega)$ existiert mit $\bar{g}|_{\partial \Omega} = g$. Damit gilt $u = \bar{u} + \bar{g}$ für eine Funktion $\bar{u}$. Eingesetzt erhalten wir
\begin{align*}
L\bar{u}(x) + L\bar{g}(x) &= f(x) , x \in \Omega\\
\bar{u}(x) + \bar{g}(x) &= g(x) , x \in \partial \Omega,
\end{align*}
was äquivalent dazu ist, dass wir folgende \ac{PDE} nach $\bar{u}$ lösen:
\begin{align*}
L\bar{u}(x) &= f(x) + L\bar{g}(x), x \in \Omega\\
\bar{u}(x) &= 0, x \in \partial \Omega.
\end{align*}

Die Idee ist jetzt, einen Kern, beziehungsweise Ansatzfunktionen zu konstruieren, welche auf dem Rand von $\Omega$ Null ist. Der Interpolant ist eine Linearkombination aus diesen Funktionen und wird demnach auf dem Rand auch Null sein. Dafür führen wir Gewichtsfunktionen ein, die dann in Verbindung mit einem gegebenen Kern das Geforderte erfüllen werden. Inspiriert ist dieses Vorgehen von \textcite{Hollig.2013}. Dort findet man auch weitere Theorie der hier sehr einfach eingeführten Gewichtsfunktionen.
\section{Gewichtsfunktionen}
\label{sec:gewicht}
\begin{definition}
Sei $\Omega \subset \mathbb{R}^d$ offen und beschränkt. Eine Funktion $w:\mathbb{R}^n \rightarrow \mathbb{R}$ heißt Gewichtsfunktion auf $\Omega$, wenn sie folgende Eigenschaften erfüllt:
\begin{enumerate}
\item $w(x) > 0$ für alle $x \in \Omega$
\item $w(x) = 0$ für alle $x \in \partial \Omega$
\item $w(x) < 0$ für alle $x \in \mathbb{R}^n \setminus \widebar{\Omega}$
\end{enumerate}
\end{definition}

\begin{theorem}
\label{thm:Gewicht}
Seien $\Omega_1, \Omega_2 \subset \mathbb{R}^d$ zwei offene und beschränkte Mengen und $w_{1}, w_{2}$ dazugehörige Gewichtsfunktionen. Dann gilt:
\begin{enumerate}
\item Für das Komplement $\Omega = \mathbb{R}^n \setminus \widebar \Omega_1$ ist $w = -w_1$ eine Gewichtsfunktion.
\item Für die Vereinigung $\Omega = \Omega_1 \cup \Omega_2$ ist $w = w_1 + w_2 + \sqrt{w_1^2 + w_2^2}$ eine Gewichtsfunktion.
\item Für den Schnitt $\Omega = \Omega_1 \cap \Omega_2$ ist $w = w_1 + w_2 - \sqrt{w_1^2 + w_2^2}$ eine Gewichtsfunktion.
\end{enumerate}
\end{theorem}
\begin{proof}
$\mbox{}$
\begin{enumerate}

\item
\begin{itemize}
\item
Sei $x \in \Omega =  \mathbb{R}^d \setminus \widebar \Omega_1$.
\begin{align*}
w(x) = - w_1(x) > 0
\end{align*}
\item
Sei $x \in \partial \Omega = \partial(\mathbb{R}^d \setminus \widebar \Omega_1) = \partial \Omega_1$
\begin{align*}
w(x) = -w_1(x) = 0
\end{align*}
\item
Sei $x \in \mathbb{R}^d \setminus \widebar{\Omega} = \Omega_1$
\begin{align*}
w(x) = -w_1(x) < 0
\end{align*}
\end{itemize}

\item
\begin{itemize}
\item
Sei $x \in \Omega_1, x \in \Omega_2$. $\Rightarrow w_1(x) >0, w_2(x)>0$
\begin{align*}
w(x) = w_1(x) + w_2(x) + \sqrt{w_1(x)^2 + w_2(x)^2} > 0
\end{align*}

\item
Sei \ac{oBdA} $x \in \Omega_1, x \notin \widebar{\Omega_2}$. $\Rightarrow w_1(x) > 0, w_2(x) <  0$
\begin{align*}
w(x) &=  w_1(x) + w_2(x) + \sqrt{w_1(x)^2 + w_2(x)^2} \\
&> w_1(x) + w_2(x) + \underbrace{\sqrt{w_2(x)^2}}_{=|w_2(x)| = -w_2(x)} \\
&= w_1(x) + w_2(x) - w_2(x)\\
& = w_1(x) > 0
\end{align*}

\item
Sei $x \notin \widebar{\Omega_1}, x \notin \widebar{\Omega_2}$. $\Rightarrow w_1(x) < 0, w_2(x) < 0$
\begin{align*}
&w(x) = w_1(x) + w_2(x) + \sqrt{w_1(x)^2 + w_2(x)^2} \overset{!}{<} 0\\
&\Leftrightarrow -w_1(x) - w_2(x) > \sqrt{w_1(x)^2 + w_2(x)^2}\\
&\Leftrightarrow w_1(x)^2 + \underbrace{2w_1(x) w_2(x)}_{>0} +w_2(x)^2 > w_1(x)^2 + w_2(x)^2
\end{align*}

\item
Sei \ac{oBdA} $x \in \partial \Omega_1, x \notin \Omega_2$. $\Rightarrow w_1(x) = 0, w_2(x) \leq 0$
\begin{align*}
w(x) &=  w_1(x) + w_2(x) + \sqrt{w_1(x)^2 + w_2(x)^2}\\
&= w_2(x) + \sqrt{w_2(x)^2} \\
&= w_2(x) - w_2(x) = 0
\end{align*}

\item
Sei \ac{oBdA} $x \in \partial \Omega_1, x \in \Omega_2$. $\Rightarrow w_1(x) = 0, w_2(x) > 0$
\begin{align*}
w(x) &= w_1(x) + w_2(x) + \sqrt{w_1(x)^2 + w_2(x)^2} \\
&= w_2(x) + \sqrt{w_2(x)^2} > 0
\end{align*}
\end{itemize}

\item
\begin{itemize}
\item
Sei $x \in \Omega_1, x \in \Omega_2$. $\Rightarrow w_1(x) >0, w_2(x) >0$
\begin{align*}
&w(x) = w_1(x) + w_2(x) - \sqrt{w_1(x)^2 + w_2(x)^2} \overset{!}{>}0\\
&\Leftrightarrow w_1(x) + w_2(x) > \sqrt{w_1(x)^2 + w_2(x)^2}\\
&\Leftrightarrow (w_1(x) + w_2(x))^2 > w_1(x)^2 + w_2(x)^2\\
&\Leftrightarrow w_1(x)^2 + 2w_1(x) w_2(x) + w_2(x)^2 > w_1(x)^2 + w_2(x)^2\\
&\Leftrightarrow 2w_1(x) w_2(x) >0
\end{align*}

\item
Sei \ac{oBdA} $x \in \Omega_1, x \notin \widebar{\Omega_2}$. $\Rightarrow w_1(x) >0, w_2(x) <0$
\begin{align*}
w(x) &= w_1(x) + w_2(x) - \sqrt{w_1(x)^2 + w_2(x)^2} \\
&< w_1(x) + w_2(x) - \sqrt{w_1(x)^2} \\
&= w_1(x)+w_2(x)-w_1(x) \\
&= w_2(x) <0
\end{align*}
\item
Sei \ac{oBdA} $x \in \partial \Omega_1, x \in \Omega_2$. $w_1(x) = 0, w_2(x) > 0$
\begin{align*}
w(x) &= w_1(x) + w_2(x) - \sqrt{w_1(x)^2+w_2(x)^2} \\
&= w_2(x) - \sqrt{w_2(x)^2} = 0
\end{align*}
\item
Sei \ac{oBdA} $x \in \partial \Omega_1, x \notin \widebar{\Omega_2}$. $w_1(x) = 0, w_2(x) < 0$
\begin{align*}
w(x) &= w_1(x) + w_2(x) - \sqrt{w_1(x)^2 + w_2(x)^2} \\
&= w_2(x) - \sqrt{w_2(x)^2} \\
&= 2w_2(x) < 0
\end{align*}
\item
Sei $x \notin \widebar{\Omega_1}, x \notin \widebar{\Omega_2}$. $\Rightarrow w_1(x) < 0, w_2(x) < 0$
\begin{align*}
w(x) = w_1(x) + w_2(x) - \sqrt{w_1(x)^2 + w_2(x)^2} < 0
\end{align*}

\item
Sei $x \in \partial \Omega_1, x \in \partial \Omega_2$. $\Rightarrow w_1(x) = 0, w_2(x) = 0$
\begin{align*}
w(x) = w_1(x) + w_2(x) - \sqrt{w_1(x)^2 + w_2(x)^2} = 0
\end{align*}
\end{itemize}
\end{enumerate}
\end{proof}

\begin{example}
\label{ex:Gewicht}
Sei $\Omega = (-1,1) \times (-1,1)$. Dann können wir $\Omega$ schreiben als $\Omega = \Omega_1  \cap \Omega_2$ mit $\Omega_1 = (-1,1) \times (- \infty, \infty), \Omega_2 =   (- \infty, \infty) \times (-1,1)$.
Dann sind 
\begin{align*}
w_1(x,y) &= -x^2 +1\\
w_2(x,y) &= -y^2 +1 
\end{align*}
Gewichtsfunktionen auf $\Omega_1$ bzw, $\Omega_2$. Nach Satz \ref{thm:Gewicht} ist dann die Gewichtsfunktion für $\Omega$ gegeben durch:
\begin{align*}
w(x,y) &= w_1(x,y) + w_2(x,y) - \sqrt{w_1(x,y)^2 + w_2(x,y)^2}\\
&= -x^2 +1 -y^2 +1 - \sqrt{(-x^2+1)^2 + (-y^2+1)^2}\\
&= -x^2-y^2+2 - \sqrt{x^4 -2x^2 + y^4 -2y^2+2}
\end{align*}
\end{example}
Wir wollen jetzt einen Kern und eine Gewichtsfunktion verknüpfen und bekommen damit eine neue Funktion, die auf dem Rand unseres Definitionsgebiets konstant Null ist. Dazu betrachten wir wieder zwei verschiedene Ansätze.

\section{Symmetrische Kollokation}
\begin{theorem}
\label{thm:gewichtKern}
Sei $\Omega$ eine Menge, $K':\Omega \times \Omega \rightarrow \mathbb{R}$ ein \ac{SPD} Kern und $g:\Omega \rightarrow \mathbb{R} \setminus \{0\}$ eine Funktion. Dann ist 
\begin{align*}
K(x,y) := g(x)K'(x,y)g(y)
\end{align*}
ein \gls{SPD} Kern und es gilt für den entsprechenden \ac{RKHR}:
\begin{align*}
\mathcal{H}_{K}(\Omega) = g \mathcal{H}_{K'}(\Omega) := \left\{ gf|f \in \mathcal{H}_K'(\Omega)\right\}
\end{align*}
\end{theorem}
\begin{proof}

Wir zeigen zunächst, dass $K(x,y)$ ein \ac{SPD} Kern ist.

Die Symmetrie erhalten wir mit
\begin{align*}
K(x,y)= g(x)K'(x,y)g(y) = g(y) K'(y,x) g(x) = K(y,x).
\end{align*}

Für die strikte positive Definitheit betrachten wir eine Punktmenge \\
$X_N := \{x_i \in \Omega| 1 \le i \le N \}\subset \Omega$. Wir erhalten für die Kernmatrix:
\begin{align*}
A_K &= 
\begin{pmatrix}
g(x_1)K'(x_1, x_1)g(x_1) & \cdots & g(x_1)K'(x_1, x_N)g(x_N) \\ 
\vdots & \ddots & \vdots \\ 
g(x_N)K'(x_N, x_1)g(x_1) & \cdots & g(x_N)K'(x_N,x_N)g(x_N)
\end{pmatrix}\\
&=
\begin{pmatrix}
g(x_1) &  &  \\ 
& \ddots &  \\ 
&  & g(x_N)
\end{pmatrix} 
\begin{pmatrix}
K'(x_1, x_1) & \cdots & K'(x_1, x_N) \\ 
\vdots & \ddots & \vdots \\ 
K'(x_N, x_1) & \cdots & K'(x_N,x_N)
\end{pmatrix}
\begin{pmatrix}
g(x_1) &  &  \\ 
& \ddots &  \\ 
&  & g(x_N)
\end{pmatrix} \\
&=: D A_{K'} D
\end{align*}
Die Matrix $D$ ist symmetrisch und hat vollen Rang, da $g(x) \neq 0, x \in \Omega$. Daraus folgt für alle $\alpha \neq 0$
\begin{align*}
	\alpha^T D A_{K'} D \alpha = (D \alpha)^T A_{K'} (D \alpha) > 0,
\end{align*}
da $A_{K'}$ positiv definit ist und $D \alpha \neq 0$ aufgrund des vollen Ranges von $D$. Also ist $K$ ein \ac{SPD} Kern.

Es fehlt noch der zweite Teil des Satzes. Dafür stellen wir zunächst fest, dass für alle $y \in \Omega$
\begin{align*}
K(\cdot,y) = g(\cdot) K'(\cdot,y) g(y) \in g \mathcal{H}_K' (\Omega).
\end{align*}
Als nächstes zeigen wir, dass $\mathcal{H}_K(\Omega)$ tatsächlich ein Hilbertraum ist. Sei dafür
\begin{align*}
s : \mathcal{H}_{K'} (\Omega) &\rightarrow g\mathcal{H}_{K'} (\Omega)\\
f &\mapsto gf
\end{align*}
$s$ ist bijektiv, da $g \neq 0$ ist. Damit können wir auf $\mathcal{H}_K (\Omega)$ ein Skalarprodukt definieren:
\begin{align*}
\left(\cdot, \cdot \right)_{\mathcal{H}_K(\Omega)} : \mathcal{H}_K(\Omega) \times \mathcal{H}_K(\Omega) &\rightarrow \mathbb{R}\\
(gf, gh) &\mapsto \left(s^{-1}(gf), s^{-1}(gh)\right)_{\mathcal{H}_{K'}(\Omega)} = \left( f,h \right)_{\mathcal{H}_{K'}(\Omega)}
\end{align*}
Damit wird $\mathcal{H}_K(\Omega)$ zu einem Hilbertraum. 

Wir zeigen noch die Reproduzierbarkeit auf $\mathcal{H}_K(\Omega)$, dann folgt aus der Eindeutigkeit des Kerns aus Satz \ref{thm:EindeutigkeitKern} die Behauptung. Sei dafür $x \in \Omega$ und $h = gf \in \mathcal{H}_K (\Omega)$.
\begin{align*}
\left(h, K(\cdot,x) \right)_{\mathcal{H}_K(\Omega)} &= \left(gf, gK'(\cdot, x) g(x)\right)_{\mathcal{H}_K(\Omega)}\\
&= g(x) \left( gf, gK'(\cdot, x)\right)_{\mathcal{H}_K(\Omega)}\\
&= g(x) \left( f, K'(\cdot, x)\right)_{\mathcal{H}_{K'}(\Omega)}\\
&= g(x) f(x)\\
&= h(x)
\end{align*}
\end{proof}

Hiermit haben wir einen neuen Kern konstruiert, der auf dem Rand unseres Definitionsgebiets konstant Null ist, auf den wir eine analoge Konstruktion zu der aus Kapitel \ref{sec:SymKol} anwenden können.

Dafür sei $\Omega \subset \mathbb{R}^d$ offen und beschränkt, $L$ ein Differentialoperator der Ordnung $k$, $K' \in C^{2k}(\Omega \times \Omega)$ ein \ac{SPD} Kern, $g \in C^k(\Omega)$ eine Gewichtsfunktion auf $\Omega$ und folgende \ac{PDE} gegeben:
\begin{align*}
Lu(x) &= f(x), x \in \Omega\\
u(x) &= 0 , x \in \partial \Omega
\end{align*}
Für ein $N \in \mathbb{N}$ betrachten wir eine Menge $X_N := \left\{ x_i \right\}_{i=1}^N \subset \Omega$. Wir definieren die Menge $\Lambda_N := \left\{ \lambda_1, \dots, \lambda_N\right\}$ mit $\lambda_i :=  \delta_{x_i} \circ L$. Diese Funktionale sind im von \\$K(x,y) := g(x) K'(x,y) g(y)$ erzeugten \ac{RKHR} stetig. Dieser Kern ist nach Satz \ref{thm:gewichtKern} \ac{SPD}, da $g(x) > 0$ für $x \in \Omega$. Also wählen wir 
\begin{align*}
V_N :&= \text{span} \left\{\lambda_1^y K(x,y), \dots, \lambda_N^y K(x,y)\right\}\\
&= \text{span} \left\{(\delta_{x_1} \circ L)^y (g(x) K'(x,y) g(y)), \dots, (\delta_{x_N} \circ L)^y (g(x) K'(x,y) g(y))\right\}
\end{align*}
als Ansatzfunktionen. Diese sind auf dem Rand von $\Omega$ gleich Null, die Randbedingung der \gls{PDE} ist also erfüllt.

Damit erhalten wir folgenden Interpolanten:
\begin{align*}
s_u(x) &= \sum_{j=1}^N \alpha_j \lambda_j^y K(x,y)\\
&= \sum_{j=1}^N \alpha_j (\delta_{x_j} \circ L)^y( g(x)K'(x,y)g(y))
\end{align*}

Die $\alpha_j$ erhält man als Lösung des \ac{LGS} $A\alpha = b$ mit 
\begin{align*}
A_{i,j} &= (\delta_{x_i} \circ L)^x (\delta_{x_j} \circ L)^y (g(x)K'(x,y)g(y))\\
b_i &= f(x_i)
\end{align*}

Die Matrix $A$ ist wieder symmetrisch und positiv definit und das \ac{LGS} ist damit lösbar.
\section{Nicht-Symmetrische Kollokation}
Wie bei der Standardkollokation können wir einen wesentlich simpleren Ansatz wählen. Es sei die gleiche Problemstellung wie gerade gegeben. Wir wählen 
\begin{align*}
V_N:= \text{span} \left\{g(x)K'(x,x_1), \dots, g(x)K'(x,x_N)\right\}
\end{align*}
als Ansatzfunktionen und bekommen damit folgenden Interpolanten:
\begin{align*}
s_u (x) = \sum_{j=1}^N \alpha_j g(x)K'(x,x_j)
\end{align*}
Die $\alpha_j$ erhält man als Lösung des \ac{LGS} $A\alpha = b$ mit 
\begin{align*}
A_{i,j} &= (\delta_{x_i} \circ L)^x (g(x) K(x,x_j))\\
b_i &= f(x_i)
\end{align*}

Erneut kann man keine Aussage über die Lösbarkeit des \ac{LGS} oder über die numerische Stabilität des Verfahrens treffen.