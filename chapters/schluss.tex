\chapter{Zusammenfassung und Ausblick}
\label{cha:schluss}

In diesem Kapitel fassen wir die wichtigsten Erkenntnisse dieser Arbeit zusammen und zeigen interessante weiterführende Fragen auf.

Wir haben in Kapitel \ref{cha:Grundlagen} und \ref{cha:Standardkollokation} die bekannte Theorie der Kernkollokation vorgestellt, welche wir in Kapitel \ref{cha:Gewichtet} um Gewichtsfunktionen erweitert haben. Diese könnte man noch weiter untersuchen, auch im Hinblick darauf, Gewichtsfunktionen für beliebige Gebiete mit möglichst wenig Singularitäten in ihren Ableitungen zu finden. Als Ansatz dafür bietet sich die in Kapitel \ref{sec:andereGewicht} experimentell vorgestellte Methode an, in der wir eine Gewichtsfunktion nur auf einer Umgebung gegeben hatten.

Daraufhin haben wir in Kapitel \ref{cha:Implementierung} eine Implementierung der theoretisch hergeleiteten Verfahren vorgestellt, welche wir in Kapitel \ref{cha:NumerischeTests} ausgewertet haben. Dort haben wir sehr verschiedene Ergebnisse für unterschiedliche \acp{PDE} erhalten. Insbesondere haben die gewichteten Verfahren teilweise sehr schlechte Ergebnisse geliefert. Diese Beobachtung haben wir auf die Singularitäten der partiellen Ableitungen der Gewichtsfunktionen zurückgeführt.

Wir haben außerdem verschiedene Möglichkeiten der Kollokationspunktwahl untersucht. Dabei haben wir festgestellt, dass eine Greedy-Punktwahl bezüglich der Anzahl der Kollokationspunkte die geschickteste der vorgestellten Wahlmöglichkeiten ist. Hier wäre es noch interessant einen anderen Fehlerschätzer zur Punktwahl zu benutzen, beispielsweise mit den $C^1, C^2, \dots$ Normen. Außerdem wäre eine Greedy-Punktwahl wünschenswert, bei der die Punkte auf dem Rand nicht festgesetzt sind, sondern auch erst nach und nach gesetzt werden. Dafür könnte man auf dem Rand einen zweiten Fehlerschätzer einführen und diesen mithilfe einer Konstante gegen den Fehler im Inneren abwägen.

Wir haben die Kernkollokation zum Großteil nur in zwei Dimensionen getestet, haben aber gesehen, dass einer Umsetzung in höheren Dimensionen nichts im Wege steht. Diesen Teil haben wir allerdings nur kurz angeschnitten und man könnte dort noch weitere Experimente durchführen, um auf eine größere Allgemeingültigkeit zu schließen.